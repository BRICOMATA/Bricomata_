----------  *PROBABILITY DENSITY FUNCTIONS* ----------

          Probability Density Functions
            for Various Distributions

 - - - - - - - - - - - - - - - - - - - - - - - - -
              Symmetric Distributions


Uniform          f(x) = 1    with x in (0,1)

Normal           f(x) = (1/sqrt(2pi)) * exp(-0.5*x**2)

Logistic         f(x) = exp(x) / (1 + exp(x))**2

Double Exp.      f(x) = 0.5 * exp(-x)

Cauchy           f(x) = (1/pi) * 1/(1+x**2)

Tukey Lambda     f(x) not in closed form          
Tukey Lambda          lambda = 0.5 --U-shaped
Tukey Lambda          lambda = 1.0 --exactly uniform
Tukey Lambda          lambda = 0.14--approximately normal
Tukey Lambda          lambda = 0.0 --exactly logistic
Tukey Lambda          lambda = -1.0--approximately Cauchy

Anglit           f(x) = sin(2x+pi/2)    with x in (-pi/4,pi/4)

Triangular       f(x) = 1 - abs(x)   with x in (-1,1)

Arcsin           f(x) = (1/pi) * 1 / sqrt(x(1-x))

Student t        f(x) = c / [1 + (x*x/nu)]**((nu+1)/2)

 - - - - - - - - - - - - - - - - - - - - - - - - -
                Skewed Distributions

Chi-squared      f(x) = c * [x**((nu/2)-1)] * exp(-x/2) 
Chi-squared             with x >= 0 and nu > 0
Chi-squared             and where c = gamma function of nu/2

Gamma            f(x) = c * [x**(gamma-1)] * exp(-x) 
Gamma                   with x >= 0 and gamma > 0
Gamma                   and where c = gamma function of gamma

Exponential      f(x) = exp(-x)       with x >= 0

Lognormal        f(x) = (1/(x*sqrt(2*pi))) * exp(-0.5*(log(x))**2)
Lognormal        with x >= 0

Half-Normal      f(x) = (2/sqrt(2*pi)) * exp(-0.5*x**2)
Half-Normal      with x >= 0
Half-Normal             where GN(p) is normal N(0,1) ppf

Extreme Value I  f(x) = exp(-x) * exp(-exp(-x))

Extreme Value II f(x) = gamma * x**(-gamma-1) * exp(-(x**(-gamma))
Extreme Value II        with x >= 0 and gamma > 0

Weibull          f(x) = gamma * x**(gamma-1) * exp(-(x**gamma))
Weibull                 with x > 0 and gamma > 0

Pareto           f(x) = gamma / (x**(gamma+1))

Beta             f(x) = c * x**(a-1) * (1-x)**(b-1) with x in (0,1)
Beta                    where c = beta function of a and b

 - - - - - - - - - - - - - - - - - - - - - - - - -
               Discrete Distributions

Binomial

Geometric

Poisson

