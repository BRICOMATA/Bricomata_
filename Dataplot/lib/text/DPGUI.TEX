Dataplot Graphical User Interface   (Spring, 1995)
James J. Filliben, Statistical Engineering Division, CAML
 
Dataplot Menu Subsystem
 
The experienced Dataplot user usually runs Dataplot
in a command mode--sequentially entering commands, generating
output, and repeating as needed.
The beginning Dataplot user
who has no familiarity with the Dataplot commands
and/or conventions may alternatively choose to execute
Dataplot from a parallel point-and-click GUI menu mode.
Aside from adding structure to the vast collection
of Dataplot commands, running in the menu mode also
automatically forms intermediate Dataplot commands
and displays them to the analyst if interested.
To enter the menu subsytem while at the usual Dataplot prompt,
the analyst simply enters MENU.
 
The GUI is written in Turbo-C;
Dataplot's historic Fortran base has been only minimally
distrubed with but a single link bridging the Fortran and C.
In effect, the front end and back end are independent of one another.
Design Philosophy:
   The front end must be independent of the back end
   The GUI must be easily extensible
      (=> file-driven)
      The GUI must be data-centric
         (400+ data files on-line)
      The GUI must be literature-based
         (links to texts)
   The GUI must be graphical:
      Every quantitative solution has a better graphical alternative
         (confidence limits => confidence distributions)
The menu system must self-teach by its structure
Not all users are the same, therefore
   the menu system must be both problem-centric
   and technique-centric
Hypertext file structure
Stat-specific menus
Engineering unit based: sigma, not sigma-squared
 
Advantages
   1. File-driven menu system (extensible, portable)
   2. Multi-level user expertise (commands => menus)
   3. Multi-entry: data/technique/goal
   4. multi-SED-member expertise
   5. optional (ignorable)
   6. file-connectable (hypertext: submenu, menu, literature)
   7. dovetails nicely with Alan documentation
   8. some areas: quite extensive (e.g., probability)
   9. some areas: unique, new techiques (block plot, DDS, DEX)
  10. NIST-sensitive menu structure (cv, calibration)
  11. Graphics-representation of results (e.g., t test conf. limits)
  12. data-centric
  13. data from books & courses
  14. stat-specific menus (Windows generic menu top menus too general)
  15. Graphics output (e.g., confidence distributions)
  16. Engineering-based: sigma, not sigma squared
 
  17. extensive data resources
  18. high-level roadmaps (via rooms)
  19. connectable to literature (e.g., Natrella)
  20. built-in extensible point and click menus for techniques
  21. Postscript-graphics-based
 
Disadvantages
   1. Turbo-C (not Microsoft Windows)
   2. Under construction items
 
Links with other NIST projects:
   1. HPCC/SIMA
   2. Coatings Consortium
   3. SEMATECH electronic handbook
   4. DEX Workshops
   5. future Stat for Scientist Courses
   6. Graybill/Iyer textbook
   7. Jeff Fong Multimedian conferencing via SCOOT
