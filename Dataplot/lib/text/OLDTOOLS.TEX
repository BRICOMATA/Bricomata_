----------  *7 (OLD) TOOLS*  ----------

THE 7 (OLD) TOOLS FOR PROBLEM-SOLVING IN TOTAL QUALITY CONTROL (TQC)

1. Check Sheet (To Categorize the Customer Causes of Rejection).
   To promote the collection of purposeful, reliable
   data by making data easy to obtain and easy to use.
   Usually used at Inspection Stage to accumulate
   rejection causes by various types.  This answers the
   question "Why was each item rejected?"
   Example--Note and accumulate how many defective
   springs are from pinholes, dimensions, cracking,
   scratching, other.

2. Pareto Diagram (To Order the Customer Causes of Rejection).
   To indicate which problem should be solved first to
   eliminate defects and improve the operation.  This
   orders the causes-of-rejection.  Example--the
   ordering of cause-of-rejection is cracking, scratching,
   pinholes, dimensions, other.  Note what customer cause-
   of-rejection is most frequent.

3. Cause and Effect Diagram (Fishbone Diagram, Ishikawa Diagram).
   To relate a selected customer requirement to
   various measurable product characteristics; also, 
   to relate a selected measurable product 
   characteristic to various input/process/output 
   characteristics.  To categorize and enumerate the 
   causes for quality dispersion (4 general 
   categories--material, machine, man, method).  It is
   a useful graphical technique to assist engineering 
   brainstorming to relate a selected 
   requirement/characteristic to various 
   component/product/process characteristics.  
   Example--cracking may be due to input materials, 
   machine, man (shifts), method (assembly process), 
   hardening temperature, quenching temperature, the 
   inspection process itself, the output spring type, 
   etc.

4. Histogram (of Some Primary Product Characteristic).
   To show the central value, dispersion, and shape of
   the distribution of measurements from a production
   line.  Example--if cracking is the primary cause of
   rejection, then examine the distribution of some
   physical property of cracking, such as length, depth,
   location, etc.

5. Stratification (of Histograms).
   Examine the distribution as a function of various
   factors and levels within factors.  Example--compare
   the distribution of the size of cracks for spring
   type A with the distribution of size of cracks for
   spring type B.  A bihistogram may be useful.

6. Scatter Diagram.
   To show the nature of the relationship between 2
   kinds of data (variables).  This is really more
   stratification--but by scatter plot rather than by
   by histogram.  Example--how does crack size vary as
   a function of some continuous variable (such as
   oven temperature)?

7. Graphs and Charts (including Control Charts).
   To determine if material, man, machine, and method
   effect the production output over time, and to
   determine if the process is statistically "in
   control".  Example--if crack size is related to
   oven temperature, what does the control chart of
   temperature reveal in terms of instability & unusual
   patterns.


   Source  --Total Quality Control (TQC): The Japanese Way.
             Q. C. Trends, August 1985, pages 9-10.
   See Also--Box & Bisgaard (1987). The Scientific Context
             of Quality Improvement.  Quality Progress,
             page 55-56.
