500.      0           0. Initial Menu
1000.     1               1. Univariate Analysis
1028.     1.1                   1. General Discussion
1037.     1.2                   2. Computing Summary/Descriptive Statistics
1051.     1.2.1                       1. Location Estimation
1078.     1.2.2                       2. Variation (Scale) Estimation
1106.     1.2.3                       3. Skewness Estimation
1133.     1.2.4                       4. Tail Length Estimation
1156.     1.2.5                       5. Autocorrelation estimation
1182.     1.3                   3. Determining General Distributional Characteri
1201.     1.3.1                       1. Frequency Tabulation
490.      1.3.2                       2. Histograms and Cumulative Histograms
490.      1.3.3                       3. Stem and Leaf Diagrams
490.      1.3.4                       4. Frequency Plots and Cumulative Frequenc
490.      1.3.5                       5. Percent Point Plots
490.      1.3.6                       6. Pie Charts
490.      1.4                   4. Selecting a "Good-Fitting" Distribution
490.      1.4.1                       1. Probability Plots
490.      1.4.2                       2. PPCC Plots
490.      1.4.3                       3. Maximum Likelihood Estimation
490.      1.5                   5. Estimating the Parameters of the Distribution
490.      1.5.1                       1. Maximum Likelihood Estimation
490.      1.5.2                       2. Robust Estimates
490.      1.6                   6. Assessing the Goodness of Fit of a Distributi
490.      1.6.1                       1. Superimposing Probability Density Funct
490.      1.6.2                       2. Superimposing Root Density Function and
490.      1.6.3                       3. Probability Plot
490.      1.6.4                       4. Chi-Squared Test
490.      1.6.5                       5. Kolmogorov-Smirnoff Test
490.      1.7                   7. Testing Underlying Assumptions
490.      1.7.1                       1. 4-Plot Analysis
490.      1.8                   8. Testing for Randomness
490.      1.8.1                       1. Lag Plot
490.      1.8.2                       2. Runs Analysis
490.      1.8.3                       3. Distribution-Free Tests
490.      1.8.4                       4. Autocorrelation Plot
490.      1.8.5                       5. Spectral Plot
490.      1.9                   9. Testing for Fixed Location (No Shifts in Loca
490.      1.9.1                       1. t Test
490.      1.9.2                       2. Distribution-Free Tests
490.      1.10                 10. Testing for Fixed Variation (Homoscedasticity
490.      1.10.1                      1. Homoscedasticity Plot
490.      1.11                 11. Transforming to Homoscedasticity
490.      1.11.1                      1. Box-Cox Homoscedasticity Plot
490.      1.11.2                      2. Chi-squared Tests
490.      1.12                 12. Testing for Fixed Distribution
490.      1.12.1                      1. Bihistogram
490.      1.12.2                      2. 4-Plot Analysis
490.      1.12.3                      3. Distribution-Free Tests
490.      1.12.4                      4. Homoscedasticity Plot
490.      1.13                 13. Testing for Symmetry
490.      1.13.1                      1. Symmetry Plot
490.      1.14                 14. Transforming to Symmetry
490.      1.14.1                      1. Box-Cox Symmetry Plot
490.      1.15                 15. Testing for Normality
490.      1.15.1                      1. Normal Probability Plot
490.      1.15.2                      2. Tukey PPCC Plot
490.      1.15.3                      3. t PPCC Plot
490.      1.16                 16. Transforming to Normality
490.      1.16.1                      1. Box-Cox Normality Plot
490.      1.17                 17. Testing for Normal Outliers
490.      1.18                 18. Computing Confidence Limits for Distributiona
490.      1.19                 19. Hypothesis Testing on Distributional Paramete
1500.     2               2. Time Series Analysis (1 Variable)
490.      2.1                   1. General discussion
490.      2.2                   2. Checking for Time-Domain Structure
490.      2.2.1                       1. Run Sequence Plot
490.      2.2.2                       2. Lag Plot
490.      2.2.3                       3. Autocorrelation Plot
490.      2.2.4                       4. Partial-Autocorrelation Plot
490.      2.2.5                       5. Complex Demodulation Plots
490.      2.3                   3. Checking for Frequency-Domain Structure
490.      2.3.1                       1. Spectral Plot
490.      2.3.2                       2. Periodogram
490.      2.4                   4. Checking for Time and Frequency Domain Struct
490.      2.4.1                       1. 4-Plot Analysis
490.      2.5                   5. Testing White Noise (Randomness)
490.      2.5.1                       1. Lag Plot
490.      2.5.2                       2. Runs Analysis
490.      2.5.3                       3. Distribution-Free Tests
490.      2.5.4                       4. Autocorrelation Plot
490.      2.5.5                       5. Spectral Plot
490.      2.5.6                       6. 4-Plot Analysis
490.      2.6                   6. Checking for Trends
490.      2.6.1                       1. Run Sequence Plot
490.      2.6.2                       2. Correlation With Time
490.      2.6.3                       3. Linear Fit Over Time
490.      2.7                   7. Fitting Box-Jenkins Models
490.      2.7.1                       1. Lag Plot
490.      2.7.2                       2. Autocorrelation Plot
490.      2.7.3                       3. Partial Autocorrelation Plot
490.      2.8                   8. Smoothing
490.      2.8.1                       1. Moving Average Smoothing
490.      2.8.2                       2. Least Squares Smoothing
490.      2.8.3                       3. Median Smoothing
490.      2.8.4                       4. Robust Smoothing
490.      2.8.5                       5. Exponential Smoothing
490.      2.8.6                       6. Assessing the Goodness of the Smoothing
490.      2.8.6.1                           1. Residual Standard Deviation
490.      2.8.6.2                           2. Superimposing Raw Data and Fitted
490.      2.8.6.3                           3. Scatter Plots of Residuals
490.      2.8.6.4                           4. Normal Probability Plot of Residu
490.      2.8.6.5                           5. 5-Plot of Residuals
490.      2.9                   9. Filtering
490.      2.9.1                       1. Low-Pass Filters
490.      2.9.2                       2. High-Pass Filters
490.      2.9.3                       3. Assessing the Goodness of the Filtering
1700.     3               3. Time Series Analysis--2 Variables
490.      3.1                   1. General Discussion
490.      3.2                   2. Checking for Time-Domain Structure
490.      3.2.1                       1. Scatter Plot
490.      3.2.2                       2. Multi-Trace Plots
490.      3.2.3                       3. Cross-Spectral Plot
490.      3.2.4                       4. Bihistogram
490.      3.3                   3. Checking for Frequency-Domain Structure
490.      3.3.1                       1. Cross-Spectrum
490.      3.3.2                       2. Coherency Spectrum
490.      3.3.3                       3. Quadrature Spectrum
490.      3.3.4                       4. Co-Spectrum
490.      3.3.5                       5. Gain Spectrum
490.      3.3.6                       6. Argand Spectrum
490.      3.4                   4. Checking for Time and Frequency Domain Struct
490.      3.4.1                       1. 4-Plot Analysis
1800.     4               4. Correlation Analysis
490.      4.1                   1. General Discussion
490.      4.2                   2. Multi-Scatter Plots
490.      4.3                   3. Multi-ANOP Plots
490.      4.4                   4. Multi-Box Plots
490.      4.5                   5. Cross-Correlation Tabulation
490.      4.6                   6. Transforming Variables
490.      4.7                   7. Distribution-free Tests
1900.     5               5. Fitting (1 Independent Variable)
1917.     5.1                   1. General discussion
1954.     5.2                   2. Selecting a Model
1979.     5.2.1                       1. Plotting the Data
2009.     5.2.2                       2. Generate Reference Data Plots
2562.     5.2.2.1                           1. Shape 1--Quadratic
2592.     5.2.2.2                           2. Shape 2--Monotonic up
2622.     5.2.2.3                           3. Shape 3--Monotonic up
2652.     5.2.2.4                           4. Shape 4--S-Shaped
2682.     5.2.2.5                           5. Shape 5--Quadratic
2712.     5.2.2.6                           6. Shape 6--Linear
2742.     5.2.2.7                           7. Shape 7--Square Root
2772.     5.2.2.8                           8. Shape 8--Asymptote
2802.     5.2.2.9                           9. Shape 9--Cubic
2832.     5.2.2.10                         10. Shape 10--Monotonic
2862.     5.2.2.11                         11. Shape 11--Logarithm
2892.     5.2.2.12                         12. Shape 12--Asymptote
2922.     5.2.2.13                         13. Shape 13--Skewed
2952.     5.2.2.14                         14. Shape 14--Skewed
2982.     5.2.2.15                         15. Shape 15--Quadratic
3012.     5.2.2.16                         16. Shape 16--Skewed
3042.     5.2.2.17                         17. Shape 17--Bell-Shaped
3072.     5.2.2.18                         18. Shape 18--Hyberbolic
3102.     5.2.2.19                         19. Shape 19--Neg. Exponential
3132.     5.2.2.20                         20. Shape 20--Z-Shaped
3162.     5.3                   3. Fitting a Model
3174.     5.3.1                       1. Fitting Linear Models
490.      5.3.2                       2. Fitting Polynomial Models
490.      5.3.3                       3. Fitting Non-Linear Models
490.      5.3.4                       4. Fitting Rational Functions
3218.     5.3.5                       5. Fitting Splines
3254.     5.4                   4. Assessing the Goodness of Fit of the Model
490.      5.4.1                       1. Residual Standard Deviation
490.      5.4.2                       2. Lack of Fit F Tests
3297.     5.4.3                       3. Superimposing Raw Data and Fitted Curve
3330.     5.4.4                       4. 4-Plot of Residuals
3366.     5.4.5                       5. Scatter Plots of Residuals
3399.     5.4.5.1                           1. Reference Scatter Plots of Residu
490.      5.4.5.1.1                               1  Residual Scatter Plot--Idea
490.      5.4.5.1.2                               2  Residual Scatter Plot--Wedg
490.      5.4.5.1.3                               3  Residual Scatter Plot--Wedg
490.      5.4.5.1.4                               4  Residual Scatter Plot--Wedg
490.      5.4.5.1.5                               5  Residual Scatter Plot--Line
490.      5.4.5.1.6                               6  Residual Scatter Plot--Quad
490.      5.4.5.1.7                               7  Residual Scatter Plot--Quad
3670.     5.4.5.1.8                               8  Residual Scatter Plot--Spli
490.      5.4.6                       6. Lag Plot of Residuals
490.      5.4.7                       7. Histograms of Residuals
490.      5.4.8                       8. Normal Probability Plot of Residuals
490.      5.5                   5. Improving the Model
490.      5.5.1                       1. Transforming to Simplify the Model
490.      5.5.2                       2. Transforming to Achieve Homogeneity
490.      5.5.3                       3. Transforming to Achieve Normality
490.      5.5.4                       4. Adding New Variables
490.      5.5.5                       5. Changing the Form of the Model
490.      5.6                   6. Fitting With Weights
490.      5.7                   7. Fitting With Constraints
490.      5.7                   8. Fitting With Other Criteria (e.g., L1 Fitting
4000.     6               6. Fitting (2 or More Independent Variable)
4018.     6.1                   1. General discussion
490.      6.2                   2. Selecting Variables for To Be Included in the
490.      6.2.1                       1. Multi-Run Sequence Plots
490.      6.2.2                       2. Multi-Histograms
490.      6.2.3                       3. Multi-Scatter Plots
490.      6.2.4                       4. Multi-ANOP Plots
490.      6.2.5                       5. Cross-Correlation Tabulation
490.      6.2.6                       6. Box Plots
490.      6.2.7                       7. Cp Plot
490.      6.3                   3. Selecting a Model
490.      6.3.1                       1. Plotting the Data
490.      6.3.2                       2. Making Use of Reference Curves
490.      6.4                   4. Fitting a Model
490.      6.4.1                       1. Fitting Multi-Linear Models
490.      6.4.2                       2. Fitting Non-Linear Models
490.      6.5                   5. Assessing the Goodness of Fit of the Model
490.      6.5.1                       1. Residual Standard Deviation
490.      6.5.2                       2. Lack of Fit F Tests
490.      6.5.3                       3. Superimposing Raw Data and Fitted Curve
490.      6.5.4                       4. Scatter Plots of Residuals
490.      6.5.5                       5. Normal Probability Plot of Residuals
490.      6.5.6                       6. 4-Plot of Residuals
490.      6.6                   6. Improving the Model
490.      6.6.1                       1. Transforming to Simplify the Model
490.      6.6.2                       2. Transforming to Achieve Homogeneity
490.      6.6.3                       3. Transforming to Achieve Normality
490.      6.6.4                       4. Adding New Variables
490.      6.6.5                       5. Changing the Form of the Model
490.      6.7                   7. Fitting With Weights
490.      6.8                   8. Fitting With Constraints
490.      6.8                   8. Fitting With Other Criteria (e.g., L1 Fitting
4200.     7               7. ANOVA Modeling
490.      7.1                   1. General discussion
490.      7.2                   2. Selecting Variables for To Be Included in the
490.      7.2.1                       1. Multi-Run Sequence Plots
490.      7.2.2                       2. Multi-Histograms
490.      7.2.3                       3. Multi-Scatter Plots
490.      7.2.4                       4. Multi-ANOP Plots
490.      7.3                   3. Examining 1-Factor Models
490.      7.3.1                       1. 1-Way ANOVA
490.      7.3.2                       2. 1-Way GANOVA
490.      7.3.3                       3. Scatter Plots
490.      7.3.4                       4. Box Plot
490.      7.3.5                       5. ANOP Line Plot
490.      7.3.6                       6. ANOP Character Plot
490.      7.3.7                       7. I Plot
490.      7.3.8                       8. Distribution-free Tests
490.      7.3.9                       9. Correlation
490.      7.3.10                     10. Categorical Data Analauysis
490.      7.3.11                     11. Cross-Corrleation
490.      7.3.12                     12. Discrete Contour Plot
490.      7.3.13                     13. Frequency Tabulation
490.      7.3.14                     14. Cross-Tabulation
490.      7.3.15                     15. Chi-squared
490.      7.4                   4. Examining 2-Factor Models
490.      7.4.1                       1. 2-Way ANOVA
490.      7.4.2                       2. 2-Way GANOVA
490.      7.4.3                       3. Median Polish
490.      7.4.4                       4. Multi-Trace Plots
490.      7.4.5                       5. 3-D Plot
490.      7.4.6                       6. Spike Plots
490.      7.5                   5. Examining 3-Factor Models
490.      7.5.1                       1. 3-Way ANOVA
490.      7.5.2                       2. 3-Way GANOVA
490.      7.5.3                       3. Multi-Cell Plots
490.      7.6                   6. Examining 4-factor Models
490.      7.6.1                       1. 4-Way ANOVA
490.      7.6.2                       2. Multi-Plot 2-Way GANOVA
490.      7.6.3                       3. Multi-Plot 3-Way GANOVA
490.      7.7                   7. Examining 5-Factor Models
490.      7.7.1                       1. 5-Way ANOVA
490.      7.7.2                       2. Multi-Plot 3-Way GANOVA
490.      7.8                   8. Examining 1-Factor Models With Only 2 Treatme
490.      7.8.1                       1. 1-Way ANOVA
490.      7.8.2                       2. t Test
490.      7.8.3                       3. Bihistogram
490.      7.9                   9. Examining 2**k Models
490.      7.9.1                       1. Square Plots, Cube Plots, etc.
490.      7.10                 10. Assessing the Goodness of Fit of the Model
490.      7.10.1                      1. Residual Standard Deviation
490.      7.10.2                      2. Lack of Fit F Tests
490.      7.10.3                      3. GANOVA, Parallelism, and Non-Additivity
490.      7.10.4                      4. Superimposing Raw Data and Fitted Curve
490.      7.10.5                      5. Scatter Plots of Residuals
490.      7.10.6                      6. Normal Probability Plot of Residuals
490.      7.10.7                      7. 4-Plot of Residuals
490.      7.11                 11. Improving the Model
490.      7.11.1                      1. Residual Standard Deviations For Sub-Mo
490.      7.11.2                      2. F Tests For Sub-Models
490.      7.11.3                      3. Transforming to Simplify the Model
490.      7.11.4                      4. Transforming to Achieve Additivity
490.      7.11.5                      5. Transforming to Achieve Homogeneity
490.      7.11.6                      6. Transforming to Achieve Normality
490.      7.11.7                      7. Omitting Variables From the Model
490.      7.11.7.1                          1. F Tests for Sub-Models
490.      7.11.8                      8. Selecting Additional Variables For the
490.      7.11.8.1                          1. Scatter Plots of Residuals on New
490.      7.11.8.2                          2. Box Plots of Residuals on New Var
490.      7.11.9                      9. Changing the Form of the Model
4600.     8               8. Multivariate Analysis
490.      8.1                   1. General Discussion
490.      8.2                   2. Cluster Analysis
490.      8.3                   3. Discriminant Analysis
490.      8.4                   4. Principal Component Analysis
490.      8.5                   5. Canonical Analysis
490.      8.6                   6. Testing Multivariate Normality--Q-Q Plot
490.      8.5
4700.     9               9. Probability Analysis
490.      9.1                   1. General Discussion
490.      9.2                   2. Generating Random Numbers/Simulation/Monte Ca
490.      9.3                   3. Computing Percent Points
490.      9.4                   4. Computing Probability Density Functions
490.      9.5                   5. Computing Cumulative Distribution Functions
490.      9.6                   6. Plotting Percent Points
490.      9.7                   7. Plotting Probability Density Functions
490.      9.8                   8. Plotting Cumulative Distribution Functions
490.      9.9                   9. Superimposing Probability Density Funtions on
4800.     10             10. Quality Control
490.      10.1                  1. General Discussion
490.      10.2                  2. Testing for Trends
490.      10.2.1                      1. Run Sequence Plot
490.      10.2.2                      2. Mean Control Chart
490.      10.3                  3. Testing for Shifts in Location
490.      10.3.1                      1. Run Sequence Plot
490.      10.3.2                      2. Mean Control Chart
490.      10.4                  4. Testing for Shifts in Variation
490.      10.4.1                      1. Range Control Chart
490.      10.4.2                      2. Standard Deviation Control Chart
490.      10.5                  5. Testing for Outliers
490.      10.6                  6. Interlaboratory Testing
490.      10.6.1                      1. Youden Plots
4900.     11             11. Distribution-Free Analysis
490.      11.1                  1. General Discussion
490.      11.2                  2. Testing for Randomness
490.      11.2.1                      1. Runs Analysis
490.      11.2.2                      2. Sign Test
490.      11.2.3                      3. Median Test
490.      11.3                  2. Testing for Fixed Location (No Shifts)
490.      11.3.1                      1. Sign Test
490.      11.4                  3. Testing for Fixed Variation (Homoscedasticity
490.      11.4.1                      1. Sign Test on First Differences
490.      11.5                  4. Testing for Goodness of Fit of a Distribution
490.      11.5.1                      1. Kolmogorov-Smirnoff Test
490.      11.6                  5. Testing for Correlation
490.      11.6.1                      1. Rank Correlation Coefficient
490.      11.7                  6. ANOVA Modeling
490.      11.7.1                      1. ANOVA On Ranks
490.      11.7.2                      2. Mann-Whitney Tests
5000.     12             12. Robust Analysis
490.      12.1                  1. General Discussion
490.      12.2                  2. Univariate Estimation
490.      12.2.1                      1. Robust Location Estimation
490.      12.2.2                      2. Robust Scale Estimation
490.      12.3                  2. Smoothing
490.      12.3.1                      1. Median Smoothing
490.      12.3.2                      2. Robust Smoothing
490.      12.4                  4. Fitting
490.      12.4.1                      1. L1 Fitting
490.      12.5                  5. ANOVA Modeling
490.      12.5.1                      1. ANOVA On Ranks
490.      12.5.2                      2. Median Polish
5100.     13             13. Exploratory Data Analysis
490.      13.1                  1. General Discussion
490.      13.2                  2. Run Sequence Plots
490.      13.3                  3. Lag Plots
490.      13.4                  4. Stem and Leaf Diagrams
490.      13.5                  5. Histograms
490.      13.6                  6. Normal Probability Plots
490.      13.7                  7. Scatter Plots
490.      13.8                  8. Box Plots

































































































































----------------------------------------------------------

No menu item available.
Please enter -1 to revert
to the previous menu.





----------  *INITIAL MENU FOR DATA ANALYSIS*  ------------

0--DATA ANALYSIS

                    DATA ANALYSIS MENU

Type of data analysis problem = ?

   1. Univariate Analysis (= the Analysis of 1 Variable)
   2. Time Series Analysis (1 Variable)
   3. Time Series Analysis (2 Variable)
   4. Correlation Analysis
   5. Fitting (1 Independent Variable)
   6. Fitting (2 or More Independent Variables)
   7. ANOVA Modeling
   8. Multivariate Analysis
   9. Probability Analysis
  10. Quality Control
  11. Distribution-Free Analysis
  12. Robust Analysis
  13. Exploratory Data Analysis

To select a menu item, enter 1 through 13.

To regenerate the same menu at any time, enter 0;
to regenerate previous menus, enter -1, -2, etc.

Note that all menus within the system are uniquely
numbered.  If you do not wish to follow the
natural progression of menus as presented, you may
at any time jump to any other menu in the system.
Do this by entering EXPERT followed by the
explicit menu identifier number, as in
EXPERT 1.3.1    and    EXPERT 5.2.2.11     .

----------------------------------------------------------
















































































































































































































































































































































































































































































----------  *UNIVARIATE ANALYSIS*  -----------------------

1--UNIVARIATE ANALYSIS

The sub-categories are--

   1. General Discussion
   2. Computing Summary/Descriptive Statistics
   3. Determining General Distributional Characteristics
   4. Selecting a "Good-Fitting" Distribution
   5. Estimating the Parameters of the Distribution
   6. Assessing the Goodness of Fit of a Distribution
   7. Testing Underlying Assumptions
   8. Testing for Randomness
   9. Testing for Fixed Location (No Shifts in Location)
  10. Testing for Fixed Variation (Homoscedasticity)
  11. Transforming to Homoscedasticity
  12. Testing for Fixed Distribution
  13. Testing for Symmetry
  14. Transforming to Symmetry
  15. Testing for Normality
  16. Transforming to Normality
  17. Testing for Normal Outliers
  18. Computing Confidence Limits for Distributional Parameters
  19. Hypothesis Testing on Distributional Parameters

To select a menu item, enter 1 to 19.

----------------------------------------------------------

1.1--GENERAL DISCUSSION






----------------------------------------------------------

1.2--COMPUTING SUMMARY/DESCRIPTIVE STATISTICS

Type of estimation = ?

   1. Location Estimation
   2. Variation (Scale) Estimation
   3. Skewness Estimation
   4. Tail Length Estimation
   5. Autocorrelation estimation

To select a menu item, enter 1 through 5.

----------------------------------------------------------

1.2.1--LOCATION ESTIMATION

To compute individual statistics, use the LET
command, as in

   LET A = MEAN Y
   LET B = MEDIAN Y
   LET C = MIDRANGE Y
   LET D = MIDMEAN Y

To compute a battery of location statistics
(as well as statistics for scale, skewness,
tail length, and autocorrelation), use the
SUMMARY command, as in

   SUMMARY Y

Note that the choice of the optimal location
estimator is heavily dependent on the actual
underlying distribution for the data.  For
guidance on how to check for the underlying
distribution, revert to the second-previous
menu (enter -2), and select the menu items
dealing with distributional estimating/checking.

----------------------------------------------------------

1.2.2--VARIATION (SCALE) ESTIMATION

To compute individual statistics, use the LET
command, as in

   LET A = RANGE Y
   LET B = VARIANCE Y
   LET C = STANDARD DEVIATION Y
   LET D = RELATIVE STANDARD DEVIATION
   LET E = STANDARD DEVIATION OF THE MEAN Y

To compute a battery of scale statistics
(as well as statistics for location, skewness,
tail length, and autocorrelation), use the
SUMMARY command, as in

   SUMMARY Y

Note that the choice of the optimal scale
estimator is heavily dependent on the actual
underlying distribution for the data.  For
guidance on how to check for the underlying
distribution, revert to the second-previous
menu (enter -2), and select the menu items
dealing with distributional estimating/checking.

----------------------------------------------------------

1.2.3--SKEWNESS ESTIMATION

To compute individual statistics, use the LET
command, as in

   LET A = THIRD CENTRAL MOMENT Y
   LET B = SKEWNESS Y

To compute a battery of location, scale,
skewness, tail length, and autocorrelation
statistics, use the SUMMARY command, as in

   SUMMARY Y

For guidance on how to test for symmetry,
and how to transform to symmetry, revert
to the second previous menu (enter -2).

The skewness statistics give information
about only one aspect of the underlying
distribution of the data.  For guidance on
how to check for other aspects of the
distribution, see also the second
previous menu.

----------------------------------------------------------

1.2.4--TAIL LENGTH ESTIMATION

To compute individual statistics, use the LET
command, as in

   LET A = FOURTH CENTRAL MOMENT Y
   LET B = KURTOSIS Y

To compute a battery of location, scale,
skewness, tail length, and autocorrelation
statistics, use the SUMMARY command, as in

   SUMMARY Y

The tail length statistics give information
about an important aspect of the underlying
distribution of the data.  For guidance on
how to further check on what the underlying
distribution is, see the second previous
menu.

----------------------------------------------------------

1.2.5--AUTOCORRELATION ESTIMATION

To compute individual statistics, use the LET
command, as in

   LET A = AUTOCORRELATION Y
   LET B = AUTOCOVARIANCE Y

To compute a battery of location, scale,
skewness, tail length, and autocorrelation
statistics, use the SUMMARY command, as in

   SUMMARY Y

The autocorrelation statistics give information
about an important aspect (randomness) of the
data set.  For guidance on how to check further
on the the randomness assumption, see the
second previous menu.  If the randomness
assumption does fail, then a time series analysis
of the data may be appropriate.  For guidance
on how to carry out such a time series analysis,
see the second previous menu.

----------------------------------------------------------

1.3--DETERMINING GENERAL DISTRIBUTIONAL CHARACTERISTICS

There are several data analysis techniques
which are available for giving the analyst
a "feel" for the general distributional
characteristics of the data.  Which technique
do you wish to use?

   1. Frequency Tabulation
   2. Histograms and Cumulative Histograms
   3. Stem and Leaf Diagrams
   4. Frequency Plots and Cumulative Frequency Plots
   5. Percent Point Plots
   6. Pie Charts

To select a menu item, enter 1 through 6.

----------------------------------------------------------

1.3.1--FREQUENCY TABULATION

To generate a frequency tabulation of the
data in the variable Y, use the LET command,
as in

   LET F = FREQUENCY Y

The default class width will be set to .3*s
where s is the sample standard deviation.
The first class will have lower boundary
xbar-6*s where xbar is the sample mean.
The last class will have upper boundary
xbar+6*s.  The resulting frequencies (counts)
will be automatically stored in the
variable F.

If you wish to override the default classes,
then define a variable (M, say) which contains
the desired midpoints.  This is most easily
done via the SERIAL READ command, as in

   SERIAL READ M
   .1 .2 .3 .4 .5 .6 .7
   END OF DATA

which would define the desired class midpoints
as .1, .2, through .7.  If the midpoints are
equi-spaced, then the class width will also be;
if the midpoints are not equi-spaced, then the
class widths will be variable.
After the midpoint variable is defined,
then use the LET command again, as in

   LET F = FREQUENCY Y M

To get a frequency tabulation for the
distinct values in Y, then enter

   LET M = DISTINCT Y
   LET F = FREQUENCY Y M

For guidance on how to carry out other
techniques for gaining general information
about the distribution of the data, see
the second previous menu.

----------------------------------------------------------

1.3.2--HISTOGRAMS AND CUMULATIVE HISTOGRAMS

----------------------------------------------------------

1.3.3--STEM AND LEAF DIAGRAMS

----------------------------------------------------------

1.3.4--FREQUENCY PLOTS AND CUMULATIVE FREQUENC

----------------------------------------------------------

1.3.5--PERCENT POINT PLOTS

----------------------------------------------------------

1.3.6--PIE CHARTS

----------------------------------------------------------

1.4--SELECTING A "GOOD-FITTING" DISTRIBUTION

----------------------------------------------------------

1.4.1--PROBABILITY PLOTS

----------------------------------------------------------

1.4.2--PPCC PLOTS

----------------------------------------------------------

1.4.3--MAXIMUM LIKELIHOOD ESTIMATION

----------------------------------------------------------

1.5--ESTIMATING THE PARAMETERS OF THE DISTRIBUTION

----------------------------------------------------------

1.5.1--MAXIMUM LIKELIHOOD ESTIMATION

----------------------------------------------------------

1.5.2--ROBUST ESTIMATES

----------------------------------------------------------

1.6--ASSESSING THE GOODNESS OF FIT OF A DISTRIBUTION

----------------------------------------------------------

1.6.1--SUPERIMPOSING PROBABILITY DENSITY FUNCTIONS

----------------------------------------------------------

1.6.2--SUPERIMPOSING ROOT DENSITY FUNCTION AND

----------------------------------------------------------

1.6.3--PROBABILITY PLOT

----------------------------------------------------------

1.6.4--CHI-SQUARED TEST

----------------------------------------------------------

1.6.5--KOLMOGOROV-SMIRNOFF TEST

----------------------------------------------------------

1.7--TESTING UNDERLYING ASSUMPTIONS

----------------------------------------------------------

1.7.1--4-PLOT ANALYSIS

----------------------------------------------------------

1.8--TESTING FOR RANDOMNESS

----------------------------------------------------------

1.8.1--LAG PLOT

----------------------------------------------------------

1.8.2--RUNS ANALYSIS

----------------------------------------------------------

1.8.3--DISTRIBUTION-FREE TESTS

----------------------------------------------------------

1.8.4--AUTOCORRELATION PLOT

----------------------------------------------------------

1.8.5--SPECTRAL PLOT

----------------------------------------------------------

1.9--TESTING FOR FIXED LOCATION (NO SHIFTS IN LOCATION

----------------------------------------------------------

1.9.1--T TEST

----------------------------------------------------------

1.9.2--DISTRIBUTION-FREE TESTS

----------------------------------------------------------

1.10--TESTING FOR FIXED VARIATION (HOMOSCEDASTICITY)

----------------------------------------------------------

1.10.1--HOMOSCEDASTICITY PLOT

----------------------------------------------------------

1.11--TRANSFORMING TO HOMOSCEDASTICITY

----------------------------------------------------------

1.11.1--BOX-COX HOMOSCEDASTICITY PLOT

----------------------------------------------------------

1.11.2--CHI-SQUARED TESTS

----------------------------------------------------------

1.12--TESTING FOR FIXED DISTRIBUTION

----------------------------------------------------------

1.12.1--BIHISTOGRAM

----------------------------------------------------------

1.12.2--4-PLOT ANALYSIS

----------------------------------------------------------

1.12.3--DISTRIBUTION-FREE TESTS

----------------------------------------------------------

1.12.4--HOMOSCEDASTICITY PLOT

----------------------------------------------------------

1.13--TESTING FOR SYMMETRY

----------------------------------------------------------

1.13.1--SYMMETRY PLOT

----------------------------------------------------------

1.14--TRANSFORMING TO SYMMETRY

----------------------------------------------------------

1.14.1--BOX-COX SYMMETRY PLOT

----------------------------------------------------------

1.15--TESTING FOR NORMALITY

----------------------------------------------------------

1.15.1--NORMAL PROBABILITY PLOT

----------------------------------------------------------

1.15.2--TUKEY PPCC PLOT

----------------------------------------------------------

1.15.3--T PPCC PLOT

----------------------------------------------------------

1.16--TRANSFORMING TO NORMALITY

----------------------------------------------------------

1.16.1--BOX-COX NORMALITY PLOT

----------------------------------------------------------

1.17--TESTING FOR NORMAL OUTLIERS

----------------------------------------------------------

1.18--COMPUTING CONFIDENCE LIMITS FOR DISTRIBUTIONA

----------------------------------------------------------

1.19--HYPOTHESIS TESTING ON DISTRIBUTIONAL PARAMETERS











































----------  *TIME SERIES ANALYSIS (1 VARIABLE)*  ---------

2--TIME SERIES ANALYSIS (1 VARIABLE)

The sub-categories are--

   1. General Discussion
   2. Checking for Time-Domain Structure
   3. Checking for Frequency-Domain Structure
   4. Checking for Time and Frequency Domain Struct
   5. Testing White Noise (Randomness)
   6. Checking for Trends
   7. Fitting Box-Jenkins Models
   8. Smoothing
   9. Filtering

To select a menu item, enter 1 to 9.

----------------------------------------------------------

2.1--GENERAL DISCUSSION

----------------------------------------------------------

2.2--CHECKING FOR TIME-DOMAIN STRUCTURE

----------------------------------------------------------

2.2.1--RUN SEQUENCE PLOT

----------------------------------------------------------

2.2.2--LAG PLOT

----------------------------------------------------------

2.2.3--AUTOCORRELATION PLOT

----------------------------------------------------------

2.2.4--PARTIAL-AUTOCORRELATION PLOT

----------------------------------------------------------

2.2.5--COMPLEX DEMODULATION PLOTS

----------------------------------------------------------

2.3--CHECKING FOR FREQUENCY-DOMAIN STRUCTURE

----------------------------------------------------------

2.3.1--SPECTRAL PLOT

----------------------------------------------------------

2.3.2--PERIODOGRAM

----------------------------------------------------------

2.4--CHECKING FOR TIME AND FREQUENCY DOMAIN STRUCTURE

----------------------------------------------------------

2.4.1--4-PLOT ANALYSIS

----------------------------------------------------------

2.5--TESTING WHITE NOISE (RANDOMNESS)

----------------------------------------------------------

2.5.1--LAG PLOT

----------------------------------------------------------

2.5.2--RUNS ANALYSIS

----------------------------------------------------------

2.5.3--DISTRIBUTION-FREE TESTS

----------------------------------------------------------

2.5.4--AUTOCORRELATION PLOT

----------------------------------------------------------

2.5.5--SPECTRAL PLOT

----------------------------------------------------------

2.5.6--4-PLOT ANALYSIS

----------------------------------------------------------

2.6--CHECKING FOR TRENDS

----------------------------------------------------------

2.6.1--RUN SEQUENCE PLOT

----------------------------------------------------------

2.6.2--CORRELATION WITH TIME

----------------------------------------------------------

2.6.3--LINEAR FIT OVER TIME

----------------------------------------------------------

2.7--FITTING BOX-JENKINS MODELS

----------------------------------------------------------

2.7.1--LAG PLOT

----------------------------------------------------------

2.7.2--AUTOCORRELATION PLOT

----------------------------------------------------------

2.7.3--PARTIAL AUTOCORRELATION PLOT

----------------------------------------------------------

2.8--SMOOTHING

----------------------------------------------------------

2.8.1--MOVING AVERAGE SMOOTHING

----------------------------------------------------------

2.8.2--LEAST SQUARES SMOOTHING

----------------------------------------------------------

2.8.3--MEDIAN SMOOTHING

----------------------------------------------------------

2.8.4--ROBUST SMOOTHING

----------------------------------------------------------

2.8.5--EXPONENTIAL SMOOTHING

----------------------------------------------------------

2.8.6--ASSESSING THE GOODNESS OF THE SMOOTHING

----------------------------------------------------------

2.8.6.1--RESIDUAL STANDARD DEVIATION

----------------------------------------------------------

2.8.6.2--SUPERIMPOSING RAW DATA AND FITTED VALUES

----------------------------------------------------------

2.8.6.3--SCATTER PLOTS OF RESIDUALS

----------------------------------------------------------

2.8.6.4--NORMAL PROBABILITY PLOT OF RESIDUALS

----------------------------------------------------------

2.8.6.5--5-PLOT OF RESIDUALS

----------------------------------------------------------

2.9--FILTERING

----------------------------------------------------------

2.9.1--LOW-PASS FILTERS

----------------------------------------------------------

2.9.2--HIGH-PASS FILTERS

----------------------------------------------------------

2.9.3--ASSESSING THE GOODNESS OF THE FILTERING











----------  *TIME SERIES ANALYSIS (2 VARIABLES)*  --------

3--TIME SERIES ANALYSIS--2 VARIABLES

The sub-categories are--

   1. General discussion
   2. Checking for time-domain structure
   3. Checking for frequency-domain structure
   4. Checking for time- & freq.-domain structure

To select a menu item, enter 1 to 4.

----------------------------------------------------------

3.1--GENERAL DISCUSSION

----------------------------------------------------------

3.2--CHECKING FOR TIME-DOMAIN STRUCTURE

----------------------------------------------------------

3.2.1--SCATTER PLOT

----------------------------------------------------------

3.2.2--MULTI-TRACE PLOTS

----------------------------------------------------------

3.2.3--CROSS-SPECTRAL PLOT

----------------------------------------------------------

3.2.4--BIHISTOGRAM

----------------------------------------------------------

3.3--CHECKING FOR FREQUENCY-DOMAIN STRUCTURE

----------------------------------------------------------

3.3.1--CROSS-SPECTRUM

----------------------------------------------------------

3.3.2--COHERENCY SPECTRUM

----------------------------------------------------------

3.3.3--QUADRATURE SPECTRUM

----------------------------------------------------------

3.3.4--CO-SPECTRUM

----------------------------------------------------------

3.3.5--GAIN SPECTRUM

----------------------------------------------------------

3.3.6--ARGAND SPECTRUM

----------------------------------------------------------

3.4--CHECKING FOR TIME AND FREQUENCY DOMAIN STRUCTURE

----------------------------------------------------------

3.4.1--4-PLOT ANALYSIS




























----------  *CORRELATION ANALYSIS*  ----------------------

4--CORRELATION ANALYSIS

The sub-categories are--

   1. General discussion
   2. Multi-scatter plots
   3. Multi-ANOP plots
   4. Multi-box plots
   5. Cross-correlation tabulation
   6. Transforming variables
   7. Distribution-free tests

To select a menu item, enter 1 to 7.

----------------------------------------------------------

4.1--GENERAL DISCUSSION

----------------------------------------------------------

4.2--MULTI-SCATTER PLOTS

----------------------------------------------------------

4.3--MULTI-ANOP PLOTS

----------------------------------------------------------

4.4--MULTI-BOX PLOTS

----------------------------------------------------------

4.5--CROSS-CORRELATION TABULATION

----------------------------------------------------------

4.6--TRANSFORMING VARIABLES

----------------------------------------------------------

4.7--DISTRIBUTION-FREE TESTS

























































----------  *FITTING (1 INDEPENDENT VARIABLE)*  ----------

5--FITTING (1 INDEPENDENT VARIABLE)

The sub-categories are--

   1. General Discussion
   2. Selecting a Model
   3. Fitting a Model
   4. Assessing the Goodness of Fit of a Model
   5. Improving the Model
   6. Fitting with Weights
   7. Fitting with Constraints
   8. Fitting with Other Criteria (e.g., L1 Fitting)

To select a menu item, enter 1 through 8.

----------------------------------------------------------

5.1 GENERAL DISCUSSION

Fitting (= regression = model-buiding) is the
process of determining a mathematical function
which relates the primary variable of interest
(also called the response variable or the
dependent variable) to a second variable (also
called the independent variable).  In a plot of
the 2 variables, the response variable usually
appears on the vertical axis, and the independent
variable on the horizontal axis.  Most fitting is
done using the least squares criteria (minimizing
the sum of squared deviations).

The fitting process consists of 4 distinct
steps--selecting the model, fitting the model,
assessing the goodness of fit of a model, and
improving the model.

A typical DATAPLOT command sequence for the above
is

   PLOT Y X
   FIT Y = <some model>
   CHARACTERS X BLANK
   LINES BLANK SOLID
   PLOT Y PRED VERSUS X
   4-PLOT RES X

Other aspects of model fitting may include
fitting with weights, fitting with constraints,
and fitting with non-least squares criteria.

Enter -1 to revert to the previous menu.

----------------------------------------------------------

5.2--SELECTING A MODEL (1 INDEPENDENT VARIABLE)

Model selection is the process of deciding on
the functional form of the equation to relate Y with X.
No amount of least squares fitting will make
up for a poorly-chosen model, so careful
selection of the model is of prime importance.

Some models suggest themselves from physical
theory; others must be determined from the data.
If your model is known, then revert to the previous
menu for instructions on fitting the model.
If your model is unknown then the first step
is always to plot the data, and the second step
is to compare the plotted data with reference
curves to arrive at allowable mathematical
functions.

   1) plot Y versus X
   2) generate Y versus X reference curves

To select a menu item, enter 1 or 2

----------------------------------------------------------

5.2.1--PLOT Y VERSUS X

To generate a plot of variable Y versus variable X
(a scatter plot), use the PLOT command, as in

   PLOT Y X

The variable Y will be plotted vertically;
the variable X, horizontally.  The current
DATAPLOT character, line, color, etc. settings
will be in effect.

Note the general shape of the resulting plot.
The primary question to be addressed is
what mathematical function has any chance
at all of fitting the shape of the data?
A given mathematical function usually has only a
limited range of shapes that it can fit.
Some mathematical functions (e.g., rational
functions) are more flexible than others.
The principle of parsimony dictates that we choose
the simplest function that is adequate.

To generate a set of reference shapes with which
to compare our data plot, revert to the
previous menu (enter -1) and select the
menu item to do so.

----------------------------------------------------------

5.2.2--GENERATE Y VERSUS X REFERENCE CURVES

 `
 `5hg$E
 7af!S
 `
 5hg$E1
 7af!S
 
 `
 `1n|#F1n|#J1b}#N1c}#R1k}#V1l}#[1d~#_1l~$C1i$G2e`$K2aa$O2bb$S2bc$W2cd$[
 `2cd$[2ge$_2kf%C2lg%H2di%L2mj%P2il%T2an%X2no%\2nq&@2ns&D2ou&H2ow&L2`z&Q
 `2`z&Q2d|&U2l~&Y3ea&]3mc'A3jf'E3fi'I3bl'M3co'Q3cr'U3cu'Y3dx'^3h{(B3m~(F
 `3m~(F4eb(J4me(N4fi(R4bm(V4np(Z4kt(^4kx)B4h|)G5l`)K5`e)O5ei)S
 `
 `1n|#F1m|)S
 `
 `1m|)S5a)S
 `
 `5a)S5b#F
 `
 `5b#F1n|#F
 `
 `
 `
 `
 7af!S
 `
 `5hg*R
 7af!S
 `
 5hg*R2
 7af!S
 
 `
 `1m|)S1a})W1j})[1j~)_1n*C2ca*G2kb*K2gd*O2`f*T2lg*X2mi*\2mk+@2mm+D2no+H
 `2no+H2br+L2gt+P2gv+T2kx+X2lz+]2d},A2h,E3ma,I3ad,M3jf,Q3nh,U3fk,Y3km,]
 `3km,]3cp-A3hr-F3lt-J3dw-N3my-R3e|-V3m~-Z4fa-^4jc.B4cf.F4kh.J4ck.N4hm.S
 `4hm.S4`p.W4ir.[4au._4iw/C4bz/G4j|/K4b/O5ka/S5gd/W5lf/\5di0@
 `
 `1m|)S1l|0@
 `
 `1l|0@5`0@
 `
 `5`0@5a)S
 `
 `5a)S1m|)S
 `
 `
 `
 `
 7af!S
 `
 `5kg0^
 7af!S
 `
 5kg0^3
 7af!S
 
 `
 `1l|0D1a}0H1m}0L1j~0P1n0T2ba0X2ob0\2kd1@2gf1D2dh1I2dj1M2il1Q2in1U2mp1Y
 `2mp1Y2bs1]2fu2A2kw2E2oy2I2c|2M2d~2R3d`2V3hb2Z3md2^3ag3B3bi3F3fk3J3jm3N
 `3jm3N3ko3R3kq3V3ls3[3lu3_3lw4C3my4G3m{4K3m}4O3n4S4ja4W4kc4[4ge4_4gg5C
 `4gg5C4`i5H4`k5L4ml5P4in5T4ep5X4br5\4ns6@4ju6D4gw6H4cy6L
 `
 `1l|0@1o|6L
 `
 `1o|6L5c6L
 `
 `5c6L5`0@
 `
 `5`0@1l|0@
 `
 `
 `
 `
 7af!S
 `
 `5jg7K
 7af!S
 `
 5jg7K4
 7af!S
 
 `
 `1k}6L1o}6N1l}6Q1`~6S1`~6U1d~6W1d~6Y1i~6[1m~6]1m~6_1a7A1e7C1j7E1n7G
 `1n7G2f`7I2j`7K2ba7M2fa7O2oa7Q2gb7S2cc7U2kc7W2gd7Y2`e7\2le7^2hf8@2hg8B
 `2hg8B2hh8D2mi8F2ak8H2el8J2mm8L2eo8N2np8P2jr8R2jt8T2jv8V2jx8X2nz8Z2g}8\
 `2g}8\2o8^3gb9@3ce9B3ch9D3lj9G3`n9I3`q9K3dt9M3dw9O3iz9Q3m}9S4aa9U4ed9W
 `4ed9W4ig9Y4nj9[4nm9]4np9_4ns:A4jv:C4gy:E4c|:G4k~:I5o`:K5cc:M5ge:O5dg:R
 `5dg:R5`i:T5lj:V5hl:X5`n:Z5io:\5mp:^5ar;@5as;B5at;D5bu;F5bv;H5nv;J5jw;L
 `5jw;L5bx;N5ox;P5gy;R5oy;T5gz;V5oz;X5`{;[5h{;]5l{;_5`|<A5d|<C5h|<E5m|<G
 `5m|<G5a}<I5e}<K5e}<M5i}<O5j}<Q5n}<S5n}<U5b~<W5b~<Y
 `
 `1o|6L1n|<Y
 `
 `1n|<Y5b<Y
 `
 `5b<Y5c6L
 `
 `5c6L1o|6L
 `
 `
 `
 `
 7af!S
 `
 `1de$E
 7af!S
 `
 1de$E5
 7af!S
 
 `
 `-jz#F.ha#[.ai$O.gq%C.iz%X/od&L/ap'A/k|'U0ej(J0ky(^1ij)S
 `
 `-jz#F-iz)S
 `
 `-iz)S1m|)S
 `
 `1m|)S1n|#F
 `
 `1n|#F-jz#F
 `
 `
 `
 `
 7af!S
 `
 `1de*R
 7af!S
 `
 1de*R6
 7af!S
 
 `
 `-iz)S.it*\/ln,E0lh-N1lb.W1l|0@
 `
 `-iz)S-hz0@
 `
 `-hz0@1l|0@
 `
 `1l|0@1m|)S
 `
 `1m|)S-iz)S
 `
 `
 `
 `
 7af!S
 `
 `1ge0^
 7af!S
 `
 1ge0^7
 7af!S
 
 `
 `-hz0@.hg0B.`m0D.aq0F.it0H.mw0J.iz0L.a}0N.j0P/ja0R/nc0T/ne0V/ng0X/ji0Z
 `/ji0Z/gk0\/cm0^/on1@/gp1B/oq1D/ds1G/lt1I/dv1K/hw1M/`y1O/ez1Q/m{1S/a}1U
 `/a}1U/e~1W/i1Y0n`1[0na1]0bc1_0fd2A0fe2C0kf2E0kg2G0oh2I0oi2K0oj2M0ok2O
 `0ok2O0`m2R0`n2T0`o2V0`p2X0`q2Z0ar2\0mr2^0ms3@0mt3B0mu3D0jv3F0jw3H0jx3J
 `0jx3J0fy3L0fz3N0c{3P0c|3R0o|3T0o}3V0k~3X0h3[1d`3]1`a3_1`b4A1lb4C1hc4E
 `1hc4E1ed4G1ae4I1af4K1mf4M1ig4O1fh4Q1bi4S1ni4U1jj4W1fk4Y1cl4[1ol4]1km4_
 `1km4_1gn5A1co5C1lo5F1hp5H1dq5J1lq5L1hr5N1es5P1at5R1mt5T1eu5V1av5X1mv5Z
 `1mv5Z1jw5\1bx5^1nx6@1jy6B1bz6D1oz6F1k{6H1c|6J1o|6L
 `
 `-hz0@-kz6L
 `
 `-kz6L1o|6L
 `
 `1o|6L1l|0@
 `
 `1l|0@-hz0@
 `
 `
 `
 `
 7af!S
 `
 `1fe7K
 7af!S
 `
 1fe7K8
 7af!S
 
 `
 `-kz6L.`g6Q.dr6U.d|6Y/ie6]/mm7A/fu7E/b|7I0fb7M0og7Q0ol7U0kq7Y0hu7^0dy8B
 `0dy8B0m|8F0m8J1ib8N1be8R1fg8V1fi8Z1ck8^1ol9B1dn9G1lo9K1`q9O1ar9S1as9W
 `1as9W1bt9[1nt9_1ju:C1gv:G1ov:K1kw:O1`x:T1hx:X1mx:\1ey;@1iy;D1ny;H1bz;L
 `1bz;L1gz;P1kz;T1oz;X1`{;]1d{<A1d{<E1i{<I1i{<M1n{<Q1n{<U1n{<Y
 `
 `-kz6L-jz<Y
 `
 `-jz<Y1n|<Y
 `
 `1n|<Y1o|6L
 `
 `1o|6L-kz6L
 `
 `
 `
 `
 7af!S
 `
 `-dc$E
 7af!S
 `
 -dc$E9
 7af!S
 
 `
 `)jx#F)fz#G)f|#H)b~#I*b`#J*na#K*jc#L*fe#M*nf#N*jh#O*cj#P*kk#Q*gm#R*on#S
 `*on#S*cp#T*kq#U*cs#V*gt#W*ku#X*cw#Y*dx#[*hy#\*hz#]*l{#^*`}#_*`~$@*`$A
 `*`$A+d`$B+da$C+db$D+dc$E+ad$F+ae$G+af$H+mf$I+ig$J+ih$K+ei$L+aj$M+mj$N
 `+mj$N+ik$O+bl$P+nl$Q+jm$R+bn$S+nn$T+fo$U+no$V+fp$W+np$X+fq$Y+nq$Z+gr$[
 `+gr$[+or$\+gs$]+ks$^+ct$_+gt%@+ot%A+cu%B+gu%C+ku%D+`v%F+dv%G+hv%H+lv%I
 `+lv%I+`w%J+dw%K+dw%L+hw%M+lw%N+`x%O+ax%P+ex%Q+ex%R+ix%S+ix%T+mx%U+mx%V
 `+mx%V+ay%W+ay%X+ay%Y+ay%Z+fy%[+fy%\+fy%]+fy%^+fy%_+fy&@+jy&A+jy&B+jy&C
 `+jy&C+jy&D+ky&E+ky&F+ky&G+ky&H+ky&I+ky&J+ky&K+ky&L+ky&M+ky&N+ky&O+hy&Q
 `+hy&Q+hy&R+hy&S+hy&T+hy&U+hy&V+hy&W+hy&X+hy&Y+ly&Z+my&[+my&\+my&]+my&^
 `+my&^+my&_+az'@+az'A+az'B+ez'C+ez'D+jz'E+jz'F+nz'G+nz'H+b{'I+b{'J+f{'K
 `+f{'K+j{'L+j{'M+n{'N+b|'O+g|'P+k|'Q+o|'R+c}'S+g}'T+o}'U+c~'V+g~'W+o~'X
 `+o~'X+c'Y+h'[+l'\,d`'],l`'^,da'_,ha(@,`b(A,lb(B,dc(C,lc(D,dd(E,ae(F
 `,ae(F,ie(G,ef(H,mf(I,ig(J,eh(K,ai(L,mi(M,ij(N,ek(O,fl(P,bm(Q,bn(R,nn(S
 `,nn(S,no(T,np(U,nq(V,nr(W,ns(X,bu(Y,bv(Z,gw([,gx(\,ky(],oz(^,c|(_,g})@
 `,g})@,o~)A-c`)B-ka)C-cc)D-dd)F-le)G-dg)H-`i)I-hj)J-dl)K-lm)L-ho)M-dq)N
 `-dq)N-`s)O-au)P-mv)Q-mx)R-iz)S
 `
 `)jx#F)ix)S
 `
 `)ix)S-iz)S
 `
 `-iz)S-jz#F
 `
 `-jz#F)jx#F
 `
 `
 `
 `
 7af!S
 `
 `-dc*R
 7af!S
 `
 -dc*R10
 7af!S
 
 `
 `*dg)Z*ks*@*c~*G+bg*N+mn*U+iu*\+h{+C,o`+I,ke+P,ni+W,mm+^,dq,E,dt,L,gw,R
 `,gw,R,bz,Y,j|-@,m~-G-`a-N-`c-U-od-[-jf.B-fh.I-mi.P-dk.W-kl.]-cn/D-fo/K
 `-fo/K-ep/R-iq/Y-hr0@
 `
 `)ix)S)hx0@
 `
 `)hx0@-hz0@
 `
 `-hz0@-iz)S
 `
 `-iz)S)ix)S
 `
 `
 `
 `
 7af!S
 `
 `-gc0^
 7af!S
 `
 -gc0^11
 7af!S
 
 `
 `*hj0D*i|0H+eg0L+nn0P+jt0T+fy0X+g}0\,o`1@,oc1D,hf1I,`i1M,ek1Q,em1U,eo1Y
 `,eo1Y,bq1],nr2A,gt2E,ou2I,gw2M,hx2R,ly2V,`{2Z,e|2^,e}3B,f~3F,j3J-f`3N
 `-f`3N-ga3R-gb3V-`c3[-`d3_-ld4C-ie4G-ef4K-ag4O-ng4S-jh4W-gi4[-cj4_-kj5C
 `-kj5C-dk5H-lk5L-il5P-am5T-mm5X-fn5\-nn6@-fo6D-oo6H-kp6L
 `
 `)hx0@)kx6L
 `
 `)kx6L-kz6L
 `
 `-kz6L-hz0@
 `
 `-hz0@)hx0@
 `
 `
 `
 `
 7af!S
 `
 `-fc7K
 7af!S
 `
 -fc7K12
 7af!S
 
 `
 `)hx6Q+hy6U,do6Y,az6]-i`7A-nd7E-bh7I-fj7M-cl7Q-km7U-on7Y-lo7^-hp8B-eq8F
 `-eq8F-ar8J-ir8N-bs8R-fs8V-ns8Z-ct8^-gt9B-lt9G-`u9K-du9O-eu9S-iu9W-nu9[
 `-nu9[-bv9_-bv:C-gv:G-gv:K-kv:O-hv:T-lv:X-mv:\-aw;@-aw;D-fw;H-fw;L-gw;P
 `-gw;P-kw;T-kw;X-hw;]-hw<A-lw<E-mw<I-mw<M-nw<Q-bx<U-bx<Y
 `
 `)kx6L)jx<Y
 `
 `)jx<Y-jz<Y
 `
 `-jz<Y-kz6L
 `
 `-kz6L)kx6L
 `
 `
 `
 `
 7af!S
 `
 `)`a$E
 7af!S
 `
 )`a$E13
 7af!S
 
 `
 `%fv#F%fx#H%f}#J&nd#L&jm#N&gw#P'ca#R'ck#T'kt#V'o}#X(`f#[(hm#](dt#_(dz$A
 `(dz$A(`$C)dc$E)ef$G)mh$I)ij$K)ek$M)mk$O)jk$Q)jj$S)fi$U)jg$W)je$Y)gc$[
 `)gc$[)k`$](o}$_(kz%A(kw%C(dt%F(lp%H(dm%J(`j%L(hf%N(ac%P'm%R'e|%T'ay%V
 `'ay%V'mu%X'mr%Z'jo%\'nl%^'ni&@'bg&B'jd&D'oa&F&g&H&c}&J&oz&L&ox&N&hv&Q
 `&hv&Q&lt&S&lr&U&`q&W&ho&Y&mm&[&el&]&ak&_&ii'A&eh'C&fg'E&bf'G&be'I&bd'K
 `&bd'K&bc'M&fb'O&ka'Q&o`'S&c`'U%g'W%o~'Y%`~'\%h}'^%`}(@%h|(B%`|(D%m{(F
 `%m{(F%e{(H%a{(J%iz(L%ez(N%bz(P%ny(R%jy(T%fy(V%by(X%nx(Z%kx(\%kx(^%gx)@
 `%gx)@%cx)B%cx)D%lw)G%lw)I%hw)K%hw)M%hw)O%ew)Q%ew)S
 `
 `%fv#F%ev)S
 `
 `%ev)S)ix)S
 `
 `)ix)S)jx#F
 `
 `)jx#F%fv#F
 `
 `
 `
 `
 7af!S
 `
 `)`a*R
 7af!S
 `
 )`a*R14
 7af!S
 
 `
 `%ev)S&ms)U'ik)W'i~)Y(jm)[(by)])ja)_)fg*A)fk*C)gm*E)cn*G)km*I)ok*K)ki*M
 `)ki*M)kf*O)dc*R(d*T(d{*V(`w*X(lr*Z(en*\(aj*^(ie+@(ea+B'a}+D'by+F'bu+H
 `'bu+H'fq+J'nm+L'fj+N'of+P'oc+R'o`+T&o}+V&g{+X&lx+[&dv+]&`t+_&`r,A&`p,C
 `&`p,C&`n,E&el,G&mj,I&ei,K&mg,M&if,O&fe,Q&fd,S&bc,U&bb,W&fa,Y&g`,[%k,]
 `%k,]%o~,_%g~-A%k}-C%`}-F%h|-H%`|-J%h{-L%`{-N%mz-P%ez-R%az-T%my-V%iy-X
 `%iy-X%ey-Z%by-\%nx-^%jx.@%fx.B%fx.D%cx.F%ow.H%ow.J%kw.L%kw.N%dw.Q%dw.S
 `%dw.S%dw.U%`w.W%`w.Y%aw.[%aw.]%mv._%mv/A%mv/C%nv/E%nv/G%nv/I%jv/K%jv/M
 `%jv/M%jv/O%kv/Q%kv/S%kv/U%kv/W%kv/Y%hv/\%hv/^%hv0@
 `
 `%ev)S%dv0@
 `
 `%dv0@)hx0@
 `
 `)hx0@)ix)S
 `
 `)ix)S%ev)S
 `
 `
 `
 `
 7af!S
 `
 `)ca0^
 7af!S
 `
 )ca0^15
 7af!S
 
 `
 `%dv0@%lx0A%h{0B%`~0C&h`0D&`c0E&ie0F&mg0G&ej0H&il0I&ao0J&eq0K&is0L&av0M
 `&av0M&ex0N&ez0O&j|0P&n~0Q'ba0R'bc0S'be0T'fg0U'fi0V'fk0W'fm0X'fo0Y'bq0Z
 `'bq0Z'cs0['cu0\'ov0]'kx0^'kz0_'g|1@'c~1A'o1B(ka1C(cc1D(ld1F(df1G(`h1H
 `(`h1H(hi1I(`k1J(hl1K(`n1L(ho1M(`q1N(hr1O(ms1P(eu1Q(iv1R(mw1S(ey1T(iz1U
 `(iz1U(m{1V(m|1W(a~1X(e1Y)e`1Z)ja1[)jb1\)jc1])jd1^)je1_)jf2@)jg2A)jh2B
 `)jh2B)fi2C)fj2D)ck2E)cl2F)ol2G)km2H)gn2I)co2J)ko2K)gp2L)cq2M)kq2N)cr2O
 `)cr2O)lr2Q)ds2R)ls2S)dt2T)lt2U)`u2V)hu2W)lu2X)dv2Y)hv2Z)mv2[)aw2\)ew2]
 `)ew2])iw2^)mw2_)ax3@)ax3A)ex3B)ex3C)ex3D)jx3E)jx3F)jx3G)fx3H)fx3I)fx3J
 `)fx3J)bx3K)bx3L)nw3M)jw3N)fw3O)cw3P)ov3Q)kv3R)gv3S)ou3T)ku3U)cu3V)ot3W
 `)ot3W)gt3X)os3Y)ds3[)lr3\)`r3])hq3^)`q3_)dp4@)ho4A)`o4B)dn4C)hm4D)ll4E
 `)ll4E)al4F)ak4G)ej4H)ei4I)ih4J)ig4K)if4L)ie4M)id4N)ic4O)jb4P)ja4Q)f`4R
 `)f`4R(f4S(b~4T(n|4U(n{4V(jz4W(fy4X(nw4Y(jv4Z(gu4[(os4\(kr4](cq4^(ko4_
 `(ko4_(cn5@(kl5A(ck5B(ki5C(ch5D(df5F(ld5G(`c5H(ha5I'l5J'`~5K'd|5L'hz5M
 `'hz5M'hx5N'lv5O'au5P'as5Q'aq5R'eo5S'em5T'ek5U'ei5V'eg5W'ae5X'ac5Y'aa5Z
 `'aa5Z&n~5[&j|5\&fz5]&fx5^&bv5_&js6@&fq6A&bo6B&jl6C&fj6D&og6E&ke6F&cc6G
 `&cc6G&k`6H%c~6I%k{6J%ox6K%gv6L
 `
 `%dv0@%gv6L
 `
 `%gv6L)kx6L
 `
 `)kx6L)hx0@
 `
 `)hx0@%dv0@
 `
 `
 `
 `
 7af!S
 `
 `)ba7K
 7af!S
 `
 )ba7K16
 7af!S
 
 `
 `%gv6L'dz6Q(hs6U)dc6Y)il6])aq7A)nq7E)bp7I)jl7M)ch7Q)ob7U(g}7Y(lw7^(dr8B
 `(dr8B(am8F(ah8J(ec8N'b8R'b{8V'bw8Z'ks8^'gp9B'dm9G'hj9K'`h9O'ie9S'ec9W
 `'ec9W'fa9[&f9_&j}:C&o{:G&gz:K&ox:O&hw:T&`v:X&mt:\&ms;@&mr;D&jq;H&np;L
 `&np;L&oo;P&co;T&cn;X&dm;]&hl<A&`l<E&ek<I&ij<M&bj<Q&ji<U&bi<Y
 `
 `%gv6L%fv<Y
 `
 `%fv<Y)jx<Y
 `
 `)jx<Y)kx6L
 `
 `)kx6L%gv6L
 `
 `
 `
 `
 7af!S
 `
 `$l~$E
 7af!S
 `
 $l~$E17
 7af!S
 
 `
 `!by#F!fy#H!jy#J!ny#L!bz#N!gz#P!kz#R!oz#T!c{#V!k{#X!l{#[!`|#]!h|#_!`}$A
 `!`}$A!h}$C!`~$E!i~$G!a$I!m$K"e`$M"aa$O"bb$Q"nb$S"nc$U"nd$W"bf$Y"gg$[
 `"gg$["oh$]"gj$_"gl%A"cn%C"dp%F"lr%H"hu%J"lx%L"`|%N#a`%P#id%R#ii%T#ao%V
 `#ao%V#eu%X#a|%Z$jc%\$jk%^$ns&@$f|&B%fd&D%kk&F%gq&H%cu&J%gv&L%cu&N%dq&Q
 `%dq&Q%hk&S%dd&U$d|&W$ls&Y$ik&[$ic&]#a|&_#eu'A#ao'C#ji'E#jd'G#b`'I"b|'K
 `"b|'K"nx'M"ju'O"or'Q"gp'S"cn'U"gl'W"gj'Y"lh'\"dg'^"`f(@"ld(B"lc(D"mb(F
 `"mb(F"ab(H"aa(J"e`(L!m(N!b(P!j~(R!b~(T!j}(V!b}(X!j|(Z!c|(\!o{(^!k{)@
 `!k{)@!c{)B!oz)D!hz)G!dz)I!`z)K!ly)M!hy)O!ey)Q!ay)S
 `
 `!bt#F!at)S
 `
 `!at)S%ev)S
 `
 `%ev)S%fv#F
 `
 `%fv#F!bt#F
 `
 `
 `
 `
 7af!S
 `
 `$l~*R
 7af!S
 `
 $l~*R18
 7af!S
 
 `
 `%ev)S%ai)W$f~)[$nt)_$bm*C$cf*G$c`*K#oz*O#`v*T#lq*X#an*\#ij+@#eg+D#jd+H
 `#jd+H#na+L"k+P"g}+T"g{+X"dy+]"hw,A"lu,E"et,I"mr,M"jq,Q"fp,U"bo,Y"om,]
 `"om,]"ol-A"lk-F"lj-J"li-N"ai-R"ah-V"eg-Z"jf-^"ne.B"ce.F"kd.J"oc.N"`c.S
 `"`c.S"hb.W"ab.["ia._"aa/C"j`/G"b`/K!j/O!c/S!k~/W!`~/\!l}0@
 `
 `!at)S!`t0@
 `
 `!`t0@%dv0@
 `
 `%dv0@%ev)S
 `
 `%ev)S!at)S
 `
 `
 `
 `
 7af!S
 `
 `$o~0^
 7af!S
 `
 $o~0^19
 7af!S
 
 `
 `%dv0@%ih0J$b|0T$op0^$hf1I#e}1S#bu1]#km2G#hf2R#i`2\"nz3F"ou3P"dq3["`m4E
 `"`m4E"ei4O"ne4Y"ob5C"``5N!i}5X!f{6B!gy6L
 `
 `!`t0@!ct6L
 `
 `!ct6L%gv6L
 `
 `%gv6L%dv0@
 `
 `%dv0@!`t0@
 `
 `
 `
 `
 7af!S
 `
 `$n~7K
 7af!S
 `
 $n~7K20
 7af!S
 
 `
 `%gu6L%gu6N%du6Q%`u6S%`u6U%lt6W%lt6Y%it6[%et6]%at6_%at7A%ms7C%js7E%bs7G
 `%bs7G%nr7I%jr7K%br7M%jq7O%gq7Q%kp7S%cp7U%ko7W%on7Y%`n7\%dm7^%dl8@%hk8B
 `%hk8B%dj8D%ei8F%ah8H%mf8J%ee8L%mc8N%fb8P%f`8R$j~8T$j|8V$fz8X$bx8Z$ou8\
 `$ou8\$gs8^$kp9@$cn9B$ck9D$`h9G$`e9I$`b9K#l~9M#l{9O#ix9Q#eu9S#ar9U#mn9W
 `#mn9W#ik9Y#fh9[#fe9]#fb9_"f:A"j|:C"oy:E"cw:G"kt:I"gr:K"cp:M"om:O"lk:R
 `"lk:R"`j:T"dh:V"hf:X"`e:Z"ic:\"eb:^"aa;@"a`;B!a;D!b~;F!b};H!f|;J!j{;L
 `!j{;L!nz;N!gz;P!oy;R!gy;T!ox;V!gx;X!`x;[!hw;]!dw;_!`w<A!lv<C!hv<E!ev<G
 `!ev<G!av<I!mu<K!mu<M!iu<O!fu<Q!fu<S!fu<U!bu<W!bu<Y
 `
 `!ct6L!bt<Y
 `
 `!bt<Y%fv<Y
 `
 `%fv<Y%gv6L
 `
 `%gv6L!ct6L
 `
 `
` jw#F
7af!S
8
`
:
 jw#FWHICH OF THE ABOVE PLOTS IS CLOSEST IN APPEARANCE TO YOUR DATA?
7af!S
8
`
:
 ni#FENTER 1 TO 20 TO SELECT A PLOT.


----------------------------------------------------------

5.2.2.1--CURVE 1 SELECTED FROM Y VERSUS X REFERENCE PLOTS
         (QUADRATIC)

This curve was generated by

   PLOT X**2 FOR X = 0 .1 5

Characteristics--
   monotonic up
   no inflection point
   for small x, function approaches 0
   for small x, slope    approaches 0
   for large x, function approaches infinity
   for large x, slope    approaches infinity

Suggested models--
   y = a0 + a1*x + a2*x**2
   y = a0 + a1*x + a2*x**2 + a3*x**3
   y = exp(x)

To fit a given model, use the FIT command
as in FIT Y = A0 + A1*X + A2*X**2
For more details on fitting, revert back 2 menus
(enter -2) and select the menu item on
fitting a model.



----------------------------------------------------------

5.2.2.2--CURVE 2 SELECTED FROM Y VERSUS X REFERENCE PLOTS
         (MONOTONIC)

This curve was generated by

   PLOT X**2 / (1+X) FOR X = 0 .1 5

Characteristics--
   monotonic up
   no inflection point
   for small x, function approaches 0
   for small x, slope    approaches 0
   for large x, function approaches infinity
   for large x, slope    approaches constant

Suggested models--
   y = a2*x**2 / (1+b1*x)
   y = quadratic / linear
   y = cubic / quadratic

To fit a given model, use the FIT command
as in FIT Y = A2*X**2 / (1+B1*X)
For more details on fitting, revert back 2 menus
(enter -2) and select the menu item on
fitting a model.



----------------------------------------------------------

5.2.2.3--CURVE 3 SELECTED FROM Y VERSUS X REFERENCE PLOTS
         (MONOTONIC)

This curve was generated by

   PLOT X**3 / (1+X)**2.5 FOR X = .1 .1 5

Characteristics--
   monotonic up
   1 inflection point
   for small x, function approaches 0
   for small x, slope    approaches 0
   for large x, function approaches infinity
   for large x, slope    approaches constant

Suggested models--
   y = a0*x**3 / (1+b1*x)**2.5

To fit a given model, use the FIT command
as in FIT Y = A0*X**3 / (1+B1*X)**2.5
For more details on fitting, revert back 2 menus
(enter -2) and select the menu item on
fitting a model.





----------------------------------------------------------

5.2.2.4--CURVE 4 SELECTED FROM Y VERSUS X REFERENCE PLOTS
         (S-SHAPED)

This curve was generated by

   PLOT 1 / (1+EXP(-X)) FOR X = -5 .1 5

Characteristics--
   monotonic up
   1 inflection point
   for small x, function approaches 0
   for small x, slope    approaches 0
   for large x, function approaches constant
   for large x, slope    approaches 0

Suggested models--
   y = a0 / (1+b1*exp(-x))
   y = a2*x**2 / (1 + b2*x**2)
   y = quadratic / quadratic
   y = cubic / cubic

To fit a given model, use the FIT command
as in FIT Y = A2 / (1+B2*EXP(-X))
For more details on fitting, revert back 2 menus
(enter -2) and select the menu item on
fitting a model.


----------------------------------------------------------

5.2.2.5--CURVE 5 SELECTED FROM Y VERSUS X REFERENCE PLOTS
         (QUADRATIC)

This curve was generated by

   PLOT EXP(-X) FOR X = 0 .1 1

Characteristics--
   monotonic up
   no inflection point
   for small x, function approaches 0
   for small x, slope    approaches constant
   for large x, function approaches infinity
   for large x, slope    approaches infinity

Suggested models--
   y = a0 + a1*exp(-a2*x)  with a2 > 0
   y = quadratic
   y = cubic
   y = quartic

To fit a given model, use the FIT command
as in FIT Y = A0 + A1*EXP(-A2*X)
For more details on fitting, revert back 2 menus
(enter -2) and select the menu item on
fitting a model.


----------------------------------------------------------

5.2.2.6--CURVE 6 SELECTED FROM Y VERSUS X REFERENCE CURVES
         (LINEAR)

This curve was generated by

   PLOT X FOR X = 0 1 5

Characteristics--
   monotonic up
   no inflection point
   for small x, function approaches -infinity
   for all   x, slope    is         constant
   for large x, function approaches infinity

Suggested models--
   y = a0 + a1*x
   y = a0 + a1*x**a2
   log(y) = a0 + a1*log(x)

To fit a given model, use the FIT command
as in FIT Y = A0 + A1*X
For more details on fitting, revert back 3 menus
(enter -3) and select the menu item on
fitting a model.




----------------------------------------------------------

5.2.2.7--CURVE 7 SELECTED FROM Y VERSUS X REFERENCE PLOTS
         (SQUARE ROOT)

This curve was generated by

   PLOT SQRT(X) FOR X = 0 .01 1

Characteristics--
   monotonic up
   no inflection point
   convex down
   for small x, function approaches 0
   for small x, slope    approaches constant
   for large x, function approaches infinity
   for large x, slope    approaches infinity

Suggested models--
   y = a0 + a1*sqrt(x)
   y = a0 + a1*(x**a2)
   y = a0 + a1*log(x+1)
   y = a0 + a1*log(x+a2)

To fit a given model, use the FIT command
as in FIT Y = A0 + A1*SQRT(X)
For more details on fitting, revert back 2 menus
(enter -2) and select the menu item on
fitting a model.

----------------------------------------------------------

5.2.2.8--CURVE 8 SELECTED FROM Y VERSUS X REFERENCE PLOTS
         (ASYMPTOTE)

This curve was generated by

   PLOT 1 - EXP(-X) FOR X = 0 .1 5

Characteristics--
   monotonic up
   no inflection point
   for small x, function approaches 0
   for small x, slope    approaches constant
   for large x, function approaches constant
   for large x, slope    approaches 0

Suggested models--
   y = a0 + a1*exp(-a2*x)    with a1 < 0 and a2 > 0
   y = a0*x / (1+b1*x)
   y = quadratic /quadratic

To fit a given model, use the FIT command
as in FIT Y = A0 + A1*EXP(-A2*X)
For more details on fitting, revert back 2 menus
(enter -2) and select the menu item on
fitting a model.



----------------------------------------------------------

5.2.2.9--CURVE 9 SELECTED FROM Y VERSUS X REFERENCE PLOTS
         (CUBIC)

This curve was generated by

   PLOT X**3 FOR X = -10 .1 10

Characteristics--
   monotonic up
   1 inflection point
   for small x, function approaches -infinity
   for small x, slope    approaches -infinity
   for large x, function approaches infinity
   for large x, slope    approaches infinity

Suggested models--
   y = a0 + a1*x**3
   y = a0 + a1*x + a2*x**2 + a3*x**3

To fit a given model, use the FIT command
as in FIT Y = A0 + A1*X**3
For more details on fitting, revert back 2 menus
(enter -2) and select the menu item on
fitting a model.




----------------------------------------------------------

5.2.2.10--CURVE 10 SELECTED FROM Y VERSUS X REFERENCE PLOTS

This curve was generated by

   PLOT 1 / (1 + 1/X) FOR X = 0 .1 3   APPROX

Characteristics--
   monotonic up
   no inflection point
   for small x, function approaches 0
   for small x, slope    approaches -infinity
   for large x, function approaches infinity
   for large x, slope    approaches constant

Suggested models--
   y = xxx

To fit a given model, use the FIT command
as in FIT Y = XXX
For more details on fitting, revert back 2 menus
(enter -2) and select the menu item on
fitting a model.






----------------------------------------------------------

5.2.2.11--CURVE 11 SELECTED FROM Y VERSUS X REFERENCE PLOTS
          (LOGARITHM)

This curve was generated by

   PLOT LOG(X) FOR X = .1 .1 5

Characteristics--
   monotonic up
   no inflection point
   convex down
   for small x, function approaches -infinity
   for small x, slope    approaches -infinity
   for large x, function approaches infinity
   for large x, slope    approaches infinity

Suggested models--
   y = a0 + a1*log(x)
   y = a0 + a1*log(x+a2)
   y = a0 + a1*sqrt(x)
   y = a0 + a1*x**a2     with a2 < 1

To fit a given model, use the FIT command
as in FIT Y = A0 + A1*LOG(X)
For more details on fitting, revert back 2 menus
(enter -2) and select the menu item on
fitting a model.

----------------------------------------------------------

5.2.2.12--CURVE 12 SELECTED FROM Y VERSUS X REFERENCE PLOTS
          (ASYMPTOTE)

This curve was generated by

   PLOT 1 - 1/X FOR X = .1 .1 5

Characteristics--
   monotonic up
   no inflection point
   for small x, function approaches 0, constant, or -infinity
   for small x, slope    approaches -infinity
   for large x, function approaches constant
   for large x, slope    approaches 0

Suggested models--
   y = a0 + a1/x     with a1 < 0
   y = a0 + a1/x**2     with a1 < 0
   y = a0 + a1/x**a2     with a1 < 0 and a2 >0

To fit a given model, use the FIT command
as in FIT Y = A0 + A1/X
For more details on fitting, revert back 2 menus
(enter -2) and select the menu item on
fitting a model.



----------------------------------------------------------

5.2.2.13--CURVE 13 SELECTED FROM Y VERSUS X REFERENCE PLOTS
          (SKEWED)

This curve was generated by

   PLOT X**2*EXP(-X) FOR X = 0 .1 10

Characteristics--
   non-monotonic
   convex down
   skewed
   for small x, function approaches 0
   for small x, slope    approaches 0
   for large x, function approaches 0
   for large x, slope    approaches 0

Suggested models--
   y = a0 * x**2 * exp(-x)
   y = a0 * x**a1 * exp(-x)

To fit a given model, use the FIT command
as in  FIT Y = A0 * X**2 * EXP(-X)
For more details on fitting, revert back 2 menus
(enter -2) and select the menu item on
fitting a model.



----------------------------------------------------------

5.2.2.14--CURVE 14 SELECTED FROM Y VERSUS X REFERENCE PLOTS
          (SKEWED)

This curve was generated by

   PLOT X*EXP(-X) for X = 0 .1 10

Characteristics--
   non-monotonic
   convex down
   unimodal
   skewed
   for small x, function approaches 0
   for small x, slope    approaches constant
   for large x, function approaches constant
   for large x, slope    approaches 0

Suggested models--
   y = a0*x * exp(-a1*x)
   y = a0*x**b1 * exp(-a1*x)

To fit a given model, use the FIT command
as in FIT Y = A0*X * EXP(-A1*X)
For more details on fitting, revert back 2 menus
(enter -2) and select the menu item on
fitting a model.


----------------------------------------------------------

5.2.2.15--CURVE 15 SELECTED FROM Y VERSUS X REFERENCE PLOTS
          (QUADRATIC)

This curve was generated by

   PLOT -X**2 FOR X = -10 .1 10

Characteristics--
   non-monotonic
   convex down
   unimodal
   symmetric
   for small x, function approaches -infinity
   for small x, slope    approaches -infinity
   for large x, function approaches infinity
   for large x, slope    approaches infinity

Suggested models--
   y = a0 + a1*x + a2*x**2

To fit a given model, use the FIT command
as in FIT Y = A0 + A1*X + A2*X**2
For more details on fitting, revert back 2 menus
(enter -2) and select the menu item on
fitting a model.



----------------------------------------------------------

5.2.2.16--CURVE 16 SELECTED FROM Y VERSUS X REFERENCE PLOTS
          (SKEWED)

This curve was generated by

   PLOT SQRT(X) / (1 + X**2) FOR X = 0 .1 5

Characteristics--
   non-monotonic
   convex down
   unimodal
   skewed
   for small x, function approaches 0, constant, or -infinity
   for small x, slope    approaches -infinity
   for large x, function approaches 0
   for large x, slope    approaches 0

Suggested models--
   y = a0*sqrt(x) / (1 + b2*x**2)

To fit a given model, use the FIT command
as in FIT Y = A0*SQRT(X) / (1 + B2*X**2)
For more details on fitting, revert back 2 menus
(enter -2) and select the menu item on
fitting a model.



----------------------------------------------------------

5.2.2.17--CURVE 17 SELECTED FROM Y VERSUS X REFERENCE PLOTS
          (BELL-SHAPED)

This curve was generated by

   PLOT 1 / (1+X**2) FOR X = -5 .1 5

Characteristics--
   no inflection point
   convex down
   symmetric
   for small x, function approaches 0
   for small x, slope    approaches 0
   for large x, function approaches 0
   for large x, slope    approaches 0

Suggested models--
   y = a0 / (1+a2*x**2)
   y = a0 * exp(-a2*x**2)
   y = a0*exp(-a1*x) / (1 + a2*exp(-a1*x))

To fit a given model, use the FIT command
as in FIT Y = A0 / (1 + A2*X**2)
For more details on fitting, revert back 2 menus
(enter -2) and select the menu item on
fitting a model.


----------------------------------------------------------

5.2.2.18--CURVE 18 SELECTED FROM Y VERSUS X REFERENCE PLOTS
          (HYPERBOLIC)

This curve was generated by

   PLOT 1 / (1+X) FOR X = 0 .1 5

Characteristics--
   monotonic down
   no inflection point
   convex up
   for small x, function approaches constant or infinity
   for small x, slope    approaches constant or infinity
   for large x, function approaches 0
   for large x, slope    approaches 0

Suggested models--
   y = a0 / (1 + a1*x)
   y = a0 * exp(-a2*x)
   y = constant / quadratic
   y = constant / cubic

To fit a given model, use the FIT command
as in FIT Y = A0 / (1 + A1*X)
For more details on fitting, revert back 2 menus
(enter -2) and select the menu item on
fitting a model.

----------------------------------------------------------

5.2.2.19--CURVE 19 SELECTED FROM Y VERSUS X REFERENCE PLOTS
          (NEGATIVE EXPONENTIAL)

This curve was generated by

   PLOT EXP(-X) FOR X = 0 .1 2

Characteristics--
   monotonic decreasing
   convex up
   for small x, function approaches constant
   for small x, slope    approaches constant
   for large x, function approaches 0
   for large x, slope    approaches 0

Suggested models--
   y = a0 + a1*exp(-a2*x)     with a1 > 0 and a2 > 0

To fit a given model, use the FIT command
as in FIT Y = A0 + A1*EXP(-A2*X)
For more details on fitting, revert back 2 menus
(enter -2) and select the menu item on
fitting a model.





----------------------------------------------------------

5.2.2.20--CURVE 20 SELECTED FROM Y VERSUS X REFERENCE PLOTS
          (Z-SHAPED)

This curve was generated by

   PLOT 1 / (1+EXP(X)) FOR X = -5 .1 5

Characteristics--
   monotonic down
   1 inflection point
   for small x, function approaches constant
   for small x, slope    approaches 0
   for large x, function approaches 0
   for large x, slope    approaches 0

Suggested models--
   y = a0 / (1 + b1*exp(-b2*x))

To fit a given model, use the FIT command
as in FIT Y = A0 / (1 + B1*EXP(-B2*X))
For more details on fitting, revert back 2 menus
(enter -2) and select the menu item on
fitting a model.





----------------------------------------------------------

5.3--FITTING A MODEL (1 INDEPENDENT VARIABLE)

   1. Fitting Linear Models
   2. Fitting Polynomial Models
   3. Fitting Non-Linear Models
   4. Fitting Rational Functions
   5. Spline Fitting

To select a menu item, enter 1 through 5.

----------------------------------------------------------

5.3.1--FITTING LINEAR MODELS

To fit a linear model, use the FIT command , as in

   FIT Y = A+B*X

or

   LINEAR FIT Y X

FIT output is generated.  In addition, the
predicted values (= fitted values)
from the fit are automatically stored in the
variable PRED, and the residuals (= deviations)
are automatically stored in the variable RES.
Both PRED and RES may be used by the analyst
in any fashion (e.g., plotting) for further
examination of the model.

Before accepting the linear model as valid,
it is critically important that the
residuals (= raw data - predicted values)
be examined in detail (this is referred to
as "residual analysis".  For guidance on how to
carry out such model-validation procedures,
revert back 2 menus (enter -2) and select
the menu item dealing with assessing the
goodness of fit of the model.

----------------------------------------------------------

5.3.2--FITTING POLYNOMIAL MODELS

----------------------------------------------------------

5.3.3--FITTING NON-LINEAR MODELS


----------------------------------------------------------

5.3.4 FITTING RATIONAL FUNCTIONS

----------------------------------------------------------

5.3.5--SPLINE FITTING

To carry out a spline fit, use the SPLINE FIT
command, as in

   LINEAR SPLINE FIT Y X K
   QUADRATIC SPLINE FIT Y X K
   CUBIC SPLINE FIT Y X K
   etc.

In the above, Y is the response variable,
X is the independent variable, and K is
the variable containing the desired knots.

Spline fit output is generated.  In addition, the
predicted values (= fitted values)
from the fit are automatically stored in the
variable PRED, and the residuals (= deviations)
are automatically stored in the variable RES.
Both PRED and RES may be used by the analyst
in any fashion (e.g., plotting) for further
examination of the model.

Before accepting the spline model as valid,
it is critically important that the
residuals (= raw data - predicted values)
be examined in detail (this is referred to
as "residual analysis".  For guidance on how to
carry out such model-validation procedures,
revert back 2 menus (enter -2) and select
the menu item dealing with assessing the
goodness of fit of the model.


----------------------------------------------------------

5.4--ASSESSING THE GOODNESS OF FIT OF A MODEL

There are several statistical criteria
that come to bear in determining if the
model we have chosen adequately fits the data--

   o Is the amount of variation left over
     after the fit as small as we expected?
   o If we had replication in the data, is
     our (model-dependent) residual variation comparable to
     our (model-free) replication standard deviation?
   o Are the fitted values about the same as the
     raw data values?
   o Are the residuals free of any functional
     dependency on X or any other variable?
   o Is the variation in the residuals about
     the same over the entire range of X?
   o Are our residuals random in appearance?
   o Are our residuals normally distributed?

These above questions are addressed in the following--

   1. Residual Standard Deviation
   2. Lack of Fit F Tests
   3. Superimposing Raw Data and Fitted Curve
   4. 4-Plot of Residuals
   5. Scatter Plot of Residuals
   6. Lag Plot of Residuals
   7. Histogram of Residuals
   8. Normal Probability Plot of Residuals

Enter 1 to 8 to select a menu item.

----------------------------------------------------------

5.4.1--RESIDUAL STANDARD DEVIATION

----------------------------------------------------------

5.4.2--LACK OF FIT F TESTS

----------------------------------------------------------

5.4.3--SUPERIMPOSE RAW DATA AND FITTED CURVE

The superimposed plot of raw data and predicted
values is a convenient graphical technique for
assessing the goodness of a fitted model.
To generate a superimposed plot of the raw data
(in Y) and the predicted values (in PRED)
versus the independent variable (in X),
use the PLOT command, as in

   CHARACTERS X BLANK
   LINES BLANK SOLID
   PLOT Y PRED VERSUS X

(The predicted values from the last fit have
automatically been stored in the DATAPLOT
variable PRED).  The X's on the plot are
the raw data; the solid line is the
predicted values.  For a good fit, the
solid line should overlay the data points
over the entire domain of the data.
Search in particular for regions of poor fit
which indicates that the model is not
adequate over the entire domain.

A more thorough assessment of model adequacy
is obtained via various residual plots.
Revert back to the previous menu (enter -1)
and select the menu items dealing with
plotting residuals.

----------------------------------------------------------

5.4.4--4-PLOT OF RESIDUALS

The 4-plot of the residuals is a convenient
1-page graphical summary of the residuals.
To generate a 4-plot, use the 4-PLOT command,
as in

   4-PLOT RES X

(The residuals from the fit have automatically
been stored in the DATAPLOT variable RES).
The 4-plots to appear are

   1) scatter plot (RES versus X)
   2) lag plot (RES(i) versus RES(i-1))
   3) histogram of RES
   4) normal probability plot of RES

Plots 1 and 2 should be structureless.
Plot 3 should be bell-shaped, and plot 4
should be linear.

For comparison, the residuals from an ideal fit
may be generated by entering

   LET N = NUMBER X
   LET Z = NORMAL RANDOM NUMBERS FOR I = 1 1 N
   4-PLOT Z X

For guidance on how to interpret each of the
4 plots, revert to the previous menu (enter -1)
and select the menu item dealing with the
specific plot.

----------------------------------------------------------

5.4.5--SCATTER PLOT OF RESIDUALS

A scatter plot of the residuals is the single
most important plot for testing model adequacy.
To generate a scatter plot of the residuals
(in RES) versus any other variable (say, X),
use the PLOT command, as in

   PLOT RES X

(The residuals from the fit have automatically
been stored in the DATAPLOT variable RES).
If the model is adequate then the plot
should be structureless and random in
appearance.  Any structure or pattern is a
signal to the analyst that the underlying
assumptions to carry out the fit have
been violated, and/or the current model
is inadequate and may be improved upon.

For comparison, a scatter plot of "ideal" residuals
may be generated by entering

   LET N = NUMBER X
   LET Z = NORMAL RANDOM NUMBERS FOR I = 1 1 N
   PLOT Z X

To generate a set of reference residual scatter
plots, enter 1.  To revert to the previous menu,
enter -1.

----------------------------------------------------------

5.4.5.1--GENERATE REFERENCE RESIDUAL SCATTER PLOTS
 a
 a5hg$E
 7af!S
 `
 5hg$E1
 7af!S
 
 a
 a0lh$F0oh$U0nh%E0mh%U0lh&E0oh&T0nh'D0mh'T0lh(D0oh(S
 a
 a3lp$F3op$U3np%E3mp%U3lp&E3op&T3np'D3mp'T3lp(D3op(S
 `
 `-jz#F-iz)S
 `
 `-iz)S5a)S
 `
 `5a)S5b#F
 `
 `5b#F-jz#F
 `
 `
 `
 `
 7af!S
 a
 a5hg*R
 7af!S
 `
 5hg*R2
 7af!S
 
 a
 a2gt*R2j~+B3ai+R3hs,B3o},Q4fh-A4mr-Q4`}.A5kg.P5br/@
 a
 a1gr*R1bh+B0i}+R0`s,B0oh,Q/f~-A/ms-Q/hi.A.c.P.jt/@
 `
 `-iz)S-hz0@
 `
 `-hz0@5`0@
 `
 `5`0@5a)S
 `
 `5a)S-iz)S
 `
 `
 `
 `
 7af!S
 a
 a5kg0^
 7af!S
 `
 5kg0^3
 7af!S
 
 a
 a5jt0_5aj1O4l1_4gu2N4nj2^4i`3N3`v3^3kk4M3fa4]2mv5M
 a
 a.be0_.io1O.ly1_/gd2N/nn2^/ay3N0hc3^0om4M0fx4]1mb5M
 `
 `-hz0@-kz6L
 `
 `-kz6L5c6L
 `
 `5c6L5`0@
 `
 `5`0@-hz0@
 `
 `
 `
 `
 7af!S
 a
 a5jg7K
 7af!S
 `
 5jg7K4
 7af!S
 
 a
 a/ia7L0``7\0ou8K1nb8[1ag9K1lb9[0ou:J0b`:Z/ia;J
 a
 a4mw7L3hy7\3oc8K2nv8[2ir9K2lv9[3oc:J3jy:Z4mw;J
 `
 `-kz6L-jz<Y
 `
 `-jz<Y5b<Y
 `
 `5b<Y5c6L
 `
 `5c6L-kz6L
 `
 `
 `
 `
 7af!S
 `
 `-dc$E
 7af!S
 `
 -dc$E5
 7af!S
 
 
 &fi$@X
 'ec$PX
 'd}%@X
 (gw%OX
 )jq%_X
 *ik&OX
 +he&_X
 +k'NX
 ,jy'^X
 `
 `%fv#F%ev)S
 `
 `%ev)S-iz)S
 `
 `-iz)S-jz#F
 `
 `-jz#F%fv#F
 `
 `
 `
 `
 7af!S
 `
 `-dc*R
 7af!S
 `
 -dc*R6
 7af!S
 
 
 %mq*MX
 %ly*]X
 &of+LX
 &by+\X
 'ep,LX
 (`m,\X
 )on-KX
 *nu-[X
 ,ab.KX
 `
 `%ev)S%dv0@
 `
 `%dv0@-hz0@
 `
 `-hz0@-iz)S
 `
 `-iz)S%ev)S
 `
 `
 `
 `
 7af!S
 `
 `-gc0^
 7af!S
 `
 -gc0^7
 7af!S
 
 
 'd}0ZX
 )cz1IX
 +je1YX
 +i2IX
 ,dh2YX
 +k3HX
 +je3XX
 )az4HX
 'd}4XX
 `
 `%dv0@%gv6L
 `
 `%gv6L-kz6L
 `
 `-kz6L-hz0@
 `
 `-hz0@%dv0@
 `
 `
 `
 `
 7af!S
 `
 `-fc7K
 7af!S
 `
 -fc7K8
 7af!S
 
 
 &gv7FX
 'f}7VX
 )ed8FX
 *hk8VX
 +kr9EX
 
 *jk9UX
 )ed:EX
 'd}:UX
 &gv;DX
 `
 `%gv6L%fv<Y
 `
 `%fv<Y-jz<Y
 `
 `-jz<Y-kz6L
 `
 `-kz6L%gv6L
 `
 `
 `
 `
 7af!S
 
 
 
 
 
 
 
 
 
 
 
 
`
:
 jw#FWHICH OF THE ABOVE PLOTS IS CLOSEST IN APPEARANCE TO YOUR DATA?
7af!S
8
`
:
 ni#FENTER 1 TO 8 TO SELECT A PLOT.



----------------------------------------------------------

5.4.5.1.1--RESIDUAL PLOT--IDEAL

----------------------------------------------------------

5.4.5.1.2--RESIDUAL PLOT--WEDGE UP

----------------------------------------------------------

5.4.5.1.3--RESIDUAL PLOT--WEDGE DOWN

----------------------------------------------------------

5.4.5.1.4--RESIDUAL PLOT--WEDGE MIDDLE

----------------------------------------------------------

5.4.5.1.5--RESIDUAL PLOT--LINEAR TERM

----------------------------------------------------------

5.4.5.1.6--RESIDUAL PLOT--QUADRATIC TERM

----------------------------------------------------------

5.4.5.1.7--RESIDUAL PLOT--QUADRATIC TERM 2

----------------------------------------------------------

5.4.5.1.8--RESIDUAL PLOT--RESIDUAL PLOT--SPLINE

Residual scatter plots with this sharp break
are usually indicative of a model inadequacy
which may be corrected by use of a linear
spline fit model.

To carry out a linear spline fit,
use the SPLINE FIT command as in

   SPLINE FIT Y X K

where K is a variable containing
the desired "knot" points.

For more details on spline fitting,
go to menu 5.2.5 (by entering EXPERT 5.2.5)

----------------------------------------------------------

5.4.6--LAG PLOT OF RESIDUALS


----------------------------------------------------------

5.4.7--HISTOGRAM OF RESIDUALS


----------------------------------------------------------

5.4.8--NORMAL PROBABILITY PLOT OF RESIDUALS


----------------------------------------------------------

5.5--IMPROVING THE MODEL

----------------------------------------------------------

5.5.1--TRANSFORMING TO SIMPLIFY THE MODEL

----------------------------------------------------------

5.5.2--TRANSFORMING TO ACHIEVE HOMOGENEITY

----------------------------------------------------------

5.5.3--TRANSFORMING TO ACHIEVE NORMALITY

----------------------------------------------------------

5.5.4--ADDING NEW VARIABLES

----------------------------------------------------------

5.5.5--CHANGING THE FORM OF THE MODEL

----------------------------------------------------------

5.6--FITTING WITH WEIGHTS

----------------------------------------------------------

5.7--FITTING WITH CONSTRAINTS

----------------------------------------------------------

5.8--FITTING WITH OTHER CRITERIA (E.G., L1 FITTING)




































































































































































































































































----------  *FITTING (2 OR MORE IND. VARIABLES)*  --------

6--FITTING (2 OR MORE INDEPENDENT VARIABLES)

The sub-categories are--

   1. General Discussion
   2. Selecting Variables to be Included in the Model
   3. Selecting a Model
   4. Fitting a Model
   5. Assessing the Goodness of Fit of a Model
   6. Improving the Model
   7. Fitting with Weights
   8. Fitting with Constraints
   9. Fitting with Other Criteria (e.g., L1 Fitting)

To select a menu item, enter 1 through 8.

----------------------------------------------------------

6.1 GENERAL DISCUSSION

Fitting (= regression = model-building) with 2 or
more independent variables is the process of
determining a mathematical function which relates
the primary variable of interest (also called the
response variable or the dependent variable) to
the set of secondary variables (also called
independent variables).  In a plot of the
variables, the response variable usually appears
on the vertical axis, and some independent
variable on the horizontal axis.  A synonym for
fitting is regression.  Most fitting is done using
the least squares criteria (minimizing the sum of
squared deviations).

The fitting (or model-building) process consists
of 5 distinct steps--selecting variables to be included
in the model, selecting the model, fitting the model,
assessing the goodness of fit of a model, and
improving the model.  Other aspects include
fitting with weights, fitting with constraints,
and fitting with non-least squares criteria.

----------------------------------------------------------

6.1--SELECTING VARIABLES FOR TO BE INCLUDED IN THE MODEL

----------------------------------------------------------

6.2.1--MULTI-RUN SEQUENCE PLOTS

----------------------------------------------------------

6.2.2--MULTI-HISTOGRAMS

----------------------------------------------------------

6.2.3--MULTI-SCATTER PLOTS

----------------------------------------------------------

6.2.4--MULTI-ANOP PLOTS

----------------------------------------------------------

6.2.5--CROSS-CORRELATION TABULATION

----------------------------------------------------------

6.2.6--BOX PLOTS

----------------------------------------------------------

6.2.7--CP PLOT

----------------------------------------------------------

6.3--SELECTING A MODEL

----------------------------------------------------------

6.3.1--PLOTTING THE DATA

----------------------------------------------------------

6.3.2--MAKING USE OF REFERENCE CURVES

----------------------------------------------------------

6.4--FITTING A MODEL

----------------------------------------------------------

6.4.1--FITTING MULTI-LINEAR MODELS

----------------------------------------------------------

6.4.2--FITTING NON-LINEAR MODELS

----------------------------------------------------------

6.5--ASSESSING THE GOODNESS OF FIT OF THE MODEL

----------------------------------------------------------

6.5.1--RESIDUAL STANDARD DEVIATION

----------------------------------------------------------

6.5.2--LACK OF FIT F TESTS

----------------------------------------------------------

6.5.3--SUPERIMPOSING RAW DATA AND FITTED CURVE

----------------------------------------------------------

6.5.4--SCATTER PLOTS OF RESIDUALS

----------------------------------------------------------

6.5.5--NORMAL PROBABILITY PLOT OF RESIDUALS

----------------------------------------------------------

6.5.6--4-PLOT OF RESIDUALS

----------------------------------------------------------

6.6--IMPROVING THE MODEL

----------------------------------------------------------

6.6.1--TRANSFORMING TO SIMPLIFY THE MODEL

----------------------------------------------------------

6.6.2--TRANSFORMING TO ACHIEVE HOMOGENEITY

----------------------------------------------------------

6.6.3--TRANSFORMING TO ACHIEVE NORMALITY

----------------------------------------------------------

6.6.4--ADDING NEW VARIABLES

----------------------------------------------------------

6.6.5--CHANGING THE FORM OF THE MODEL

----------------------------------------------------------

6.7--FITTING WITH WEIGHTS

----------------------------------------------------------

6.8--FITTING WITH CONSTRAINTS

----------------------------------------------------------

6.9--FITTING WITH OTHER CRITERIA (E.G., L1 FITTING)

----------------------------------------------------------



































----------  *ANOVA MODELING*  -----------------------

7--ANOVA MODELING

The sub-categories are--

   1. General discussion
   2. Selecting variables to be included
   3. Examining 1-factor models
   4. Examining 2-factor models
   5. Examining 3-factor models
   6. Examining 4-factor models
   7. Examining 5-factor models
   8. Examining 1-factor models with only  2 treatments
   9. Examining 2**k models
  10. Assessing the goodness of fit of the model
  11. Improving the model

To select a menu item, enter 1 to 11.

----------------------------------------------------------

7.1--GENERAL DISCUSSION OF ANOVA

----------------------------------------------------------

7.2--SELECTING VARIABLES FOR TO BE INCLUDED IN THE MODEL

----------------------------------------------------------

7.2.1--MULTI-RUN SEQUENCE PLOTS

----------------------------------------------------------

7.2.2--MULTI-HISTOGRAMS

----------------------------------------------------------

7.2.3--MULTI-SCATTER PLOTS

----------------------------------------------------------

7.2.4--MULTI-ANOP PLOTS

----------------------------------------------------------

7.3--EXAMINING 1-FACTOR MODELS

----------------------------------------------------------

7.3.1--1-WAY ANOVA

----------------------------------------------------------

7.3.2--1-WAY GANOVA

----------------------------------------------------------

7.3.3--SCATTER PLOTS

----------------------------------------------------------

7.3.4--BOX PLOT

----------------------------------------------------------

7.3.5--ANOP LINE PLOT

----------------------------------------------------------

7.3.6--ANOP CHARACTER PLOT

----------------------------------------------------------

7.3.7--I PLOT

----------------------------------------------------------

7.3.8--DISTRIBUTION-FREE TESTS

----------------------------------------------------------

7.3.9--CORRELATION

----------------------------------------------------------

7.3.10--CATEGORICAL DATA ANALYSIS

----------------------------------------------------------

7.3.11--CROSS-CORRLEATION

----------------------------------------------------------

7.3.12--DISCRETE CONTOUR PLOT

----------------------------------------------------------

7.3.13--FREQUENCY TABULATION

----------------------------------------------------------

7.3.14--CROSS-TABULATION

----------------------------------------------------------

7.3.15--CHI-SQUARED

----------------------------------------------------------

7.4--EXAMINING 2-FACTOR MODELS

----------------------------------------------------------

7.4.1--2-WAY ANOVA

----------------------------------------------------------

7.4.2--2-WAY GANOVA

----------------------------------------------------------

7.4.3--MEDIAN POLISH

----------------------------------------------------------

7.4.4--MULTI-TRACE PLOTS

----------------------------------------------------------

7.4.5--3-D PLOT

----------------------------------------------------------

7.4.6--SPIKE PLOTS

----------------------------------------------------------

7.5--EXAMINING 3-FACTOR MODELS

----------------------------------------------------------

7.5.1--3-WAY ANOVA

----------------------------------------------------------

7.5.2--3-WAY GANOVA

----------------------------------------------------------

7.5.3--MULTI-CELL PLOTS

----------------------------------------------------------

7.6--EXAMINING 4-FACTOR MODELS

----------------------------------------------------------

7.6.1--4-WAY ANOVA

----------------------------------------------------------

7.6.2--MULTI-PLOT 2-WAY GANOVA

----------------------------------------------------------

7.6.3--MULTI-PLOT 3-WAY GANOVA

----------------------------------------------------------

7.7--EXAMINING 5-FACTOR MODELS

----------------------------------------------------------

7.7.1--5-WAY ANOVA

----------------------------------------------------------

7.7.2--MULTI-PLOT 3-WAY GANOVA

----------------------------------------------------------

7.8--EXAMINING 1-FACTOR MODELS WITH ONLY 2 TREATMENTS

----------------------------------------------------------

7.8.1--1-WAY ANOVA

----------------------------------------------------------

7.8.2--T TEST

----------------------------------------------------------

7.8.3--BIHISTOGRAM

----------------------------------------------------------

7.9--EXAMINING 2**K MODELS

----------------------------------------------------------

7.9.1--SQUARE PLOTS, CUBE PLOTS, ETC.

----------------------------------------------------------

7.10--ASSESSING THE GOODNESS OF FIT OF THE MODEL

----------------------------------------------------------

7.10.1--RESIDUAL STANDARD DEVIATION

----------------------------------------------------------

7.10.2--LACK OF FIT F TESTS

----------------------------------------------------------

7.10.3--GANOVA, PARALLELISM, AND NON-ADDITIVITY

----------------------------------------------------------

7.10.4--SUPERIMPOSING RAW DATA AND FITTED CURVE

----------------------------------------------------------

7.10.5--SCATTER PLOTS OF RESIDUALS

----------------------------------------------------------

7.10.6--NORMAL PROBABILITY PLOT OF RESIDUALS

----------------------------------------------------------

7.10.7--4-PLOT OF RESIDUALS

----------------------------------------------------------

7.11--IMPROVING THE MODEL

----------------------------------------------------------

7.11.1--RESIDUAL STANDARD DEVIATIONS FOR SUB-MODELS

----------------------------------------------------------

7.11.2--F TESTS FOR SUB-MODELS

----------------------------------------------------------

7.11.3--TRANSFORMING TO SIMPLIFY THE MODEL

----------------------------------------------------------

7.11.4--TRANSFORMING TO ACHIEVE ADDITIVITY

----------------------------------------------------------

7.11.5--TRANSFORMING TO ACHIEVE HOMOGENEITY

----------------------------------------------------------

7.11.6--TRANSFORMING TO ACHIEVE NORMALITY

----------------------------------------------------------

7.11.7--OMITTING VARIABLES FROM THE MODEL

----------------------------------------------------------

7.11.7.1--F TESTS FOR SUB-MODELS

----------------------------------------------------------

7.11.8--SELECTING ADDITIONAL VARIABLES FOR THE MODEL

----------------------------------------------------------

7.11.8.1--SCATTER PLOTS OF RESIDUALS ON NEW

----------------------------------------------------------

7.11.8.2--BOX PLOTS OF RESIDUALS ON NEW VAR

----------------------------------------------------------

7.11.9--CHANGING THE FORM OF THE MODEL

















































































































----------  *MULTIVARIATE ANALYSIS*  -----------------------

8--MULTIVARIATE ANALYSIS

The sub-categories are--

   1. General discussion
   2. Cluster analysis
   3. Discriminant analysis
   4. Principal component analysis
   5. Canonical analysis
   6. Testing multivariate normality--Q-Q plot

To select a menu item, enter 1 to 6.

----------------------------------------------------------

8.1--GENERAL DISCUSSION

----------------------------------------------------------

8.2--CLUSTER ANALYSIS

----------------------------------------------------------

8.3--DISCRIMINANT ANALYSIS

----------------------------------------------------------

8.4--PRINCIPAL COMPONENT ANALYSIS

----------------------------------------------------------

8.5--CANONICAL ANALYSIS

----------------------------------------------------------

8.6--TESTING MULTIVARIATE NORMALITY--Q-Q PLOT






























































----------  *PROBABILITY ANALYSIS*  ----------------------

9--PROBABILITY ANALYSIS

The sub-categories are--

   1. General discussion
   2. Generating random numbers/simulation/Monte Carlo
   3. Computing percent points
   4. Computing probability density functions
   5. Computing cumulative distribution functions
   6. Plotting percent points
   7. Plotting probability density functions
   8. Plotting cumulative distribution functions
   9. Superimposing Prob. density function on histograms

To select a menu item, enter 1 to 9.

----------------------------------------------------------

9.1--GENERAL DISCUSSION

----------------------------------------------------------

9.2--GENERATING RANDOM NUMBERS/SIMULATION/MONTE CARLO

----------------------------------------------------------

9.3--COMPUTING PERCENT POINTS

----------------------------------------------------------

9.4--COMPUTING PROBABILITY DENSITY FUNCTIONS

----------------------------------------------------------

9.5--COMPUTING CUMULATIVE DISTRIBUTION FUNCTIONS

----------------------------------------------------------

9.6--PLOTTING PERCENT POINTS

----------------------------------------------------------

9.7--PLOTTING PROBABILITY DENSITY FUNCTIONS

----------------------------------------------------------

9.8--PLOTTING CUMULATIVE DISTRIBUTION FUNCTIONS

----------------------------------------------------------

9.9--SUPERIMPOSING PROBABILITY DENSITY FUNTIONS ON HISTOGRAMS















































----------  *QUALITY CONTROL*  -----------------------

10--QUALITY CONTROL

The sub-categories are--

   1. General discussion
   2. Testing for trends
   3. Testing for shifts in location
   4. Testinf for shifts in variation
   5. Testing for outliers
   6. Interlaboratory testing

To select a menu item, enter 1 to 6.

----------------------------------------------------------

10.1--GENERAL DISCUSSION

----------------------------------------------------------

10.2--TESTING FOR TRENDS

----------------------------------------------------------

10.2.1--RUN SEQUENCE PLOT

----------------------------------------------------------

10.2.2--MEAN CONTROL CHART

----------------------------------------------------------

10.3--TESTING FOR SHIFTS IN LOCATION

----------------------------------------------------------

10.3.1--RUN SEQUENCE PLOT

----------------------------------------------------------

10.3.2--MEAN CONTROL CHART

----------------------------------------------------------

10.4--TESTING FOR SHIFTS IN VARIATION

----------------------------------------------------------

10.4.1--RANGE CONTROL CHART

----------------------------------------------------------

10.4.2--STANDARD DEVIATION CONTROL CHART

----------------------------------------------------------

10.5--TESTING FOR OUTLIERS

----------------------------------------------------------

10.6--INTERLABORATORY TESTING

----------------------------------------------------------

10.6.1--YOUDEN PLOTS


































----------  *DISTRIBUTION-FREE ANALYSIS*  ----------------

11--DISTRIBUTION-FREE ANALYSIS

The sub-categories are--

   1. General discussion
   2. testing for randomness
   3. Testing for fixed location (no shifts)
   4. Testing for fixed variation (homoscedasticity)
   5. Testing for goodness of fit of a distribution
   6. Testing for correlation
   7. ANOVA modeling

To select a menu item, enter 1 to 7.

----------------------------------------------------------

11.1--GENERAL DISCUSSION

----------------------------------------------------------

11.2--TESTING FOR RANDOMNESS

----------------------------------------------------------

11.2.1--RUNS ANALYSIS

----------------------------------------------------------

11.2.2--SIGN TEST

----------------------------------------------------------

11.2.3--MEDIAN TEST

----------------------------------------------------------

11.3--TESTING FOR FIXED LOCATION (NO SHIFTS)

----------------------------------------------------------

11.3.1--SIGN TEST

----------------------------------------------------------

11.4--TESTING FOR FIXED VARIATION (HOMOSCEDASTICITY)

----------------------------------------------------------

11.4.1--SIGN TEST ON FIRST DIFFERENCES

----------------------------------------------------------

11.5--TESTING FOR GOODNESS OF FIT OF A DISTRIBUTION

----------------------------------------------------------

11.5.1--KOLMOGOROV-SMIRNOFF TEST

----------------------------------------------------------

11.6--TESTING FOR CORRELATION

----------------------------------------------------------

11.6.1--RANK CORRELATION COEFFICIENT

----------------------------------------------------------

11.7--ANOVA MODELING

----------------------------------------------------------

11.7.1--ANOVA ON RANKS

----------------------------------------------------------

11.7.2--MANN-WHITNEY TESTS





















----------  *ROBUST ANALYSIS*  -----------------------

12--ROBUST ANALYSIS

The sub-categories are--

   1. General discussion
   2. Univariate estimation
   3. Smoothing
   4. Fitting
   5. Anova modeling

To select a menu item, enter 1 to 5.

----------------------------------------------------------

12.1--GENERAL DISCUSSION

----------------------------------------------------------

12.2--UNIVARIATE ESTIMATION

----------------------------------------------------------

12.2.1--ROBUST LOCATION ESTIMATION

----------------------------------------------------------

12.2.2--ROBUST SCALE ESTIMATION

----------------------------------------------------------

12.3--SMOOTHING

----------------------------------------------------------

12.3.1--MEDIAN SMOOTHING

----------------------------------------------------------

12.3.2--ROBUST SMOOTHING

----------------------------------------------------------

12.4--FITTING

----------------------------------------------------------

12.4.1--L1 FITTING

----------------------------------------------------------

12.5--ANOVA MODELING

----------------------------------------------------------

12.5.1--ANOVA ON RANKS

----------------------------------------------------------

12.5.2--MEDIAN POLISH







































----------  *EXPLORATORY DATA ANALYSIS*  -----------------

13--EXPLORATORY DATA ANALYSIS

The sub-categories are--

   1. General discussion
   2. Run sequence plots
   3. Lag plots
   4. Stem and leaf diagrams
   5. Histograms
   6. Normal probability plots
   7. Scatter plots
   8. Box plots

To select a menu item, enter 1 to 8.

----------------------------------------------------------

13.1--GENERAL DISCUSSION

----------------------------------------------------------

13.2--RUN SEQUENCE PLOTS

----------------------------------------------------------

13.3--LAG PLOTS

----------------------------------------------------------

13.4--STEM AND LEAF DIAGRAMS

----------------------------------------------------------

13.5--HISTOGRAMS

----------------------------------------------------------

13.6--NORMAL PROBABILITY PLOTS

----------------------------------------------------------

13.7--SCATTER PLOTS

----------------------------------------------------------

13.8--BOX PLOTS

----------------------------------------------------------


















































----------  *OPTIMAL ESTIMATION (NOT CURRENTLY USED)*  ---
THE FOLLOWING IS FOR OPTIMAL ESTIMATION (NOT CURRECTLY USED)



SSSSS  *1$ MENU FOR UNIVARIATE ANALYSIS*  ------------------




SSSSS  *SDE1$ SPECIFY DISTRIBUTION FOR LOC. & SCALE ESTIMATION*



The best estimator for the location and scale
parameters depends on the distribution of the data.
What distribution does your data have?

   1. normal                 14. exponential
   2. uniform                15. gamma
   3. logistic               16. beta
   4. double exponential     17. Weibull
   5. Cauchy                 18. extreme value type 1
   6. Tukey lambda           19. extreme value type 2
   7. semi-circular          20. Pareto
   8. triangular
                             21. binomial
   9. lognormal              22. geometric
  10. halfnormal             23. Poisson
  11. t                      24. negative binomial
  12. chi-squared
  13. F                      25. unknown
SSSSS


25.
E1NO$
E1UN$
E1LO$
E1DE$
E1CA$
S3TU$
E1SC$
E1TR$
E1LN$
E1HN$
E1ST$
E1CS$
E1SF$
E1EX$
S3GA$
E1BE$
S3WE$
E1E1$
S3E2$
S3PA$
E1BI$
E1GE$
E1PO$
E1NB$
EDDF$

SSSSS  *E1NO$ ESTIMATE LOC. & SCALE FOR NORMAL DISTRIBUTION



The normal distribution has density function

   f(x) = (1/(sqrt(2*pi)*b)) * exp(-0.5*z**2)
          where z = (x-a)/b

The optimal estimate for the location parameter a
is the sample mean.
The near-optimal estimate for the scale parameter b
is the sample standard deviation.
To compute them, use the ... MAXIMUM LIKELIHOOD
ESTIMATION command, as in

   NORMAL MAXIMUM LIKELIHOOD ESTIMATION Y

To compute other estimates of location and
scale, use the SUMMARY command, as in

   SUMMARY Y

After computing the above, enter    EXPERT NEXT
for further questions regarding the validity
of the estimate.
SSSSS


1.0
CVE1$

SSSSS  *E1UN$ ESTIMATE LOC. & SCALE FOR UNIFORM DISTRIBUTION



The uniform distribution has density function

   f(x) = 1/b   for a-(b/2) <= x <= a+(b/2)
        = 0     for x outside of this interval

The optimal estimate for the location parameter a
is the sample midrange = (min + max)/2.
The optimal estimate for the scale parameter b
is the sample range = max - min.
To compute them, use the ... MAXIMUM LIKELIHOOD
ESTIMATION command, as in

   UNIFORM MAXIMUM LIKELIHOOD ESTIMATION Y

To compute other estimates of location and
scale, use the SUMMARY command, as in

   SUMMARY Y

After computing the above, enter    EXPERT NEXT
for further questions regarding the validity
of the estimates.
SSSSS


1.0
CVE1$

SSSSS  *E1LO$ ESTIMATE LOC. & SCALE FOR LOGISTIC DISTRIBUTION



The logistic distribution has density function

   f(x) = (1/b) * exp(z) / (1+exp(z))
          where z = (x-a)/b

To compute optimal estimates for the location
parameter a and the scale parameter b,
use the ... MAXIMUM LIKELIHOOD ESTIMATION
command, as in

   LOGISTIC MAXIMUM LIKELIHOOD ESTIMATION Y

To compute other estimates of location and
scale, use the SUMMARY command, as in

   SUMMARY Y

After computing the above, enter    EXPERT NEXT
for further questions regarding the validity
of the estimate.
SSSSS


1.0
CVE1$

SSSSS  *E1DE$ ESTIMATE LOC. & SCALE FOR DOUBLE EXPONENTIAL DISTRIBUTION



The double exponential (= Laplace) distribution has
density function

   f(x) = (1/b) * 0.5 * exp(-abs(z))
          where z = (x-a)/b

To compute optimal estimates for the location
parameter a and the scale parameter b,
use the ... MAXIMUM LIKELIHOOD ESTIMATION
command, as in

   DOUBLE EXPONENTIAL MAXIMUM LIKELIHOOD ESTIMATION Y

To compute other estimates of location and
scale, use the SUMMARY command, as in

   SUMMARY Y

After computing the above, enter    EXPERT NEXT
for further questions regarding the validity
of the estimate.
SSSSS


1.0
CVE1$

SSSSS  *E1CA$ ESTIMATE LOC. & SCALE FOR CAUCHY DISTRIBUTION



The Cauchy distribution has density function

   f(x) = (1/b) * (1/pi) * 1/(1 + z**2)
          where z = (x-a)/b

To compute optimal estimates for the location
parameter a and the scale parameter b,
use the ... MAXIMUM LIKELIHOOD ESTIMATION
command, as in

   CAUCHY MAXIMUM LIKELIHOOD ESTIMATION Y

To compute other estimates of location and
scale, use the SUMMARY command, as in

   SUMMARY Y

After computing the above, enter    EXPERT NEXT
for further questions regarding the validity
of the estimate.
SSSSS


1.0
CVE1$

SSSSS  *S3TU$ SPECIFY TAIL LENGTH PARAMETER FOR TUKEY LAMBDA DISTRIBUTION



The Tukey lambda distribution has no simple closed-
form density function.  It has percent point function
(= inverse cumulative distribution function)

   G(p) = a + b * [(p**lambda - (1-p)**lambda) / lambda]
          where z = (x-a)/b

The optimal estimate for the location parameter a
and the scale parameter b depends on the value
of the tail length parameter lambda.

Do you have a value for lambda?

   If yes, then enter YES, Y, or 1
   if no,  then enter NO,  N, or 2
SSSSS


2.0
E1TU$
E3TU$

SSSSS  *E1TU$ ESTIMATE LOC. & SCALE FOR TUKEY LAMBDA DISTRIBUTION



The Tukey lambda distribution has no simple closed-
form density function.  It has percent point function
(= inverse cumulative distribution function)

   G(p) = a + b * [(p**lambda - (1-p)**lambda) / lambda]
          where z = (x-a)/b

The near-optimal estimate for the location parameter a
is the sample xxx.
The optimal estimate for the scale parameter b
is the sample xxx.
To compute them, use the LET command, as in

   LET A = XXX Y
   LET B = XXX Y

To compute other estimates of location and
scale, use the SUMMARY command, as in

   SUMMARY Y

After computing the above, enter    EXPERT NEXT
for further questions regarding the validity
of the estimate.
SSSSS


1.0
CVE1$

SSSSS  *E3TU$ ESTIMATE TAIL LENGTH FOR TUKEY LAMBDA DISTRIBUTION



To graphically estimate the value of lambda
for your data, use a PPCC plot (= Probability
Plot Correlation Coefficient plot).  To generate
a PPCC plot, use the PPCC PLOT command, as in

   TUKEY LAMBDA PPCC PLOT Y

The resulting horizontal axis will be values of lambda
(with default range from -2 to +2).  Each value
of lambda represents a different member of the
Tukey lambda distributional family.

The vertical axis is a measure of how good a
distributional fit each value of lambda provides
(for details enter HELP PPCC PLOT).

The graphical estimate of lambda is the value on
the horizontal axis where the trace attains a max.

After generating the above plot, enter    EXPERT NEXT
for further questions.
SSSSS


1.0
E4TU$

SSSSS  *E4TU$ CHANGE PPCC LIMITS FOR TUKEY LAMBDA DISTRIBUTION



The default range of lambda values on the
horizontal axis of the Tukey PPCC plot is -2 to +2.
If this range is inappropriate, or if you wish
to gain an "extra decimal place" in the graphical
estimate by blowing up the region around the max,
then this may be done via the LET command, as in

   LET LAMBDA1 = 0.1
   LET LAMBDA2 = 0.2
   TUKEY PPCC PLOT Y

After generating the above plot, enter    EXPERT NEXT
for further questions.
SSSSS


1.0
SBE4$

SSSSS  *CVE1$ CHECK VALIDITY OF LOC. & SCALE ESTIMATOR



The validity of the estimator and the validity
of the uncertainty attached to the estimator
is dependent on various underlying
assumptions.  Which underlying assumptions
do you wish to check?

   1. Randomness
      Are the data uncorrelated?

   2. Correct model
      Are your sure    response = constant + error ?

   3. Fixed variation (scale)
      Is the variation in the response the
      same for all data points?

   4. Fixed, correct distribution--
      Does the data follow only one distribution?
      Is it the distribution you specified?

   5. None
SSSSS


5.
CVRA$
CVLA$
CVVA$
CVDA$
SBE1$

SSSSS  *SBE1$ SPECIFY BRANCH AFTER LOC. & SCALE ESTIMATION



Do you wish further consulting questions?
You may resume the consulting session at any of
the following points in the consulting "tree"--

   1. Check validity of underlying assumptions

   2. Specify another dist. for loc./scale estimation

   3. Specify another type of univariate analysis

   4. Specify another type of data analysis

   5. None of the above--exit out of consulting mode.
SSSSS


5.
CVUA$
SDE1$
STUA$
STDA$
DONE$

SSSSS  *SBE4$ SPECIFY BRANCH AFTER TAIL LENGTH ESTIMATION



Do you wish further consulting questions?
You may resume the consulting session at any of
the following points in the consulting "tree"--

   1. Estimate location/scale for this distribution

   2. Specify another dist. for loc./scale estimation

   3. Check validity of underlying assumptions

   4. Specify another type of univariate analysis

   5. Specify another type of data analysis

   6. None of the above--exit out of consulting mode.
SSSSS


6.
E1XX$
SDE1$
CVUA$
STUA$
STDA$
DONE$

-ENDSSSSS*-----


////////////
Univariate Analysis
   Computing Summary/Descriptive Statistics
      Location Estimation
      Variation (Scale) Estimation
      Skewness Estimation
      Tail Length Estimation
      Autocorrelation estimation
   Determining General Distributional Characteristics
      Frequency Tabulation
      Histograms and Cumulative Histograms
      Stem and Leaf Diagrams
      Frequency Plots and Cumulative Frequency Plots
      Percent Point Plots
      Pie Charts
   Selecting a "Good-Fitting" Distribution
      Probability Plots
      PPCC Plots
      Maximum Likelihood Estimation
   Estimating the Parameters of the Distribution
      Maximum Likelihood Estimation
      Robust Estimates
   Assessing the Goodness of Fit of a Distribution
      Superimposing Probability Density Function and Histogram
      Superimposing Root Density Function and Rootogram
      Probability Plot
      Chi-Squared Test
      Kolmogorov-Smirnoff Test
   Testing Underlying Assumptions
      4-Plot Analysis
   Testing for Randomness
      Lag Plot
      Runs Analysis
      Distribution-Free Tests
      Autocorrelation Plot
      Spectral Plot
   Testing for Fixed Location (No Shifts in Location)
      t Test
      Distribution-Free Tests
   Testing for Fixed Variation (Homoscedasticity)
      Homoscedasticity Plot
   Transforming to Homoscedasticity
      Box-Cox Homoscedasticity Plot
      Chi-squared Tests
   Testing for Fixed Distribution
      Bihistogram
      4-Plot Analysis
      Distribution-Free Tests
      Homoscedasticity Plot
   Testing for Symmetry
      Symmetry Plot
   Transforming to Symmetry
      Box-Cox Symmetry Plot
   Testing for Normality
      Normal Probability Plot
      Tukey PPCC Plot
      t PPCC Plot
   Transforming to Normality
      Box-Cox Normality Plot
   Testing for Normal Outliers
   Determining Confidence Limits for Distributional Parameters
   Hypothesis Testing on Distributional Parameters
   Probit Analysis???

Time Series Analysis (1 Variable)
   Checking for Time-Domain Structure
      Run Sequence Plot
      Lag Plot
      Autocorrelation Plot
      Partial-Autocorrelation Plot
      Complex Demodulation Plots
   Checking for Frequency-Domain Structure
      Spectral Plot
      Periodogram
   Checking for Time and Frequency Domain Structure
      4-Plot Analysis
   Testing White Noise (Randomness)
      Lag Plot
      Runs Analysis
      Distribution-Free Tests
      Autocorrelation Plot
      Spectral Plot
      4-Plot Analysis
   Checking for Trends
      Run Sequence Plot
      Correlation With Time
      Linear Fit Over Time
   Fitting Box-Jenkins Models
      Lag Plot
      Autocorrelation Plot
      Partial Autocorrelation Plot
   Smoothing
      Moving Average Smoothing
      Least Squares Smoothing
      Median Smoothing
      Robust Smoothing
      Exponential Smoothing
      Assessing the Goodness of the Smoothing
         Residual Standard Deviation
         Superimposing Raw Data and Fitted Curve
         Scatter Plots of Residuals
         Normal Probability Plot of Residuals
         4-Plot of Residuals
   Filtering
      Low-Pass Filters
      High-Pass Filters
      Assessing the Goodness of the Filtering

Time Series Analysis--2 Variables
   Checking for Time-Domain Structure
      Scatter Plot
      Multi-Trace Plots
      Cross-Spectral Plot
      Bihistogram
   Checking for Frequency-Domain Structure
      Cross-Spectrum
      Coherency Spectrum
      Quadrature Spectrum
      Co-Spectrum
      Gain Spectrum
      Argand Spectrum
   Checking for Time and Frequency Domain Structure
      4-Plot Analysis

Correlation Analysis
   Multi-Scatter Plots
   Multi-ANOP Plots
   Multi-Box Plots
   Cross-Correlation Tabulation
   Transforming Variables
   Distribution-free Tests

Fitting (1 Independent Variable)
   Selecting a Model
      Plotting the Data
      Making Use of Reference Curves
   Fitting a Model
      Fitting Linear Models
      Fitting Polynomial Models
      Fitting Non-Linear Models
      Fitting Splines
      Fitting Rational Functions
   Assessing the Goodness of Fit of the Model
      Residual Standard Deviation
      Lack of Fit F Tests
      Superimposing Raw Data and Fitted Curve
      Scatter Plots of Residuals
      Normal Probability Plot of Residuals
      4-Plot of Residuals
   Improving the Model
      Transforming to Simplify the Model
      Transforming to Achieve Homogeneity
      Transforming to Achieve Normality
      Adding New Variables
      Changing the Form of the Model
   Fitting With Weights
   Fitting With Constraints
   Fitting With Other Criteria (e.g., L1 Fitting)

Fitting (2 or More Independent Variable)
   Selecting Variables for To Be Included in the Model
      Multi-Run Sequence Plots
      Multi-Histograms
      Multi-Scatter Plots
      Multi-ANOP Plots
      Cross-Correlation Tabulation
      Box Plots
      Cp Plot
   Selecting a Model
      Plotting the Data
      Making Use of Reference Curves
   Fitting a Model
      Fitting Multi-Linear Models
      Fitting Non-Linear Models
   Assessing the Goodness of Fit of the Model
      Residual Standard Deviation
      Lack of Fit F Tests
      Superimposing Raw Data and Fitted Curve
      Scatter Plots of Residuals
      Normal Probability Plot of Residuals
      4-Plot of Residuals
   Improving the Model
      Transforming to Simplify the Model
      Transforming to Achieve Homogeneity
      Transforming to Achieve Normality
      Adding New Variables
      Changing the Form of the Model
   Fitting With Weights
   Fitting With Constraints
   Fitting With Other Criteria (e.g., L1 Fitting)

ANOVA Modeling
   Selecting Variables for To Be Included in the Model
      Multi-Run Sequence Plots
      Multi-Histograms
      Multi-Scatter Plots
      Multi-ANOP Plots
   Examining 1-Factor Models
      1-Way ANOVA
      1-Way GANOVA
      Scatter Plots
      Box Plot
      ANOP Line Plot
      ANOP Character Plot
      I Plot
      Distribution-free Tests
      Correlation
      Categorical Data Analauysis
      Cross-Corrleation
      Discrete Contour Plot
      Frequency Tabulation
      Cross-Tabulation
      Chi-squared
   Examining 2-Factor Models
      2-Way ANOVA
      2-Way GANOVA
      Median Polish
      Multi-Trace Plots
      3-D Plot
      Spike Plots
   Examining 3-Factor Models
      3-Way ANOVA
      3-Way GANOVA
      Multi-Cell Plots
   Examining 4-factor Models
      4-Way ANOVA
      Multi-Plot 2-Way GANOVA
      Multi-Plot 3-Way GANOVA
   Examining 5-Factor Models
      5-Way ANOVA
      Multi-Plot 3-Way GANOVA
   Examining 1-Factor Models With Only 2 Treatments
      1-Way ANOVA
      t Test
      Bihistogram
   Examining 2**k Models
      Square Plots, Cube Plots, etc.
   Assessing the Goodness of Fit of the Model
      Residual Standard Deviation
      Lack of Fit F Tests
      GANOVA, Parallelism, and Non-Additivity
      Superimposing Raw Data and Fitted Curve
      Scatter Plots of Residuals
      Normal Probability Plot of Residuals
      4-Plot of Residuals
   Improving the Model
      Residual Standard Deviations For Sub-Models
      F Tests For Sub-Models
      Transforming to Simplify the Model
      Transforming to Achieve Additivity
      Transforming to Achieve Homogeneity
      Transforming to Achieve Normality
      Omitting Variables From the Model
         F Tests for Sub-Models
      Selecting Additional Variables For the Model
         Scatter Plots of Residuals on New Variables
         Box Plots of Residuals on New Variables
      Changing the Form of the Model

Multivariate Analysis
   Cluster Analysis
   Discriminant Analysis
   Principal Component Analysis
   Canonical Analysis
   Testing Multivariate Normality--Q-Q Plot

Probability Analysis
   Generating Random Numbers/Simulation/Monte Carlo
   Computing Percent Points
   Computing Probability Density Functions
   Computing Cumulative Distribution Functions
   Plotting Percent Points
   Plotting Probability Density Functions
   Plotting Cumulative Distribution Functions
   Superimposing Probability Density Funtions on Histograms

Quality Control
   Testing for Trends
      Run Sequence Plot
      Mean Control Chart
   Testing for Shifts in Location
      Run Sequence Plot
      Mean Control Chart
   Testing for Shifts in Variation
      Range Control Chart
      Standard Deviation Control Chart
   Testing for Outliers
   Interlaboratory Testing
      Youden Plots

Exploratory Data Analysis
   Run Sequence Plots
   Lag Plots
   Stem and Leaf Diagrams
   Histograms
   Normal Probability Plots
   Scatter Plots
   Box Plots

Robust Analysis
   Univariate Estimation
      Robust Location Estimation
      Robust Scale Estimation
   Smoothing
      Median Smoothing
      Robust Smoothing
   Fitting
      L1 Fitting
   ANOVA Modeling
      ANOVA On Ranks
      Median Polish

Distribution-Free Analysis
   Testing for Randomness
      Runs Analysis
      Sign Test
      Median Test
   Testing for Fixed Location (No Shifts)
      Sign Test
   Testing for Fixed Variation (Homoscedasticity)
      Sign Test on First Differences
   Testing for Goodness of Fit of a Distribution
      Kolmogorov-Smirnoff Test
   Testing for Correlation
      Rank Correlation Coefficient
   ANOVA Modeling
      ANOVA On Ranks
      Mann-Whitney Tests


















