----------  *PERCENT POINT FUNCTIONS* ----------

              Percent Point Functions
     (Inverse Cumulative Distribution Functions)
             for Various Distributions

 - - - - - - - - - - - - - - - - - - - - - - - - -
              Symmetric Distributions

f(x) denotes probability density function
G(p) denotes percent point function


Uniform          f(x) = 1    with x in (0,1)
Uniform          G(p) = 1

Normal           f(x) = (1/sqrt(2pi)) * exp(-0.5*x**2)
Normal           G(p) not in closed form

Logistic         f(x) = exp(x) / (1 + exp(x))**2
Logistic         G(p) = log(p/(1-p))

Double Exp.      f(x) = 0.5 * exp(-x)
Double Exp.       G(p) = log(2p)         for p <= 0.5
Double Exp.            = -log(2(1-p))    for p  > 0.5

Cauchy           f(x) = (1/pi) * 1/(1+x**2)
Cauchy           G(p) = -cot(pi*p)

Tukey Lambda     f(x) not in closed form
Tukey Lambda     G(p) = (p**lambda - (1-p)**(1-lambda)) / lambda
Tukey Lambda            if lambda not = 0
Tukey Lambda          = log(p/(1-p))
Tukey Lambda            if lambda = 0
Tukey Lambda     f(x) not in closed form
Tukey Lambda          lambda = 0.5 --U-shaped
Tukey Lambda          lambda = 1.0 --exactly uniform
Tukey Lambda          lambda = 0.14--approximately normal
Tukey Lambda          lambda = 0.0 --exactly logistic
Tukey Lambda          lambda = -1.0--approximately Cauchy

Anglit           f(x) = sin(2x+pi/2)    with x in (-pi/4,pi/4)
Anglit           G(p) = arcsin(sqrt(p)) - pi/4

Triangular       f(x) = 1 - abs(x)   with x in (-1,1)
Triangular       G(p) = -1 + sqrt(2p)       for p <= 0.5
Triangular            = +1 - sqrt(2(1-p))   for p > 0.5

Arcsin           f(x) = (1/pi) * 1 / sqrt(x(1-x))
Arcsin           G(p) = (sin(pi*p/2))**2

Student t        f(x) = c / [1 + (x*x/nu)]**((nu+1)/2)
Student t        G(p) = no simple general closed form

 - - - - - - - - - - - - - - - - - - - - - - - - -
                Skewed Distributions

Chi-squared      f(x) = c * [x**((nu/2)-1)] * exp(-x/2) with x >= 0
Chi-squared             with x >= 0 and nu > 0
Chi-squared             and where c = gamma function of nu/2
Chi-squared      G(p) = no simple general closed form

Gamma            f(x) = c * [x**(gamma-1)] * exp(-x) with x >= 0
Gamma                   with x >= 0 and gamma > 0
Gamma                   and where c = gamma function of gamma
Gamma            G(p) = no simple general closed form

Exponential      f(x) = exp(-x)       with x >= 0
Exponential      G(p) = -log(1-p)

Lognormal        f(x) = (1/(x*sqrt(2*pi))) * exp(-0.5*(log(x))**2)
Lognormal        with x >= 0
Lognormal        G(p) = exp(GN(p))
Lognormal               where GN(p) is normal N(0,1) ppf

Half-Normal      f(x) = (2/sqrt(2*pi)) * exp(-0.5*x**2)
Half-Normal      with x >= 0
Half-Normal      G(p) = GN((p+1)/2)
Half-Normal             where GN(p) is normal N(0,1) ppf

Extreme Value I  f(x) = exp(-x) * exp(-exp(-x))
Extreme Value I  G(p) = -log(-log(p))

Extreme Value II f(x) = gamma * x**(-gamma-1) * exp(-(x**(-gamma))
Extreme Value II        with x >= 0 and gamma > 0
Extreme Value II G(p) = (-log(p))**(-1/gamma)

Weibull          f(x) = gamma * x**(gamma-1) * exp(-(x**gamma))
Weibull                 with x > 0 and gamma > 0
Weibull          G(p) = (-log(1-p))**(1/gamma)

Pareto           f(x) = gamma / (x**(gamma+1))
Pareto           G(p) = (1-p)**(-1/gamma)

Beta             f(x) = c * x**(a-1) * (1-x)**(b-1) with x in (0,1)
Beta                    where c = beta function of a and b
Beta             G(p) = no simple closed form

 - - - - - - - - - - - - - - - - - - - - - - - - -
               Discrete Distributions

Binomial

Geometric

Poisson

