**comment This is file   DPSNAPSH.TEX
** --------------------
**linesperinch 11
**scale 30
**font 2
** --------------------
**magnify 3.0
**lines per inch 4
**move 0325 2950
A Snapshot of Selected Dataplot Commands
 
**center off
**magnify 1.0
**lines per inch 8
**tab on
**tab number 02
**tab1 01 1150
**tab2 01 1550
**move 2230 2815
10/92
 
**comment ------start of page 1----------
**move 0325 2795
**box 0300 0400 2300 2797 0.0
 
**revvideo 0304 1000 1.0
1. Dataplot Basics
 
**revvideo 0304 0850 0.9
1.1 System Operations
 
DP
@Execute Dataplot
EXIT or STOP or HALT or QUIT or BYE
@Exit from Dataplot
 
DIMENSION 50 VARIABLES
@Redimension internal Dataplot worksheet to 50 variables (do at start of run)
FEEDBACK OFF
@Suppress feedback from all support commands (e.g., SUBSET)
ECHO ON
@Upon execution, echo all command lines back to screen
 
LIST (or L)
@List last 20 command lines entered
R  2
@Re-execute the 2-nd previous command
SAVE 12 6 TO 3 1
@Save command lines 12, 6, 5, 4, 3, and 1 for future re-execution
/
@Repeat the last set of command lines saved (& display output on screen)
/  PRINTER  or  /  LP  or  /  LPT1
@Repeat last set of comm. lines saved & divert output from screen to laser printer
 @ (Note--printer should here be in non-postscript DOS text mode)
/  OUT.
@Repeat last set of comm. lines saved & divert output from screen to file OUT.
 
DEFINE ER  ESC  FF
@Define command ER to be   escape formfeed
ER
@Execute command ER (which would erase screen & leave in text mode on PC)
 
CALL PARETO.DP
@Execute the Dataplot macro residing in file PARETO.DP
CALL <user-provided file name>
@Execute the user-written Dataplot macro residing in <user-provided file name>
 
SUBSET/EXCEPT/FOR
@Can append to any graphics & analysis commands to limit scope of execution
        e.g.,
@of a plot or of an analysis
        PLOT Y X SUBSET DAY 2
@Plot Y versus X but only for DAY = 2
        ANOVA Y X1 X2 X3 SUBSET MONTH 7 TO 9
@Do ANOVA of Y on X1, X2, and X3
 @but only for values of MONTH variable from 7 to 9
        LINEAR FIT Y X EXCEPT Y > 90
@Do linear regression of Y on X but only for Y > 90
        NORMAL PROBABILITY PLOT Y FOR I = 1 1 20
@Generate normal probability plot of first 20 elements of Y
 
RESET
@Reset all of Dataplot; in effect, exit and reexecute Dataplot
RESET VARIABLES
@Reset (= delete) all internal variables (= vectors)
RESET PARAMETERS
@Reset (= delete) all internal parameters (= scalars)
RESET FUNCTIONS
@Reset (= delete) all internal functions (= character strings)
RESET PLOT
@Reset (= blank out) all plot titles, labels, and legends
RESET IO
@Reset settings for Dataplot's I/O commands
RESET SUPPORT
@Reset settings for all of Dataplot's Support commands
 
HELP
@Display general Dataplot information & the command categories
 
HELP <command category>
@Display information about the Dataplot <category command> category
HELP GRAPHICS
@Display information about Dataplot Graphics Commands
HELP ANALYSIS
@Display information about Dataplot Analysis Commands
 
HELP <command name>
@Display information about the Dataplot <command name> command
HELP PLOT
@Display information about the Dataplot PLOT command
HELP FIT
@Display information about the Dataplot FIT command
HELP LET
@Display information about the Dataplot LET command
HELP TITLE
@Display information about the Dataplot TITLE command
HELP GRID
@Display information about the Dataplot GRID command
HELP TEXT
@Display information about the Dataplot TEXT command
HELP BOX PLOT
@Display information about the Dataplot BOX PLOT command
 
HELP LET SUBCOMMANDS
@Display information about subcommands under the LET command
HELP FUNCTIONS
@Display information about all off Dataplot's internal library functions
HELP MATH FUNCTIONS
@Display information about only Dataplot's internal mathematics functions
HELP CHARACTER TYPES
@Display information about allowable plot character types
HELP LINES TYPES
@Display information about allowable plot line types
HELP TEXT SUBCOMMANDS
@Display information about subcommands under the TEXT command
 
**comment ------start of page 2----------
**newpage
**move 0325 2795
**box 0300 0400 2300 2797 0.0
 
**revvideo 0304 1000 1.0
1. Dataplot Basics (Continued)
 
**revvideo 0304 0850 0.9
1.2 Files & I/O
 
DOS DIR *.DAT
@Execute the DOS    DIR *.BAT   command while still in Dataplot
 
SKIP 25
@Skip the first 25 header lines of a file on subsequent READs
READ BOXSPRIN.DAT Y X1 X2 X3
@Read data from file BOXSPRIN.DAT into internal variables Y, X1, X2, & 3
READ BOXSPRIN.DAT ACC OVEN CARB QUENCH
@Read data from file BOXSPRIN.DAT into int. var. ACC, OVEN, CARB, & QUENCH
 
WRITE Y X1 X2 X3
@Write variables Y, X1, X2, and X3 to screen
WRITE OUT.DAT Y X1 X2 X3
@Write variables Y, X1, X2, and X3 out to the user file   OUT.DAT
 
LIST (or L)
@Display the last 20 command lines entered onto the screen
LIST BOXSPRIN.DAT
@Display the contents of file   BOXSPRIN.DAT   onto the screen
LIST BOXSPRIN.DAT FOR I = 1 1 20
@Display the first 20 lines of file   BOXSPRIN.DAT onto the screen
 
DOS PRINT BOXSPRIN.DAT
@(DOS only) Temporarily exit to DOS & send file BOXSPRIN.DAT to laser printer
SYSTEM PRINT BOXSPRIN.DAT
@(In general) Temporarily exit to System & send file BOXSPRIN.DAT to laser prin.
 
SET READ FORMAT 4F8.0
@Set format for future READs to be (Fortran-like) 4F8.0 (default = free-format)
SET WRITE DECIMALS 3
@Set format for future WRITEs to be 3 decimal places
SET WRITE FORMAT 4F10.2
@Set format for future WRITEs to be (Fortran-like) 4F10.2
 
CAPTURE OUT.
@Start diverting all screen text output into the file OUT.
   LINEAR FIT Y X
@Do a linear fit of Y versus X (FIT output diverted to OUT.)
END OF CAPTURE
@Stop diverting all screen text output into the file OUT.
DOS PRINT OUT.
@(DOS only) Temporarily exit to DOS & send file OUT. to laser printer
 @(Note--Laser printer must be in DOS-text mode)
 
**comment ------start of page 3----------
**newpage
**move 0325 2795
**box 0300 0400 2300 2797 0.0
 
**revvideo 0304 1000 1.0
1. Dataplot Basics (Continued)
 
**revvideo 0304 0850 0.9
1.3 Data Creation/Manipulation
 
LET TARGET = 20
@Create the parameter (= scalar) TARGET and assign to it the value 20
 
LET Y = DATA 67 79 61 75 59 90 52 87
@Create variable (= vector) Y with 8 elements: 67, 79, 61, ..., 87
LET X = SEQUENCE 1 1 6
@Create variable X with first element = 1, increment = 1, & last element = 6,
 @that is, X will have the 6 elements: 1 2 3 4 5 6
LET X = SEQUENCE 1 3 1 6
@Create variable X with 18 elements: 1 1 1 2 2 2 3 3 3 4 4 4 5 5 5 6 6 6
LET X1 = PATTERN 1 2 FOR I = 1 1 8
@Create variable X1 with 8 elements: 1 2 1 2 1 2 1 2
 
LET Y2 = Y**2
@Create variable Y2 as the squared elements of Y
LET Y3 = LOG(Y)
@Create variable Y3 as the natural logarithm of the elements of Y
LET Y3 = SQRT(Y)
@Create variable Y3 as the square root of the elements of Y
LET DUMMY7 = IND(MONTH,7)
@Create variable DUMMY7 as 1 whenever MONTH = 7, and as 0 elsewhere
 @(thus IND(.,.) is an indicator function)
LET Y2 = MSD(Y)
@Create variable Y2 as the most signficant digit of Y
 
LET Y = NORMAL RANDOM NUMBERS FOR I = 1 1 8
@Create variable Y with 8 random numbers from normal N(0,1) distribution
LET Y = UNIFORM RANDOM NUMBERS FOR I = 1 1 10
@Create variable Y with 10 random numbers from uniform[0,1] distribution
 
LET GAMMA = 2.3
@Create the parameter (= scalar) GAMMA and assign to it the value 2.3
   LET Y = WEIBULL RANDOM NUMBERS FOR I = 1 1 20
@Create variable Y with 20 rand. numb. from Weib. dist. with shape param. 2.3
 
LET X = RANDOM PERMUTATION FOR I = 1 1 16
@Create variable X with a random permutation of 1 to 16
 
LET Y2 = CODE Y
@Create variable Y2 as a coded version of Y, as follows:
 @if Y = smallest, then Y2 = 1; if Y = next-smallest, then Y2 = 2, etc.
 
LET Y2 = CODE2 Y
@Create variable Y2 as a coded version of Y, as follows:
 @if Y <= median value of Y, then Y2 = 1;
 @if Y > median value of Y, then Y2 = 2
 
LET Y2 = CODE4 Y
@Create variable Y2 as a coded version of Y, as follows:
 @if Y in first/second/third/fourth quartile of Y, then Y2 = 1/2/3/4
 
LET Y2 = SORT Y
@Create variable Y2 as the sorted (smallest to largest) version of Y
 
LET Y2 = SORTC Y X1 X2 X3
@Create variable Y2 as the sorted (smallest to largest) version of Y
 @and carry along the corresponding elements in X1, X2, and X3
 
LET Y1 = Y; RETAIN Y1 SUBSET DAY 1
@Copy Y to Y1; then retain (= keep, and pack to top of vector)
 @only those elements in Y for which the corresponding
 @elements of the variable DAY are equal to 1
 
LET Y1 = Y; EXTEND Y1 Y2
@Copy Y to Y1; then extend Y1 by Y2 (= attach Y2 to the end of Y1)
 
LET Y2 = CUMULATIVE SUM Y
@Create a variable Y2 = cumulative sum of elements of Y
 @(thus Y2(1)=Y(1), Y2(2)=Y(1)+Y(2), Y2(3)=Y(1)+Y(2)+Y(3), etc.)
 
LET Y2 = DIFFERENCE Y
@Create a variable Y2 = first difference of elements in the variable Y
 @(thus Y2(1)=Y(2)-Y(1), Y2(2)=Y(3)-Y(2), Y2(3)=Y(4)-Y(3), etc.)
 
DELETE X Y Z
@Delete variables X, Y and Z
 
DELETE X Y Z SUBSET Z = 2
@Delete all elements in X, Y, and Z for which Z = 2
 @(and pack all remaining elements up to topmost positions)
 
DELETE X(1) Y(23) Z(1000)
@Delete element 1 of X, element 23 of Y, and element 1000 of Z
 @(and pack all remaining elements up to topmost positions)
 
**comment ------start of page 4----------
**newpage
**move 0325 2795
**box 0300 0400 2300 2797 0.0
 
**revvideo 0304 1000 1.0
2. Graphics
 
**revvideo 0304 0850 0.9
2.1 Plot Data
 
PLOT Y
@Plot variable Y (vertically) versus dummy index 1, 2, 3, ... (horizontally)
 @(Use current settings of title, label, legend, axis limits,
 @plot characters, plot lines, plot spikes, plot bars, etc.)
 
PLOT Y X  or  PLOT Y VERSUS X  or  PLOT Y VS X
@Plot variable Y (vertically) versus variable X (horizontally)
 
PLOT Y X TAG
@Generate multi-trace plot of Y (vertically) versus X (horizontally)
 @where distinct values of TAG variable define distinct traces
 
PLOT Y1 Y2 Y3 Y4 VERSUS X
@Generate multi-trace plot of variables Y1, Y2, Y3, and Y4 (vertically)
 @versus variable X (horizontally)
 
PLOT Y PRED VERSUS X
@Generate 2-trace plot of Y and post-fit predicted values in PRED
 @(vertically) versus variable X (horizontally)
 
PLOT Y VERSUS X AND
   PLOT PRED VERSUS X
@Generate 2-trace plot of Y and post-fit predicted values in PRED
 @(vertically) versus variable X (horizontally) <alternate method>
 
3D-PLOT Y X1 X2
@Generate a 3-dimensional trace of Y versus X1 and X2
 
3D-PLOT Y X1 X2 X1 AND
@Generate a 3-d trace of Y versus X1 & X2; cross-hatch at each X1 value
   3D-PLOT Y X1 X2 X2
@Generate a 3-d trace of Y versus X1 & X2; cross-hatch at each X2 value
 @(the net result will be a 3-d surface with cross-hatching in both direction)
 
LET X1 = SEQUENCE -5 1 5 FOR I = 1 1 121
@Create variable X1 with elements -5, -4, -3, ..., 4, 5 repeated 11 times
LET X2 = SEQUENCE -5 11 1 5
@Create variable X2 with 11 -5's, 11 -4's, 11 -3's, ..., 11 4's, 11 5's
LET Y = X1**2+X2**2
@Create variable Y = X1 squared + X2 squared
LET Y0 = DATA 10 20 30 40
@Create variable Y0 with 4 elements: 10, 20, 30, and 40
CONTOUR PLOT Y X1 X2 Y0
@Generate a contour plot of Y vs. X1 and X2 with contour lines at Y0
 
**revvideo 0304 0850 0.9
2.2 Plot Functions
 
PLOT SIN(X**2)*EXP(-X) FOR X = 0 .1 10
@Generate plot of function evaluated at X = 0, .1, .2, .3, ..., 10.0
3DPLOT EXP(-X**2-Y**2) FOR X = -2 .2 2 FOR Y = -2 .2 2
@Generate 3-D plot of function evaluated at X = -2.0, -1.8, -1.6, ... 1.8, 2.0
 @and Y = -2.0, -1.8, -1.6, ..., 1.8, 2.0
 
**revvideo 0304 0850 0.9
2.3 Character/Line/Spike/Bar Plots
 
STATUS
@Display current status of worksheet dimensions, plot characters/lines/
 @spikes/bars and existing internal Dataplot variables, parameters, & functions
CHARACTERS X BLANK CIRCLE
@Set plot character types (for future plots) as follows:
 @trace 1 = X; trace 2 = BLANK; trace 3 = CIRCLE
LINES SOLID DOT DASH
@Set plot line types:  trace 1 = SOLID; trace 2 = DOT; trace 3 = DASH
SPIKE ON ON OFF
@Set plot spikes on/off:  trace 1 = ON; trace 2 = ON; trace 3 = OFF
BAR ON OFF ON
@Set plot bars on/off:  trace 1 = ON; trace 2 = OFF; trace 3 = ON
BAR FILL ON OFF ON
@Set plot bar fills on/off:  trace 1 = ON; trace 2 = OFF; trace 3 = ON
BAR WIDTH .5 .5 .5
@Set plot bar widths:  trace 1 = .5; trace 2 = .5; trace 3 = .5
BAR DIMENSION 2 3 3
@Set plot bar dimension:  trace 1 = 2; trace 2 = 3; trace 3 = 3
BAR FILL ONTS ONTS OFF
@Set 3-d bar fill patterns:  trace 1 = ONTS; trace 2 = ONTS; trace 3 = OFF
 @Note--ONTS means fill is ON for Top & Side of 3-d bars
 
**comment ------start of page 5----------
**newpage
**move 0325 2795
**box 0300 0400 2300 2797 0.0
 
**revvideo 0304 1000 1.0
2. Graphics (Continued)
 
**revvideo 0304 0850 0.9
2.4 Annotating Plots
 
TITLE DEFECTIVE SPRINGS ANALYSIS
@Set title to DEFECTIVE SPRINGS ANALYSIS
Y1LABEL PERCENT DEFECTIVE
@Set left vertical axis label to PERCENT DEFECTIVE
X1LABEL STEEL TEMPERATURE
@Set first horizontal axis label to STEEL TEMPERATURE
LET STRING S1 = JANUARY; X2LABEL DATE = ^S1
@Define string S1 to be JANAUARY; then set second horizontal
 @axis label to   MONTH = <contents of S1 (namely, JANAURY)>
Y1LABEL AUTOMATIC
@Set left vertical axis label so that it will automatically display
 @the contents of the first argument of the PLOT command
 @(e.g., PLOT PRESSURE TEMPERATURE will automatically yield
 @PRESSURE as the left vertical axis label)
X1LABEL AUTOMATIC
@Set first horizontal axis label so that it will automatically display
 @the contents of the second argument of the PLOT command
 @(e.g., PLOT PRESSURE TEMPERATURE will automatically yield
 @TEMPERATURE as the first horizontal axis label)
X3LABEL AUTOMATIC
@Set third horizontal axis label so that it will automatically display
 @the entire command line which generated the plot
 @(e.g., NORMAL PROBABILITY PLOT TEMPERATURE will automatically yield
 @NORMAL PROBABILITY PLOT TEMPERATURE as the third horizontal axis label)
 
**revvideo 0304 0850 0.9
2.5 Modifying Plots
 
XLIMITS 0 3
@Set (for future plots) the horizontal axis limits to 0 and 3
YLIMITS 50 90
@Set vertical axis plot limits to 50 and 90
YLIMITS
@Set vertical axis plot limits back to default (= neat and float with the dat)
 
LEGEND 1 LABORATORY 3
@Define legend 1 to be LABORATORY 3
LEGEND 1 COORDINATES 20 80
@Define plot coordinates for legend 1 to be 20% over and 80% up page
 
GRID ON
@Set grid lines on plots to be automatically ON
 
XLOG ON
@Set log scale for horizontal axis to be ON
YLOG ON
@Set log scale for vertical axis to be ON
LOGLOG ON
@Set log scale for both axes to be ON
LOGLOG OFF
@Set log scale for both axes to be OFF (& therefore revert to linear scale)
 
TIC OFFSET UNITS SCREEN
@Set corner tic units to screen (0 to 100) (as opposed to data)
TIC OFFSET 5 5
@Set offset for first and last tics to be 5% from corner
 
**revvideo 0304 0850 0.9
2.6 Multiple Plots per Page
 
MULTIPLOT 2 2
@Have next 2 x 2 = 4 plots automatically appear on the same page
   PLOT Y X1; PLOT Y X2; PLOT Y X3; PLOT Y X4
@Generate 4 separate plots
MULTIPLOT OFF
@Revert back to 1 plot per page
 
TITLE HW 5 3
@Set height/width of plot title to be 5%/3% (of total screen height)
LABEL HW 4 2
@Set height & width of all plot labels to be 4% and 2%
TIC LABEL HW 4 2
@Set height & width of tic labels on plots to be 4% and 2%
 
**comment ------start of page 6----------
**newpage
**move 0325 2795
**box 0300 0400 2300 2797 0.0
 
**revvideo 0304 1000 1.0
2. Graphics (Continued)
 
**revvideo 0304 0850 0.9
2.7 Word Charts
 
ERASE
@Erase the screen (and remain in graphics mode)
 @ (as opposed to ER which would erase screen but go to text mode)
MOVE 50 50
@Move 50% across the screen & 50% up the screen (therefore, go to center of s)
JUSTIFICATION CENTER
@Set justification for future text to be center-justified
FONT TRIPLEX
@Set font for future text to be triplex
HW 10 7
@Set height & width of future text to be 10% and 7% (of screen height & width)
TEXT ABC
@Write out the text string ABC
BOX 40 40 60 60
@Draw a box with one corner at (40%,40%) and other corner at (60%,60%)
CIRCLE 40 40 60 60
@Draw a circle with one circumference point at (40%,40%) and opposing point a)
TEXT LC()ESUP()PI()I
@Write out the text string (lower case) e with superscript Greek pi and i
 
**revvideo 0304 0850 0.9
2.8 Diagrams
 
LINE 20 20 80 80
@Draw a line segment from (20%,20%) to (80%,80%)
ARROW 20 50 60 50
@Draw an arrow with tail at (20%,50%) and head at (60%,50%)
HW 3 1
@Set height and width to 3% and 1% (of total screen height and width)
AMPLIFIER 30 50 70 50
@Draw an amplifier with mid-side at (30%,50%) and with tip at (70%,50%)
 
**revvideo 0304 0850 0.9
2.9 Capturing Graphics
 
PP
@Send the current contents of the screen to a (Postscript) laser printer
 @(this is an important and heavily used command)
 
DEVICE 2 POSTSCRIPT
@Activate a secondary output device (a file) and specify format to be postscript
   PLOT SIN(X) FOR X = 0 .1 6
@Plot the sin function--it will appear on screen and also to to secondaty deve
EXIT
@Exit from Dataplot
(from DOS) PRINT DPPL1F.DAT
@Send the file DPPL1F.DAT to the postscript laser printer
 
**comment ------start of page 7----------
**newpage
**move 0325 2795
**box 0300 0400 2300 2797 0.0
 
**revvideo 0304 1000 1.0
3. Mathematics
 
LET Y = PRIME NUMBERS FOR I = 1 1 100
@Create a variable Y consisting of the first 100 primes
 
LET Y2 = LOG(Y)
@Create Y2 equal to the natural log of Y
 @(if Y is a parameter (= scalar), then Y2 will be a parameter (= scalar));
 @if Y is a variable (= vector), then Y2 will be a variable (= vector))
 
LET FUNCTION F =SIN(X**2)*EXP(-A*X)
@Create the function (= character string) F containing the following
 @19 characters:  SIN(X**2)*EXP(-A*X)
 @(F may be used directly in future operations, such as:
 @     transformations:  @LET A = 1; LET Y = F
 @     plotting:  @PLOT F for X = 0 .1 10
 @     fitting:  @FIT Y = F
 @     root-finding:  @LET R = ROOTS F WRT X FOR X = 0 TO 10
 @     analytic differentiation:  @LET FUNCTION F2 = DERIVATIVE F WRT X
 
LINEAR FIT Y X
@Perform least squares linear fit of Y on X (pred. val. => PRED, residuals => RES)
   PLOT Y PRED VERSUS X
@Generate a plot with 2 traces: Y versus X and PRED versus X
   PLOT RES X
@Generate a plot of residuals versus X (ideal: structureless)
 
QUARTIC FIT Y X
@Perform least squares quartic fit of Y on X (pred. val. => PRED, res. => RES)
   PLOT Y PRED VERSUS X
@Generate a plot with 2 traces: Y versus X and PRED versus X
   PLOT RES X
@Generate a plot of residuals versus X (ideal: structureless)
 
FIT Y X1 X2 X3
@Perform least sq. fit of Y on X1, X2, and X3 (pred. val. => PRED, res. => RES)
   PLOT Y PRED VERSUS X1
@Generate a plot with 2 traces: Y versus X1 and PRED versus X1
   PLOT Y PRED VERSUS X2
@Generate a plot with 2 traces: Y versus X2 and PRED versus X2
   PLOT Y PRED VERSUS X3
@Generate a plot with 2 traces: Y versus X3 and PRED versus X3
 
FIT Y = EXP(-ALPHA*X)/(A+B*X)
@Perform least squares non-linear fit of Y on X (pred. val. => PRED, res. => RES)
   PLOT Y PRED VERSUS X
@Generate a plot with 2 traces: Y versus X and PRED versus X
   PLOT RES X
@Generate a plot of residuals versus X (ideal: structureless)
 
FILTER WIDTH 7
@Set width of smoothing window to be 7 observations (default = 3 observations)
   SMOOTH Y
@Perform (moving average) smooth of Y (smoothed => PRED, residuals => RES)
   LET N = NUMBER Y
@Determine number of elements in the variable Y; place answer in parameter N
   LET X = SEQUENCE 1 1 N
@Create the variable X with the sequence 1, 2, 3, ..., N
   PLOT Y PRED VERSUS X
@Generate a plot with 2 traces: Y versus X and PRED versus X
   PLOT RES X
@Generate a plot of residuals (= high-frequency component) versus X
 
FILTER WIDTH 11
@Set width of smoothing window to be 11 observations (default = 3 observations)
   CUBIC SMOOTH Y
@Perform (least squares cubic) smooth of Y (smoothed => PRED, res. => RES)
   LET N = NUMBER Y
@Determine number of elements in the variable Y; place answer in parameter N
   LET X = SEQUENCE 1 1 N
@Create the variable X with the sequence 1, 2, 3, ..., N
   PLOT Y PRED VERSUS X
@Generate a plot with 2 traces: Y versus X and PRED versus X
   PLOT RES X
@Generate a plot of residuals (= high-frequency component) versus X
 
LOWESS PROPORTION 10
@Set width of smoothing window to be 10% of total data width (default = 5%)
   LOWESS SMOOTH Y X
@Perform (LOWESS) smooth of Y on X (smoothed => PRED, residuals => RES)
   PLOT Y PRED VERSUS X
@Generate a plot with 2 traces: Y versus X and PRED versus X
   PLOT RES X
@Generate a plot of residuals (= high-frequency component) versus X
 
LET A = INTEGRAL Y X
@Compute the (trapezoid-rule) integral of Y versus X, place result in parameA
 
LET Y2 = INTERPOLATE Y X X2
@Perform cubic-spline interpolation of Y on X (evaluate at X2, resultant into Y2)
  PLOT Y VERSUS X   Y2 VERSUS X2
@Generate a plot with 2 traces: Y versus X and Y2 versus X2
 
LET Y2 X2 = FFT Y X
@Perform a fast Fourier transform of the complex data in Y (real) & X (imag.)
 @the real and imaginary output are placed in Y2 and X2
LET Y2 X2 = INVERSE FFT Y X
@Perform inv. fast Fourier transform of the complex data in Y (real) and X (imag.)
 @the real and imaginary output are placed in Y2 and X2
 
 
**comment ------start of page 8----------
**newpage
**move 0325 2795
**box 0300 0400 2300 2797 0.0
 
**revvideo 0304 1000 1.0
4. Probability
 
LET P = NORCDF(2)
@Create parameter P = normal N(0,1) cum. dist. function evaluated at x = 2
LET P = TCDF(2,8)
@Create parameter P = t(nu=8) cum. dist. function evaluated at x = 2
LET P = CHSCDF(2,8)
@Create parameter P = chi-squared(nu=8) cum. dist. function eval. at x = 2
LET P = FCDF(2,8,11)
@Create parameter P = F(nu1=8,nu2=11) cum. dist. function eval. at x = 2
LET P = WEICDF(2,8)
@Create parameter P = Weibull(shape=8) cum. dist. function eval. at x = 2
LET P = WALCDF(2,8)
@Create parameter P = Wald(shape=8) cum. dist. function eval. at x = 2
LET P = FLCDF(2,8)
@Create parameter P = Fatigue Life(shape=8) cum. dist. func. eval. at x = 2
LET P = IGCDF(2,8)
@Create parameter P = Inverse Gaussian(shape=8) cum. dist. func. eval. at x = 2
LET P = RIGCDF(2,8)
@Create parameter P = Recip. Inv. Gaus.(shape=8) cum. dist. func. at x = 2
 
LET X = NORPPF(.975)
@Create parameter P = normal N(0,1) percent point function evaluated at p = .975
LET X = TPPF(.975,8)
@Create parameter P = t(nu=8) percent point function eval. at p = .975
LET X = CHSPPF(.975,8)
@Create parameter P = chi-squared(nu=8) percent point function eval. at p = .975
LET X = FPPF(.975,8,11)
@Create parameter P = F(nu1=8,nu2=11) percent point function eval. at p = .975
LET P = WEIPPF(2,8)
@Create parameter P = Weibull(shape=8) percent point function eval. at p = .975
LET P = WALPPF(2,8)
@Create parameter P = Wald(shape=8) percent point function eval. at p = .975
LET P = FLPPF(2,8)
@Create parameter P = Fatigue Life(shape=8) percent point func. eval. at p = .975
LET P = IGPPF(2,8)
@Create parameter P = Inv. Gaus.(shape=8) percent point func. eval. at p = .975
LET P = RIGCDF(2,8)
@Create parameter P = Recip. Inv. Gaus.(shape=8) percent point func. eval. at p = .975
 
PLOT NORPDF(X) FOR X = -3 .1 3
@Plot the normal N(0,1) density function eval. at x = -3, at increments ,3
PLOT NORPDF((X-100)/10) FOR X = 70 1 130
@Plot the normal N(100,10) density function evaluated at x = 70 (1) 130
PLOT WEIPDF(X,8) FOR X = .1 .1 30
@Plot the Weibull (shape=8) density function for x = .1 (.1) 30
PLOT WEIPDF((X-100)/10,8) FOR X = 101 1 200
@Plot the Weibull (min=100,scale=10,shape=8) dens. func. for x = 101 (1) 200
PLOT TPDF(P,3) FOR X = -10 .1 10
@Plot the t(nu=3) distribution dens. func. for x = -10 (.1) 10
 
PLOT TPDF(P,1) FOR X = -10 .1 10 AND
@Generate a plot with 3 traces: the t(nu=1) prob density function,
PLOT TPDF(P,2) FOR X = -10 .1 10 AND
@the t(nu=2) probability density function, and
PLOT TPDF(P,3) FOR X = -10 .1 10
@the t(nu=3) probability density function
 
PLOT TPPF(P,3) FOR P = .01 .01 .99
@Plot the t(nu=3) distribution percent points for p = .01 (.01) .99
PLOT TPPF(.975,NU) FOR NU = 1 1 100
@Plot the 97.5 percent point of the t dist. for nu = 1 (1) 100
 
LET Y = UNIFORM RANDOM NUMBERS FOR I = 1 1 20
@Create variable Y with 20 random numbers from uniform N[0,1] distribution
LET Y = NORMAL RANDOM NUMBERS FOR I = 1 1 20
@Create variable Y with 20 random numbers from normal N(0,1) distribution
 
LET Y = NORMAL RANDOM NUMBERS FOR I = 1 1 20
@Create variable Y with 20 random numbers from normal N(0,1) distribution
   LET Y2 = 100+10*Y
@Create variable Y2 with 20 random numbers from normal N(100,10) distribution
 
LET Y = CAUCHY RANDOM NUMBERS FOR I = 1 1 100
@Create variable Y with 20 random numbers from Cauchy C(0,1) distribution
 
LET NU = 5
   LET Y = T RANDOM NUMBERS FOR I = 1 1 40
@Create variable Y with 40 random numbers from t distribution with nu = 5
 
LET NU = 5
   LET Y = CHI-SQUARED RAND NUMB FOR I = 1 1 40
@Create variable Y with 40 random numbers from chi-squared dist. with nu = 5
 
LET NU1 = 5; LET NU2 = 8
LET Y = F RANDOM NUMBERS FOR I = 1 1 40
@Create variable Y with 40 random numbers from F dist. with nu1 = 5 & nu2 = 8
 
LET GAMMA = 4
   LET Y = WEIBULL RANDOM NUMBERS FOR I = 1 1 50
@Create variable Y with 50 random numbers from Weibull dist. with shape par. = 4
 
LET GAMMA = 4
   LET Y = WEIBULL RANDOM NUMBERS FOR I = 1 1 50
@Create variable Y with 50 random numbers from Weibull dist. with shape par. = 4
   LET Y2 = 100+10*Y
@Create variable Y2 with 50 random numbers from Weib. dist. with shape par. = 4
 @& with minimum = 100, and with scale = 10
 
LET RUNSEQ = RANDOM PERMUTATION FOR I = 1 1 8
@Create variable RUNSEQ with a random permutation of 1 to 8
 
**comment ------start of page 9----------
**newpage
**move 0325 2795
**box 0300 0400 2300 2797 0.0
 
**revvideo 0304 1000 1.0
5. Statistics
 
**revvideo 0304 0850 0.9
5.1 Univariate Analysis
 
4-PLOT Y
@Generate 4-plot analysis (run sequence plot, lag plot, histogram,
 @normal probability plot of the data in the variable Y
 
PLOT Y
@Generate a plot of Y versus the dummy index 1, 2, 3, ...
LAG PLOT Y
@Generate a lag plot (with lag = 1) of the data in the variable Y
 @vertical axis = Y(i); horizontal axis = Y(i-1)
HISTOGRAM Y
@Generate a histogram of Y
STEM-AND-LEAF DIAGRAM Y
@Generate a stem-and-leaf diagram of the data in the variable Y
 
NORMAL PROBABILITY PLOT Y
@Generate a normal probability plot of Y (ideal: linear)
 @(to determine if data follows a normal distribution)
 @vertical axis = sorted Y(i); horizontal axis = normal N(0,1) order stat medians
 
UNIFORM PROBABILITY PLOT Y
@Generate a uniform probability plot of Y (ideal: linear)
 @(to determine if data follows a uniform distribution)
 @vertical axis = sorted Y(i); horizontal axis = uniform U[0,1] order stat medians
 
CAUCHY PROBABILITY PLOT Y
@Generate a Cauchy probability plot of Y (ideal: linear)
 @(to determine if data follows a Cauchy distribution)
 @vertical axis = sorted Y(i); horizontal axis = Cauchy C[0,1] order stat medians
 
LOGNORMAL PROBABILITY PLOT Y
@Generate a lognormal probability plot of Y (ideal: linear)
 
HALFNORMAL PROBABILITY PLOT Y
@Generate a halfnormal probability plot of Y (ideal: linear)
 
EXPONENTIAL PROBABILITY PLOT Y
@Generate a exponential probability plot of Y (ideal: linear)
 
EV1 PROBABILITY PLOT Y
@Generate a extreme value type 1 probability plot of Y (ideal: linear)
 
LET NU = 6; T PROB PLOT Y
@Generate t (with tail parameter = 6) probability plot (ideal: linear)
 @(to determine if data follows a t distribution with shape = 6)
 @vertical axis = sorted Y(i); horizontal axis = t t(0,1,6) order stat medians
 
LET GAMMA = 2.5; WEIBULL PROB PLOT Y
@Generate Weibull (with shape parameter = 2.5) probability plot (ideal: linear)
 @(to determine if data follows a Weibull distribution with shape = 2.5)
 @vertical axis = sorted Y(i); horizontal axis = Weibull W(0,1,2.5) order stat medians
 
PPCC PLOT Y or TUKEY PPCC PLOT Y
@Generate Tukey distribution probability plot correlation coefficient plot
 @(to determine the best-fit member of the Tukey distributional family)
 
T PPCC PLOT Y
@Generate t distribution probability plot correlation coefficient plot
 @(to determine the best-fit member of the t distributional family)
 
EV2 PPCC PLOT Y
@Generate extreme value type 2 distribution prob. plot corr. coef. plot
 @(to determine the best-fit member of the ext. value dist. family)
 
PARETO PPCC PLOT Y
@Generate Pareto type 2 distribution prob. plot corr. coef. plot
 @(to determine the best-fit member of the Pareto dist. family)
 
WEIBULL PPCC PLOT Y
@Generate Weibull distribution probability plot correlation coefficient plot
 @(to determine the best-fit member of the Weibull distributional family
 @with common constraint: minimum = 0)
 
WEIBULL PLOT Y
@Generate Weibull plot of the data in the variable Y
 
INVERSE GAUSSIAN PPCC PLOT Y
@Generate Inverse Gaussian distribution prob. plot corr. coef. plot
 @(to determine the best-fit member of the Inv. Gaus. distributional family)
 
FAILURE TIME PPCC PLOT Y
@Generate Failure Time distribution prob. plot corr. coef. plot
 @(to determine the best-fit member of the Fail. Time distributional family)
 
**comment ------start of page 10----------
**newpage
**move 0325 2795
**box 0300 0400 2300 2797 0.0
 
**revvideo 0304 1000 1.0
5. Statistics (Continued)
 
**revvideo 0304 0850 0.9
5.1 Univariate Analysis (Continued)
 
BOX-COX NORMALITY PLOT Y
@Generate Box-Cox normality plot (to determine best transformation to normali)
 @vertical axis = normality measure; horizontal axis = power transformation in x
 
SUMMARY Y
@Generate battery of location, variation, autocorrelation, & distribution statis
 
LET N = NUMBER Y
@Create parameter N = number of elements in the variable (= vector) Y
LET S = SUM Y
@Create parameter S = sum of all data in the variable Y
LET A = MEAN Y
@Create parameter A = mean of all data in the variable Y
LET A = MEDIAN Y
@Create parameter A = median of all data in the variable Y
LET A = MIDMEAN Y
@Create parameter A = midmean of all data in the variable Y
LET S = STANDARD DEVIATION Y
@Create parameter S = standard deviation of all data in the variable Y
LET V = VARIANCE Y
@Create parameter V = variance of all data in the variable Y
LET SDM = STANDARD DEVIATION OF THE MEAN Y
@Create parameter SDM = standard deviation of the mean of all data in variable Y
LET A = AUTOCORRELATION Y
@Create parameter A = autocorrelation of all data in the variable Y
LET C = CORRELATION X Y
@Create parameter A = correlation of all data in the variables X and Y
 
LET TARGET = 25
@Create parameter TARGET (process target value) as 25
   LET LSL = 22
@Create parameter LSL (process lower spec limit) as 22
   LET USL = 28
@Create parameter USL (process upper spec limit) as 28
   LET A = CPK Y
@Create parameter A = Cpk (capability index) of data in variable Y
 
LET W = DATA 1 1 .9 .9 4 1.2 2 1.5
@Create variable W consisting of 8 numbers (used to define desired weights)
   LET M = WEIGHTED MEAN Y W
@Create parameter M = weighted mean of data in variable Y (with weights in W)
   LET S = WEIGHTED STANDARD DEVIATION Y W
@Create parameter S = weighted st. dev. of data in variable Y (with weights in W)
   LET V = WEIGHTED VARIANCE Y W
@Create parameter V = weighted variance of data in variable Y (with weights in W)
 
TABULATE Y
@Tabulate (= display a table of) distinct values and counts of the variable Y
 
AUTOCORRELATION PLOT Y
@Generate an autocorrelation plot of the data in the variable Y
SPECTRUM Y
@Generate a spectral plot of the data in the variable Y
 
DEMODULATION FREQUENCY .3
@Set demodulation frequency to .3 (cycles per observation)
   COMPLEX DEMODULATION AMPLITUDE PLOT Y
@Generate a complex demod. amplitude plot (at demodulation frequency = .3)
   COMPLEX DEMODULATION PHASE PLOT Y
@Generate a complex demod. phase plot (at demodulation frequency = .3)
 
FILTER WIDTH 7
@Set width of smoothing window to be 7 observations (default = 3 observations)
   SMOOTH Y
@Perform (moving average) smooth of Y (smoothed => PRED, residuals => RES)
   LET N = NUMBER Y
@Determine number of elements in the variable Y; place answer in parameter N
   LET X = SEQUENCE 1 1 N
@Create the variable X with the sequence 1, 2, 3, ..., N
   PLOT Y PRED VERSUS X
@Generate a plot with 2 traces: Y versus X and PRED versus X
   PLOT RES X
@Generate a plot of residuals (= high-frequency component) versus X
 
FILTER WIDTH 11
@Set width of smoothing window to be 11 observations (default = 3 observations)
   CUBIC SMOOTH Y
@Perform (least squares cubic) smooth of Y (smoothed => PRED, res. => RES)
   LET N = NUMBER Y
@Determine number of elements in the variable Y; place answer in parameter N
   LET X = SEQUENCE 1 1 N
@Create the variable X with the sequence 1, 2, 3, ..., N
   PLOT Y PRED VERSUS X
@Generate a plot with 2 traces: Y versus X and PRED versus X
   PLOT RES X
@Generate a plot of residuals (= high-frequency component) versus X
 
LOWESS PROPORTION 25
@Set width of smoothing window to be 25% of total data width (default = 5%)
   LOWESS SMOOTH Y X
@Perform (LOWESS) smooth of Y on X (smoothed => PRED, residuals => RES)
   PLOT Y PRED VERSUS X
@Generate a plot with 2 traces: Y versus X and PRED versus X
   PLOT RES X
@Generate a plot of residuals (= high-frequency component) versus X
 
**comment ------start of page 11----------
**newpage
**move 0325 2795
**box 0300 0400 2300 2797 0.0
 
**revvideo 0304 1000 1.0
5. Statistics (Continued)
 
**revvideo 0304 0850 0.9
5.2 Regression Analysis
 
PLOT Y X  or  PLOT Y VERSUS X  or  PLOT Y VS X
@Plot variable Y versus variable X (to assist in selecting a model)
 
LET C = CORRELATION Y X
@Create parameter C = product moment corr. coef. of Y and X
 @(to assist in measuring linear relatedness of 2 variables)
 
LET C = RANK CORRELATION Y X
@Create parameter C = Spearman's rank moment corr. coef. of Y and X
 @(to assist in measuring relatedness of 2 variables)
 
LINEAR FIT Y X
@Perform least squares linear fit of Y on X (prec. val. => PRED, res. => RES)
   CHARACTERS X BLANK; LINES BLANK SOLID
@Set plot characters for next plots to X and blank; set plot lines to blank ad
   PLOT Y PRED VERSUS X
@Generate a plot with 2 traces: Y versus X and PRED versus X
   PLOT RES X
@Generate a plot of residuals versus X (ideal: structureless)
   NORMAL PROBABILITY PLOT RES
@Generate a normal probability plot of the residuals (ideal: linear)
   PRINT A0 A1 RESSD
@Print intercept, slope, and residuals standard deviation (ideal: near-zero)
 
QUARTIC FIT Y X
@Perform least squares quartic fit of Y on X (pred. val. => PRED, res. => RES)
   PLOT Y PRED VERSUS X
@Generate a plot with 2 traces: Y versus X and PRED versus X
   PLOT RES X
@Generate a plot of residuals versus X (ideal: structureless)
 
FIT Y X1 X2 X3
@Perform least squares fit of Y on X1, X2, & X3 (pred. val. => PRED, res. => RES)
   PLOT Y PRED VERSUS X1
@Generate a plot with 2 traces: Y versus X1 and PRED versus X1
   PLOT Y PRED VERSUS X2
@Generate a plot with 2 traces: Y versus X2 and PRED versus X2
   PLOT Y PRED VERSUS X3
@Generate a plot with 2 traces: Y versus X3 and PRED versus X3
 
FIT Y = EXP(-ALPHA*X)/(A+B*X)
@Perform least squares non-linear fit of Y on X (pred. val. => PRED, res. =>RES)
   PLOT Y PRED VERSUS X
@Generate a plot with 2 traces: Y versus X and PRED versus X
 
WEIGHTS W
@Specify W as the variable containing desired weights for the upcoming fit
FIT Y = EXP(-ALPHA*X)/(A+B*X)
@Perf. weighted lst. sq. non-lin. fit of Y on X (pred. val. => PRED, res. => RES)
 
FIT Y = EXP(-ALPHA*X)/(A+B*X)
@Perf. (unweighted) lst. sq. non-lin. fit of Y on X (pred. val. => PRED, res. => RES)
LET W = BIWEIGHT RES
@Create variable W = (Tukey robust) biweights based on previous unweighted fit
WEIGHTS W
@Specify W as the variable containing desired weights for the upcoming fit
FIT Y = EXP(-ALPHA*X)/(A+B*X)
@Perform robust, weighted least sq. non-linear fit of Y on X
 @(pred. values => PRED, residuals => RES)
 
FIT Y = A1*IND(TAG,1) + B1*IND(TAG,2) + SLOPE*X
@Perform joint fit of 2 data sets with distinct intercepts (A1 and B1)
 @but both having common slope (SLOPE).  IND(.,.) is Dataplot indicator function
 
BOX-COX LINEARITY PLOT Y X
@Generate Box-Cox linearity plot (determine best transf. of X to achieve linearity)
 @vertical axis = linearity measure; horizontal axis = power transformation ix
 
LET KNOTS = DATA 10 30 80
@Create a variable KNOTS consisting of 3 values: 10, 30, and 80
   CUBIC SPLINE FIT Y X KNOTS
@Carry out a cubic spline fit of the data in Y and X with partition points inS
 
LINEAR CORRELATION PLOT Y X TAG
@Generate a linear correlation plot of Y on X
 @(For each distinct value of TAG, extract the subset consisting of
 @the corresponding values of Y and X; compute the correlation coefficient
 @of this subset; plot this correlation coefficient versus the TAG value;
 @repeat for all distinct values of TAG; therefore,
 @vertical axis = subset correlation coefficient; horizontal axis = TAG)
 
LINEAR SLOPE PLOT Y X TAG
@Generate a linear slope plot of Y on X
 @vertical axis = slope from linear fit of subset; horizontal axis = TAG)
 
LINEAR INTERCEPT PLOT Y X TAG
@Generate a linear intercept plot of Y on X
 @vertical axis = intercept from linear fit of subset; horizontal axis = TAG)
 
LINEAR RESSD PLOT Y X TAG
@Generate a linear residual standard deviation plot of Y on X
 @vertical axis = res. s.d. from linear fit of subset; horizontal axis = TAG)
 
 
**comment ------start of page 12----------
**newpage
**move 0325 2795
**box 0300 0400 2300 2797 0.0
 
**revvideo 0304 1000 1.0
5. Statistics (Continued)
 
**revvideo 0304 0850 0.9
5.3 Multi-Factor Analysis
 
BOX PLOT Y X
@Generate Box Plot of Y on X
 
MEAN PLOT Y X
@Generate a mean plot of Y on X
 @(For each distinct value of X, compute the mean of all corresponding
 @values of Y, and then plot the means versus each X)
 
MEDIAN PLOT Y X
@Generate a median plot of Y on X
SD PLOT Y X
@Generate a standard deviation plot of Y on X
 
TABULATE COUNTS Y X
@Tabulate (= display a table of) number of Y values at each distinct value in X
TABULATE MEAN Y X
@Tabulate (= display a table of) mean of Y values at each distinct value in X
TABULATE STANDARD DEVIATION Y X
@Tabulate (= display a table of) stsn. dev. of Y values at each distinct value in X
TABULATE RANGE Y X
@Tabulate (= display a table of) range of Y values at each distinct value in X
 
PLOT Y1 Y2 Y3 Y4 VS X
@Generate multi-trace plot of variables Y1, Y2, Y3, and Y4 (vertically)
 @versus variable X (horizontally)
 
MULTIPLOT 2 3
@Have next 2 x 3 = 6 plots automatically appear on the same page
   LOOP FOR K = 1 1 6
@Initiate a loop with dummy parameter K taking on the values 1, 2, ..., 6
   HISTOGRAM Y^K
@Generate a series of 6 plots: histogram of Y1, histogram of Y2, ..., histogram6
   END LOOP
@Terminate the loop
 
MULTIPLOT 2 3
@Have next 2 x 3 = 6 plots automatically appear on the same page
   LOOP FOR K = 1 1 6
@Initiate a loop with dummy parameter K taking on the values 1, 2, ..., 6
   PLOT Y^K VERSUS X
@Generate a series of 6 plots: Y1 versus X, Y2 versus X, ..., Y6 versus X
   END LOOP
@Terminate the loop
 
MULTIPLOT 2 3
@Have next 2 x 3 = 6 plots automatically appear on the same page
   LOOP FOR K = 1 1 6
@Initiate a loop with dummy parameter K taking on the values 1, 2, ..., 6
   LOOP FOR L = 1 1 6
@Initiate a loop with dummy parameter L taking on the values 1, 2, ..., 6
   PLOT Y^K VERSUS Y^L
@Generate a series of 36 plots: Y1 versus X, Y2 versus X, ..., Y6 versus X
   END LOOP
@Terminate the loop for L
   END LOOP
@Terminate the loop for K
 
ANOVA Y X1 X2 X3 X4 X5
@Carry out a (balanced) Analysis of Variance of Y on X1, X2, X3, X4, X5
MEDIAN POLISH Y X1 X2 X3 X4 X5
@Carry out a (balanced) (Tukey) Median Polish of Y on X1, X2, X3, X4, X5
 
BLOCK PLOT Y X2 X1
@Generate a block plot of Y on X2 and X1
 @vertical axis = Y; horizontal axis = X2; plot character = X1
BLOCK PLOT Y X2 X3 X4 X1
@Generate a block plot of Y on X2, X3, X4 and X1
 @vertical axis = Y; horizontal axis = X2 x X3 x X4; plot character = X1
 
**comment ------start of page 13----------
**newpage
**move 0325 2795
**box 0300 0400 2300 2797 0.0
 
**revvideo 0304 1000 1.0
5. Statistics (Continued)
 
**revvideo 0304 1000 0.9
5.4 Multi-Response (Multivariate) Analysis
 
MULTIPLOT 2 2
@Have next 2 x 2 = 4 plots automatically appear on the same page
   PLOT Y1
@Generate the 1st plot: Y1 versus the dummy index 1, 2, 3, ...
   PLOT Y2
@Generate the 2nd plot: Y2 versus the dummy index 1, 2, 3, ...
   PLOT Y1 Y2
@Generate the 3rd plot: Y1 versus Y2
MULTIPLOT OFF
@Revert back to 1 plot per page
 
BIHISTOGRAM Y1 Y2
@Generate a bihistogram of the data in Y1 and Y2
QUANTILE-QUANTILE PLOT Y1 Y2
@Generate a quantile-quantile plot of the data in Y1 and Y2
 
CROSS-TAB Y1 Y2
@Tabulate (= display a table of) distinct values & counts of the variables Y1 & Y2
T TEST Y1 Y2
@Carry out a t-test for the 2 variables Y1 and Y2
 
LET N = NUMBER Y1
@Create parameter N = number of elements in the variable (= vector) Y
LET X = SEQUENCE 1 1 N
@Create variable X with elements 1, 2, 3, 4, ..., N
PLOT Y1 Y2 Y3 Y4 Y5 Y6 Y7 Y8 VERSUS X
@Generate multi-trace plot of Y1, Y2, ..., Y8 versus X
 
MULTIPLOT 2 4
@Have next 2 x 4 = 8 plots automatically appear on the same page
   LOOP FOR K = 1 1 8
@Initiate a loop with dummy parameter K taking on the values 1, 2, ..., 8
   HISTOGRAM Y^K
@Generate a series of 8 plots: histogram of Y1, histogram of Y2, ..., histogram Y8
   END LOOP
@Terminate the loop
 
MULTIPLOT 2 4
@Have next 2 x 4 = 8 plots automatically appear on the same page
   LOOP FOR K = 1 1 8
@Initiate a loop with dummy parameter K taking on the values 1, 2, ..., 8
   PLOT Y^K
@Generate a series of 8 plots: Y1 versus i, Y2 versus i, ..., Y8 versus i
   END LOOP
@Terminate the loop
 
STAR PLOT Y1 Y2 Y3 Y4 Y5 Y6 Y7 Y8
@Generate a star plot for the 8 variables Y1 to Y8
PROFILE PLOT Y1 Y2 Y3 Y4 Y5 Y6 Y7 Y8
@Generate a profile plot for the 8 variables Y1 to Y8
 
LET V1 = FIRST PRIN COMP Y1 Y2 Y3 Y4 Y5 Y6 Y7 Y8
@Create the 8-element variable V1 = the 1st prin. comp. for the 8 variables
   LET V2 = SECOND PRIN COMP Y1 Y2 Y3 Y4 Y5 Y6 Y7 Y8
@Create the 8-element variable V2 = the 2nd prin. comp. for the 8 variables
   PLOT V1 V2
@Plot the 1st prin. comp. (vertically) vs. the 2nd prin. comp. (horizontally)
 
LET V1 = FIRST PRIN EIGEN Y1 Y2 Y3 Y4 Y5 Y6 Y7 Y8
@Create the 8-element variable V1 = the 1st prin. eigenvalue for the 8 variables
   LET V2 = SECOND PRIN EIGEN Y1 Y2 Y3 Y4 Y5 Y6 Y7 Y8
@Create the 8-element variable V2 = the 2nd prin. eigenvalue for the 8 variables
 
**comment ------start of page 14----------
**newpage
**move 0325 2795
**box 0300 0400 2300 2797 0.0
 
**revvideo 0304 1000 1.0
5. Statistics (Continued)
 
**revvideo 0304 0850 0.9
5.5 Time Series Analysis
 
4-PLOT Y
@Generate 4-plot analysis (run sequence plot, lag plot, histogram,
 @normal probability plot of the data in the variable Y
 
 
PLOT Y
@Plot the data in the variable Y (versus the dummy index 1, 2, 3, ...)
 
LAG PLOT Y
@Generate a lag plot (with lag = 1) of the data in the variable Y
 @vertical axis = Y(i); horizontal axis = Y(i-1)
LAG 2 PLOT Y
@Generate a lag plot (with lag = 2) of the data in the variable Y
 @vertical axis = Y(i); horizontal axis = Y(i-2)
LAG 12 PLOT Y
@Generate a lag plot (with lag = 12) of the data in the variable Y
 @vertical axis = Y(i); horizontal axis = Y(i-12)
 
AUTOCORRELATION PLOT Y
@Generate an autocorrelation plot of the data in the variable Y
 @vertical axis = autocorr. of Y(i) and Y(i+h); horizontal axis = lag h
SPECTRUM Y
@Generate a spectral plot of the data in the variable Y
 @vertical axis = variance component; horizontal axis = frequency (0 to .5 cyc)
 
LOGLOG
@Set both axes to be log scale for future plots
  ALLAN VARIANCE
@Generate an Allan variance plot of the data in the variable Y
 @vertical axis = Allan variance; horizontal axis = subset size
 
DEMODULATION FREQUENCY .3
@Set demodulation frequency to .3 (cycles per observation)
   COMPLEX DEMODULATION AMPLITUDE PLOT Y
@Generate a complex demod. amplitude plot (at demodulation frequency = .3)
   COMPLEX DEMODULATION PHASE PLOT Y
@Generate a complex demod. phase plot (at demodulation frequency = .3)
 
 
LET Y2 X2 = FFT Y
@Perform a fast Fourier transform of the real data in Y
 @the real and imaginary output are placed in Y2 and X2
LET Y2 X2 = FFT Y X
@Perform a fast Fourier transform of the complex data in Y and X
 @the real and imaginary output are placed in Y2 and X2
LET Y2 X2 = INVERSE FFT Y X
@Perform an inverse fast Fourier transform of the complex data in Y (real) and X
 @the real and imaginary output are placed in Y2 and X2
 
LET Y2 X2 = FOURIER TRANSFORM Y
@Perform a (slow) Fourier transform of the real data in Y
 @the real and imaginary output are placed in Y2 and X2
LET Y2 X2 = FOURIER TRANSFORM Y X
@Perform a (slow) Fourier transform of the complex data in Y and X
 @the real and imaginary output are placed in Y2 and X2
LET Y2 X2 = INVERSE FOURIER TRANSFORM Y X
@Perform an inv. (slow) Fourier transform of complex data in Y (real) & X (imag.)
 @the real and imaginary output are placed in Y2 and X2
 
FILTER WIDTH 7
@Set width of smoothing window to be 7 observations (default = 3 observations)
   SMOOTH Y
@Perform (moving average) smooth of Y (smoothed => PRED, residuals => RES)
   LET N = NUMBER Y
@Determine number of elements in the variable Y; place answer in parameter N
   LET X = SEQUENCE 1 1 N
@Create the variable X with the sequence 1, 2, 3, ..., N
   PLOT Y PRED VERSUS X
@Generate a plot with 2 traces: Y versus X and PRED versus X
   PLOT RES X
@Generate a plot of residuals (= high-frequency component) versus X
 
FILTER WIDTH 11
@Set width of smoothing window to be 11 observations (default = 3 observations)
   CUBIC SMOOTH Y
@Perform (least squares cubic) smooth of Y (smoothed => PRED, res. => RES)
   LET N = NUMBER Y
@Determine number of elements in the variable Y; place answer in parameter N
   LET X = SEQUENCE 1 1 N
@Create the variable X with the sequence 1, 2, 3, ..., N
   PLOT Y PRED VERSUS X
@Generate a plot with 2 traces: Y versus X and PRED versus X
   PLOT RES X
@Generate a plot of residuals (= high-frequency component) versus X
 
LOWESS PROPORTION 25
@Set width of smoothing window to be 25% of total data width (default = 5%)
   LOWESS SMOOTH Y X
@Perform (LOWESS) smooth of Y on X (smoothed => PRED, residuals => RES)
   PLOT Y PRED VERSUS X
@Generate a plot with 2 traces: Y versus X and PRED versus X
   PLOT RES X
@Generate a plot of residuals (= high-frequency component) versus X
 
 
**comment ------start of page 15----------
**newpage
**move 0325 2795
**box 0300 0400 2300 2797 0.0
 
**revvideo 0304 1000 1.0
5. Statistics (Continued)
 
**revvideo 0304 0850 0.9
5.6 Uncertainty Analysis
 
JACKNIFE MEAN PLOT Y
@Generate a jacknife mean plot for the data in the variable Y
 @vert. axis = mean for i-th jacknife sample; hor. axis = bootstrap)
JACKNIFE MEDIAN PLOT Y
@Generate a jacknife median plot for the data in the variable Y
 @vert. axis = median for i-th jacknife sample; hor. axis = bootstrap)
 
BOOTSTRAP MEAN PLOT Y
@Generate a bootstrap mean plot for the data in the variable Y
 @vert. axis = mean for i-th bootstrap sample; hor. axis = bootstrap)
BOOTSTRAP MEDIAN PLOT Y
@Generate a bootstrap median plot for the data in the variable Y
 @vert. axis = median for i-th bootstrap sample; hor. axis = bootstrap)
BOOTSTRAP MIDMEAN PLOT Y
@Generate a bootstrap midmean plot for the data in the variable Y
BOOTSTRAP RANGE PLOT Y
@Generate a bootstrap range plot for the data in the variable Y
BOOTSTRAP SD PLOT Y
@Generate a bootstrap sd plot for the data in the variable Y
 
BOOTSTRAP LINEAR CORRELATION PLOT Y X
@Generate a bootstrap linear correlation plot for Y and X
BOOTSTRAP LINEAR INTERCEPT PLOT Y X
@Generate a bootstrap linear intercept plot for Y and X
BOOTSTRAP LINEAR SLOPE PLOT Y X
@Generate a bootstrap linear slope plot for Y and X
BOOTSTRAP LINEAR RESSD PLOT Y X
@Generate a bootstrap linear residual s.d. plot for Y and X
 
LET NUMBOOT = 100
@Create parameter NUMBOOT and set = 100
   LET N = NUMBER Y
@Create parameter N = number of elements in the variable Y
   LOOP FOR K = 1 1 NUMBOOT
@Initiate a loop with dummy parameter K taking on the values 1, 2, ..., NUMBOOT
   LET IND = BOOTSTRAP INDEX FOR I = 1 1 N
@Create a variable IND = random sample with replacement of 1, 2, 3, ..., N
   LET Y2 = BOOTSTRAP SAMPLE Y IND
@Create a variable Y2 = bootstrap sample of Y based on current contents of IND
   WEIBULL PLOT Y2
@Generate a Weibull Plot of Y2 (which also computes B99 = est. 99 percent point)
   LET STAT(K) = B99
@Place the current value of B99 into the k-th element of the variable (vectorT
   END LOOP
@Terminate the loop
   LET MEANSTAT = MEAN STAT
@Generate the parameter MEANSTAT = mean of the B99 values in variable STAT
   LET SDSTAT = STANDARD DEVIATION STAT
@Generate the parameter SDSTAT = stan. dev. of the B99 values in variable STAT
   X2LABEL MEAN(B99) = ^MEANSTAT
@Set 2nd horiz. axis label for upcoming histogram to the computed mean B99
   X3LABEL SD(B99) = ^SDSTAT
@Set 3rd horiz. axis label for upcoming histogram to the computed s.d. of B99
   HISTOGRAM STAT
@Generate a histogram of the B99 values in the variable STAT
 
**comment ------start of page 16----------
**newpage
**move 0325 2795
**box 0300 0400 2300 2797 0.0
 
**revvideo 0304 1000 1.0
6. Design of Experiment
 
**revvideo 0304 0850 0.9
6.1 Organization
 
LIST DEXSUMSH.TEX
@List the Exp. Design Summary Sheet file DEXSUMSH.TEX to the screen
DOS PRINT DEXSUMSH.TEX
@Send the Exp. Design Summary Sheet file DEXSUMSH.TEX to laser printer
 
LIST DEXSUMS2.TEX
@List the Exp. Design Summary Sheet file DEXSUMS2.TEX to the screen
DOS PRINT DEXSUMS2.TEX
@Send the Exp. Design Summary Sheet file DEXSUMS2.TEX to laser printer
 
**revvideo 0304 0850 0.9
6.2 Construction
 
LIST DEXPLANS.TEX
@List the Dataplot file DEXPLANS.TEX onto the screen
 @DEXPLANS.TEX contains a list of all experimental plans available in Dataplot
LIST DEXRES4.TEX
@List (onto screen) the file DEXRES4.TEX containing instructions
 @for resolution 4 designs for numbers of factors = 3, 4, 5, 6, ..., 11, 15, 2
 
SKIP 25; READ LATSQ4.DAT X1 X2 X3
@Read the Exp. Des. 4-by-4 Latin Square design file into Dataplot
SKIP 25; READ 2TO3.DAT X1 X2 X3
@Read the Exp. Des. 2**3 full factorial design file into Dataplot
SKIP 25; READ 2TO8M4.DAT X1 X2 X3 X4 X5 X6 X7 X8
@Read the Exp. Des. 2**(8-4) fractional fact. design file into Dataplot
SKIP 25; READ L8.DAT X1 X2 X3 X4 X5 X6 X7
@Read the Exp. Des. Taguchi L8 design file into Dataplot
SKIP 25; READ PLACBURM.DAT X1 TO X11
@Read the Exp. Des. Plackett-Burman design file into Dataplot
 
LET X1 = PATTERN -1 +1 FOR I = 1 1 8
@Create the variable X with 8 elements: -1 +1 -1 +1 -1 +1 -1 +1
LET X1 = PATTERN 1 2 FOR I = 1 1 8
@Create the variable X1 with 8 elements: 1 2 1 2 1 2 1 2
LET X2 = DATA 1 1 2 2 1 1 2 2
@Create the variable X with 8 elements: 1 1 2 2 1 1 2 2
 
LET RUNSEQ = RANDOM PERMUTATION FOR I = 1 1 8
@Create a variable RUNSEQ consisting of a random permutation of 1 to 8
 
**comment ------start of page 17----------
**newpage
**move 0325 2795
**box 0300 0400 2300 2797 0.0
 
**revvideo 0304 1000 1.0
6. Design of Experiment (Continued)
 
**revvideo 0304 0850 0.9
6.3 Analysis
 
LIST DEXEXAMP.TEX
@List the Dataplot file DEXEXAMP.TEX onto the screen
 
 @DEXEXAMP.TEX contains a list of all experimental design data sets available
LIST BOXSPRIN.DAT
@List onto screen the contents of Dataplot file BOXSPRIN.DAT
   SKIP 25; READ BOXSPRIN.DAT Y X1 X2 X3
@Read into Dataplot the contents of Dataplot file BOXSPRIN.DAT
 
LIST BOXBIKE2.DAT
@List onto screen the contents of Dataplot file BOXBIKE2.DAT
   SKIP 25; READ BOXBIKE2.DAT Y X1 X2 X3 X4 X5 X6 X7
@Read into Dataplot the contents of Dataplot file BOXSPRIN.DAT
 
PLOT Y RUNSEQ
@Plot the data in variable Y versus the variable RUNSEQ
 
DEX SCATTER PLOT Y X1 TO X7
@Generate a Des. of Exp. Scatter Plot of Y on X1 to X7
DEX MEAN PLOT Y X1 TO X7
@Generate a Des. of Exp. Mean Plot of Y on X1 to X7 (very important)
DEX SD PLOT Y X1 TO X7
@Generate a Des. of Exp. Standard deviation Plot of Y on X1 to X7
DEX TAGUCHI SN0 PLOT Y X1 TO X7
@Gen. a Des. of Exp. Taguchi Sig.-to-Noise ("on target") Plot of Y on X1 to X7
DEX TAGUCHI SN- PLOT Y X1 TO X7
@Gen. a Des. of Exp. Taguchi Sig.-to-Noise ("small is good") Plot of Y on X1 to X7
DEX TAGUCHI SN+ PLOT Y X1 TO X7
@Gen. a Des. of Exp. Taguchi Sig.-to-Noise ("large is good") Plot of Y on X1 to X7
 
READ BOXSPRIN.DAT Y X1 X2 X3
@Read in the 2**3 full factorial data from dataplot file BOXSPIN.DAT
   DEX MEAN PLOT Y X1 TO X3
@Generate a Des. of Exp. Mean Plot of Y on X1 to X3 (very important)
   LET NUMFAC = 3
@Create the parameter NUMFAC (# of factors) as 3 (needed by DEXSTAT2.DP)
   LET DP = 2
@Create the parameter DP (# of decimals places) as 2 (needed by DEXSTAT2.DP)
   LET STRING STAT = MEAN
@Create the string STAT (desired stat) as MEAN (needed by DEXSTAT2.DP)
   YLIMITS 50 90
@ Set plot vertical axis limits as 50 to 90 (needed by DEXSTAT2.DP)
   CALL DEXSTAT2.DP
@Execute the Dataplot macro file DEXSTAT2.DP
 @which carries out a main-effect & 2-factor graphical analysis of effects
 
YATES ANALYSIS Y
@Carry out a Yates analysis of the data in variable Y (assumed to be in Yates)
 @(Note--estimated effects automatically written out to file DPST1F.DAT)
   SKIP 0; READ DPST1F.DAT TAG COEF
@Read Yates analysis estimated effects in from file
   NORMAL PROBABILITY PLOT COEF
@Generate normal probability plot of coefficients
 
ANOVA Y X1 X2 X3
@Carry out a (balanced) Analysis of Variance of Y on X1, X2, and X3
PLOT RES RUNSEQ
@Plot the residuals from a fit versus the variable RUNSEQ
NORMAL PROBABILITY PLOT RES
@Generate a normal probability plot of residuals (ideal = linear)
CALL DEXCUBE.DP
@Execute the Dataplot 2**3 full factorial menu macro DEXCUBE.DP
CALL DEXSQUAR.DP
@Execute the Dataplot 2**2 full factorial menu macro DEXSQUAR.DP
 
**comment ------start of page 18----------
**newpage
**move 0325 2795
**box 0300 0400 2300 2797 0.0
 
**revvideo 0304 1000 1.0
7. Quality and Productivity
 
**revvideo 0304 0850 0.9
7.1 General Q & P
 
4-PLOT Y
@Generate 4-plot analysis (run sequence plot, lag plot, histogram,
 @normal probability plot of the data in the variable Y
 
XBAR CHART Y X
@Generate an Xbar Control Chart of the raw data in Y with tags in X
RANGE CHART Y X
@Generate an Range Control Chart of the raw data in Y with tags in X
 
SUMMARY Y
@Generate battery of location, variation, autocorrelation, & distribution statis
LET M = MEAN Y
@Create parameter A = mean of all data in variable Y
LET S = STANDARD DEVIATION Y
@Create parameter S = standard deviation of all data in variable Y
LET SDM = STANDARD DEVIATION OF THE MEAN Y
@Create parameter SDM = standard deviation of the mean for all data in variable Y
 
CONFIDENCE LIMITS Y
@Compute a table of various confidence limits for mu and sigma for the data inY
 
HISTOGRAM Y
@Generate a histogram of the data in the variable Y
NORMAL PROBABILITY PLOT Y
@Generate a normal probability plot of the data in the variable Y (ideal: lin)
 
LET TARGET = 25
@Create parameter TARGET (process target value) as 25
   LET LSL = 22
@Create parameter LSL (process lower spec limit) as 22
   LET USL = 28
@Create parameter USL (process upper spec limit) as 28
   LET A = CPK Y
@Create parameter A = Cpk (capability index) of data in variable Y
 
**revvideo 0304 0850 0.9
7.2   7 Old Tools
 
CALL ISHIKAWA.DP
@Execute the Dataplot Ishikawa menu macro in file ISHIKAWA.DP
CALL PARETO.DP
@Execute the Dataplot Pareto Diagram menu macro in file PARETO.DP
HISTOGRAM Y
@Generate a histogram of the data in the variable Y
BIHISTOGRAM Y1 Y2
@Generate a bihistogram of the data in Y1 and the data in Y2
PLOT Y X  or  PLOT Y VERSUS X  or  PLOT Y VS X
@Generate a scatter plot of variable Y versus variable X
 
LET TARGET = 25
@Create parameter TARGET (process target value) as 25
   LET LSL = 22
@Create parameter LSL (process lower spec limit) as 22
   LET USL = 28
@Create parameter USL (process upper spec limit) as 28
   XBAR CONTROL CHART Y X
@Generate a xbar (= mean) control chart for the raw data in Y and X
 
RANGE CONTROL CHART Y X
@Generate a range control chart for the raw data in Y and X
 
 
**move 1500 0350
This is DATAPLOT file DPSNAPSH.TEX 10/92 PP 11 & 30 & 2
For questions, contact J. J. Filliben (NIST) 301-975-2855
