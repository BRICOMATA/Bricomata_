Dataplot Software System  (Spring 1995)
James J. Filliben, Statistical Engineering Division, CAML
Alan Herkert, Scientific Computing Environments Division, CAML
 
The Dataplot software system continues to serve as a
communication and application forum for the development and
dissemination of statistical methodology to the scientific and
engineering community.  Some of the more significant enhancements
over the last year are as follows:
 
1. Documentation: Alan Heckert has produced 2 volumes (1000 pages
   & 700 pages) of documentation.  The documentation was produced
   with framemaker.  It is currently in WERB review.
 
2. Probability Distributions: The collection of built-in
   distribution functions (pdf's, cdf's, ppf's, random number
   generation, probability plotting) has been expanded (Alan) and
   now exceeds 30 distributions--exceeding most/all such
   capabilities of commercial packages.
 
3. Matrix Operations: The quality of the underlying matrix
   operations has been enhanced (Alan) by the inclusion of many
   LINPACK and EISPACK routines.
 
4. Special Functions: Over 75 special functions which are
   commonly needed in scientific applications have been added
   (Alan); they were drawn mainly from the SLATEC library.
 
5. Statistics Procedures: Enhancements (Alan & Jim) include:
   DEX PHD (Design of Experiment Principal Hessian Direction) (Li)
   Lowess extended from linear to quadratic
   DDS (Data-Dependent Systems): command & extensive FAQ file (Pandit)
   Lieberman-Miller Regression Tolerance Limits
   Testing & confidence limits for coef. of variation (Vangel)
   Confidence Distribitions
 
6. Data File Augmentations: (Jim) The collection of
   interesting on-line data sets now approximates 400.
 
7. Graphical User Interface:  A Turbo-C GUI for the PC
   is near completion (Jim).  It was demonstrated in the
   SED Software Seminar Series (April 10). Design philosophy
   for the GUI involves many "power" centers:
      Data-centric (easy link to 400 data files)
      Literature-centric (connection to classic stat literature data sets)
      Graphics-centric (graphs usually are better than numbers)
            (e.g., confidence distributions improve confidence limits)
      Problem-centric (some analysts need general problem "roadmaps")
            (thus GUI structured to guide/teach as well as be used)
      Technique-centric (some analysts want specific techniques)
            (thus GUI designed to allow quick access to techniques)
   Further, the GUI menu system has been designed as:
      Extensible/Editable (therefore GUI menus are entirely file-driven)
      Multi-path menu solutions (thus menu files are hypertext)
 
8. Dataplot Links with Other NIST Programs:
   1. HPCC/SIMA
   2. Coatings Consortium
   3. NIST Design of Experiment Workshops
   4. Multimedia Conferencing via SCOOT (Jeff Fong)
   5. SEMATECH Electronic Handbook
